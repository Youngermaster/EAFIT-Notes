\section{Operating Systems Interruptions}

The idea is just research and implementation of some of the operating system interruptions.

\subsection{What is an O.S. Interruption?}

An interrupt is a signal emitted by hardware or software when a process or an event needs immediate
attention. It alerts the processor to a high-priority process requiring interruption of the current
working process.

\subsection{Buffer overflow}

A buffer overflow occurs when a program or process attempts to write more data to a fixed-length
block of memory, or buffer, than the buffer is allocated to hold. Buffers contain a defined amount
of data; any extra data will overwrite data values in memory addresses adjacent to the destination
buffer.

\begin{figure}[h]
    \centering
    \includegraphics[width=\textwidth]{\BufferOverflowDiagram}
    \caption{Buffer overflow diagram, image taken from imperva.com.}
\end{figure}

\vspace{0.2cm}

Now let's test it out with the language C and Rust. So we will have the following files for this
test:

\begin{itemize}
    \item \textit{\textbf{Makefile}}, this file allow us to build or clean the project.
    \item \textit{\textbf{avoid memory leaks.rs}}, this file shows an implementation of how Rust handles the memory leaks.
    \item \textit{\textbf{buffer overflow.c}}, this file shows an implementation of how to do a buffer overflow with C.
    \item \textit{\textbf{get memory leaks.sh}}, this script shows if Valgrind can help us with type
    of problems.
\end{itemize}

\clearpage

Now let's run the following \textit{\textbf{Makefile}} with the command \textit{\textbf{make}}, and
we should get an executable.

\begin{lstlisting}[language=Make, caption=This file allow us to build or clean the project.]
# https://github.com/Youngermaster/ST0257-Operating-Systems/blob/main/Challenges/Challenge_1/BufferOverflow/Makefile

CC=gcc
CFLAGS=-g -Wall
EXE=buffer_overflow
REXE=avoid_memory_leaks

all: $(EXE)

%: %.c
	$(CC) $(CFLAGS) $< -o $@

clean:
	rm -rf *.o $(EXE)

rust:
	rustc $(REXE).rs

cleanRust:
	rm -rf *.o $(REXE)

\end{lstlisting}

The executable is made from the following code, where we are accessing to an outbound position.

\begin{lstlisting}[language=C, caption=This file shows an implementation of how to do a buffer
    overflow with C.]
// https://github.com/Youngermaster/ST0257-Operating-Systems/blob/main/Challenges/Challenge_1/BufferOverflow/buffer_overflow.c

#include <stdio.h>

int main(int argc, char const *argv[]) {
    char *s = "hello world";
    char c = s[20];
    printf("%p - %p -> %d\n", &c, __builtin_frame_address(0), c);
    printf("%p - %p -> %s\n", &s, __builtin_frame_address(0), s);
    printf("%p - %p -> %d\n", &s[20], __builtin_frame_address(0), s[20]);
    return 0;
}
\end{lstlisting}

Now, after the build we run it, and we get the following output:

\begin{figure}[h]
    \centering
    \includegraphics[width=\textwidth]{\BufferOverflowExample}
    \caption{Screenshot of the commands made in the terminal and the results.}
\end{figure}

We get the word \textit{"hello world"}, however, as we can see we got a random number, in this case
the number \textit{45}. And what is the problem with that? The problem is that we are able to access
to a random memory value, due to security issues and/or unexpected behaviours, imageine an airplane
crash due to a little calculus modified by a Buffer Overflow.

\subsubsection{How can we avoid it?}

We have some options, but right now the most common are those two:

\begin{itemize}
    \item We should pay attention to our C programs memory management.
    \item We can use another languages that solves that problem, using \textit{Garbage Collectors},
    \textit{Borrowing and checkers}, etc.
\end{itemize}

Maybe a \textit{Garbage Collector} approach is not a bad idea, however if we want maximum
performance just maybe that is not the right path. Now let's check the \textit{Borrowing and checkers}
option with the language \textit{Rust}. Now let's run the following code with the command
\textit{rustc ourRustFile.rs}


\begin{lstlisting}[language=C++, caption=This file shows an implementation of how Rust handles the
    memory leaks.]
// https://github.com/Youngermaster/ST0257-Operating-Systems/blob/main/Challenges/Challenge_1/BufferOverflow/avoid_memory_leaks.rs

fn main() {
    // Fixed-size array (type signature is superfluous)
    let xs: [i32; 5] = [1, 2, 3, 4, 5];

    // Indexing starts at 0
    println!("first element of the array: {}", xs[0]);

    // ! Rust compilation will break this code inmediately!
    println!("first element of the array: {}", xs[20]);
}
\end{lstlisting}

And there is and interesting output, even if the \textit{C} language allowed us to get a memory
value far from our scope, although \textit{Rust} before compilation does not allow us to compile
\textit{"bad code"}.

\begin{figure}[h]
    \centering
    \includegraphics[width=\textwidth]{\BufferOverflowExampleTwo}
    \caption{Screenshot of the commands made in the terminal and the results of the Rust code.}
\end{figure}

\subsection{Heap Memory Leak}

Memory leak occurs when programmers create a memory in heap and forget to delete it. The
consequences of memory leak is that it reduces the performance of the computer by reducing the
amount of available memory. Eventually, in the worst case, too much of the available memory may
become allocated and all or part of the system or device stops working correctly, the application
fails, or the system slows down vastly.

Now let's test it out with the language C. So we will have the following files for this test:

\begin{itemize}
    \item \textit{\textbf{Makefile}}, this file allow us to build or clean the project.
    \item \textit{\textbf{buffer overflow.c}}, this file shows an implementation of how to do a buffer overflow with C.
    \item \textit{\textbf{get memory leaks.sh}}, this script shows if Valgrind can help us with type
    of problems.
\end{itemize}



\clearpage