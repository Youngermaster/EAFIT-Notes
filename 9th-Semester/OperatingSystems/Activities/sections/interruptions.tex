\section{Operating Systems Interruptions}

The idea is just research and implementation of some of the operating system interruptions.

\subsection{What is and O.S. Interruption?}

An interrupt is a signal emitted by hardware or software when a process or an event needs immediate
attention. It alerts the processor to a high-priority process requiring interruption of the current
working process.

\subsection{Buffer overflow}

A buffer overflow occurs when a program or process attempts to write more data to a fixed-length
block of memory, or buffer, than the buffer is allocated to hold. Buffers contain a defined amount
of data; any extra data will overwrite data values in memory addresses adjacent to the destination
buffer.

\begin{figure}[h]
    \centering
    \includegraphics[width=\textwidth]{\BufferOverflowDiagram}
    \caption{Buffer overflow diagram, image taken from imperva.com}
\end{figure}

\vspace{0.2cm}

Now let's test it out with the language C and Rust. So we will have the following files for this
test:

\begin{itemize}
    \item \textit{\textbf{Makefile}}, this file allow us to build or clean the project.
    \item \textit{\textbf{avoid_memory_leaks.rs}},  this file shows an implementation of how Rust
    handles the memory leaks.
    \item \textit{\textbf{buffer_overflow.c}}, this file shows an implementation of how to do a
    buffer overflow with C.
    \item \textit{\textbf{get_memory_leaks.sh}}, this script shows if Valgrind can help us with type
    of problems.
\end{itemize}

\begin{lstlisting}[language=C, caption=This script allow us to extract nmap generated info]
// https://github.com/Youngermaster/ST0257-Operating-Systems/blob/main/Challenges/Challenge_1/BufferOverflow/buffer_overflow.c

#include <stdio.h>

int main(int argc, char const *argv[]) {
    char *s = "hello world";
    char c = s[20];
    printf("%p - %p -> %d\n", &c, __builtin_frame_address(0), c);
    printf("%p - %p -> %s\n", &s, __builtin_frame_address(0), s);
    printf("%p - %p -> %d\n", &s[20], __builtin_frame_address(0), s[20]);
    return 0;
}
\end{lstlisting}


\clearpage