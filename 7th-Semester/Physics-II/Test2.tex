\documentclass{exam}
\usepackage[utf8]{inputenc}

\begin{document}

\begin{center}
\fbox{\fbox{\parbox{5.5in}{\centering
Escuela de Ingeniería,
Departamento de Ciencias
Parcial 2 - Física II}}}
\end{center}

\textbf{Nombre:} \underline{Juan Manuel Young Hoyos} \enspace\hrulefill

\textbf{Código:} \underline{201810117010} \enspace\hrulefill

\textbf{Nota:\enspace\hrulefill}

    \section*{Solución}

    \section*{Pregunta 1}

    \[ 0 Nm y -NIAB Joules \]

    \section*{Pregunta 2}

    Located in $\rightarrow$ Test2-Question2.xlsx

    \section*{Pregunta 3}

    E

    \section*{Pregunta 4}

    Es la gráfica 4

    \section*{Pregunta 5}

    Longitud $\rightarrow 1.50m$
    R $\rightarrow 10.0 Ohms$

    \section*{Pregunta 6}

    Proceso de Carga: El circuito tiene un transformador elevador, el transformador elevador  
    disminuye la corriente de salida para mantener la potencia de entrada y salida del sistema igual,
    y un rectificador, este es el dispositivo electrónico que permite convertir la corriente alterna en 
    corriente continua.​, que son usados para cargar el condensador $C$. La carga del capacitor está 
    determinada por la tensión del autotransformador variable del circuito primario. 


    Proceso de Descarga: Cuando el operador da la orden de disparo, la llave $S$ pasa a la posición  
    $2$ donde el capacitor $C$ se descarga por medio de la inductancia $L$ y de la resistencia 
    $R_L$, es decir, la resistencia transtoráxica.

    \section*{Pregunta 8}

    $r=\frac{3}{4R}$



    \section*{Pregunta 9}

    Hacia la parte inferior de la hoja

    \section*{Pregunta 10}

    D

\end{document}