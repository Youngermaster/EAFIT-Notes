\documentclass[12pt, letterpaper, twoside]{article}
\usepackage[utf8]{inputenc}
\title{Quizz 1}
\date{Febrero 17 de 2021}
\author{Juan Manuel Young Hoyos \thanks{201810117010}}

\begin{document}
    \maketitle

    \section{}

    Dos láminas de plástico no conductoras, muy grandes, cada una con espesor de 10 cm, 
    tienen densidades de carga uniforme $s_1$, $s_2$, $s_3$ y $s_4$ en sus superficies, como se ilustra en la figura. 
    Estas densidades superficiales de carga tienen los valores $s_1 = -6.00 mC/m^2$, $s_2 = +5.00 mC/m^2$, $s_3 = +2.00 mC/m^2$
    y $s_4 = +4.00 mC/m^2$. Describa cómo plantearía la ley de Gauss para encontrar el campo eléctrico en los puntos A, B y C.
    No requiere llegar a l valor numérico final

    La magnitud y dirección del campo eléctrico mencionados son:

    a) En el punto A:   $EA = 282485.8757 N/m$

    b) En el punto B:    $EB = 508474.5763 N/m$ hacia la izquierda.

    c) En el punto C:    $EC = 56497.17514 N/m$ hacia la izquierda.


   La magnitud y dirección del campo eléctrico se calcula mediante la aplicación de la ley de Gauss, de la siguiente manera

    a)    $E = \Sigma E$

        \[ EA = \sigma1/2\varepsilon o + \sigma2/2\varepsilon o + \sigma3 /2\varepsilon o + \sigma4/2\varepsilon o \]

        \[EA = (1/2\varepsilon o)* ( \sigma1 +\sigma2 +\sigma3+\sigma4) \]

        \[EA=  ( 1/2*8.85*10^-12 C2/N*m^2) * ( -6 + 5 + 2 + 4) *10^-6C/m^2 \]

        \[EA = 282485.8757 N/m \]


    b) \[ EB = \sigma1/2\varepsilon o +(- \sigma2)/2\varepsilon o + \sigma3 /2\varepsilon o + \sigma4/2\varepsilon o \]

        \[ EB = (1/2\varepsilon o)* ( \sigma1 +\sigma3 +\sigma4- \sigma2) \]

        \[ EB=  ( 1/2*8.85*10^-12 C2/N*m^2) * ( 6 + 5 + 2 - 4) *10^-6C/m^2 \]

        \[ EB =  508474.5763 N/m \;hacia\;la\;izquierda. \]


      c) \[ EC =( \mu/2\varepsilon o)* ( \sigma4+\sigma1 -\sigma2+\sigma3 ) \]

        \[ EC = ( 1*10^-6 /2*8.85 *10^-12 C2/Nm^2)* ( -6 + 5 - 2 + 4 ) \]

        \[ EC = 56497.17514 N/m\;hacia\;la\;izquierda. \]



    \section{}

    Dos cargas puntuales q1 = +2.40 nC y q2 = -6.50 nC están separadas 0.100 m. 
    El punto A está a la mitad de la distancia entre ellas; el punto B está a 0.080 m de q1 y a 0.060 m de q2 (figura). 
    Considere el potencial eléctrico como cero en el infinito. Determine:

    a) el potencial en el punto A;

    b) el potencial en el punto B;

    c) el trabajo realizado por el campo eléctrico sobre una carga de 2.50 nC que viaja del punto B al punto A.


\end{document}