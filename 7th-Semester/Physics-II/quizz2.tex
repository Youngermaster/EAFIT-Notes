\documentclass{exam}
\usepackage[utf8]{inputenc}

\begin{document}

\begin{center}
\fbox{\fbox{\parbox{5.5in}{\centering
Escuela de Ingeniería,
Departamento de Ciencias
Parcial 2 - Física II}}}
\end{center}

\textbf{Nombre:} \underline{Juan Manuel Young Hoyos} \enspace\hrulefill

\textbf{Código:} \underline{201810117010} \enspace\hrulefill

\textbf{Nota:\enspace\hrulefill}

    \section*{Solución}

    \section*{Pregunta 1}

    Primero, la fuerza magnética está dada por $\overrightarrow{F} = q \overrightarrow{v} \times \overrightarrow{B}$

    Todas las cargas experimentan una fuerza en $\overrightarrow{B}$, excepto la $c$, esto porque que la carga en el punto
    $c$ no experimenta fuerza magnética, esto porque está en dirección
    opuesta a $\overrightarrow{B}$, lo que al hacer el producto cruz, esto nos debería dar cero ($0$).

    \section*{Pregunta 3}

    Primero, el campo creado por una carga $q$ es:

    \[ \overrightarrow{B} = \frac{u_0}{4\pi} q \frac{\overrightarrow{u} \times \overrightarrow{v}}{r^2} \]
    
    \[ \overrightarrow{B_A} = 0\]
    esto porque $ \overrightarrow{u} \;y\; \overrightarrow{v} $ son paralelos.


\end{document}