\documentclass{exam}
\usepackage[utf8]{inputenc}

\begin{document}

\begin{center}
\fbox{\fbox{\parbox{5.5in}{\centering
Escuela de Ingeniería,
Departamento de Ciencias
Parcial 3 - Estructuras Discretas}}}
\end{center}

\textbf{Nombre:} \underline{Juan Manuel Young Hoyos} \enspace\hrulefill

\textbf{Código:} \underline{201810117010} \enspace\hrulefill

\textbf{Nota:\enspace\hrulefill}

    \section*{Solución}

    \section*{Pregunta 2}

    Sean $A, B, C$ y $D$ conjuntos tales que:
    
    \begin{enumerate}
        \item $A$ y $C$ tienen la misma cardinalidad.
        \item $B$ y $D$ tienen la misma cardinalidad.
    \end{enumerate}
    
    Demostrar que los conjuntos $A \times C$ y $B \times D$ tienen la misma cardinalidad. \newline

    En matemáticas, el producto cartesiano de dos conjuntos es una operación,
    que resulta en otro conjunto, cuyos elementos son todos los pares ordenados 
    que pueden formarse de forma que el primer elemento del par ordenado pertenezca
    al primer conjunto y el segundo elemento pertenezca al segundo conjunto. \newline

    Por lo que siguiendo el \textbf{Teorema 7.4.3} que dice: \newline

    Sean $A$ y $B$ dos conjuntos tales que $A \subseteq B$. Si B es contable entonces $A$
    también es contable, es decir, cualquier subconjunto de un conjunto contable
    es contable. \newline

    ¿Qué quiere decir esto? Esto indica que el producto Cruz entre $A$ y $B$ o $C$ y $D$ tienen
    la misma cardinalidad, esto porque ya sabemos que los conjuntos en los que se realiza la
    operación tienen la misma cardinalidad.

    \section*{Pregunta 3}

    Sea $A$ un conjunto y sea $R$ una relación sobre $A$. Si R es una relación de equivalencia entonces
    $R$ no es una función de $A$ en $A$. Demostrar o refutar el enunciado anterior. \newline

    siguiendo el \textbf{Teorema 8.3.1} que dice: \newline

    Sea $A$ un conjunto con una partición y sea $R$ la relación inducida por la
    partición. La relación $R$ es reflexiva, simétrica y transitiva. \newline

    Por lo que nos damos cuenta que $R$ sí es una función de $A$ en $A$, porque cumple todos los 
    "requisitos".

    

\end{document}