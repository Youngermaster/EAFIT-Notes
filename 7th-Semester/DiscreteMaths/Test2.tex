\documentclass[12pt, letterpaper, twoside]{article}
\usepackage[utf8]{inputenc}
\usepackage{geometry}
 \geometry{
 a4paper,
 total={170mm,257mm},
 left=20mm,
 top=20mm,
 }
\providecommand{\dotminus}{\mathbin{\mathpalette\xdotminus\relax}}
\newcommand{\xdotminus}[2]{%
  \ooalign{\hidewidth$\vcenter{\hbox{$#1\dot{}$}}$\hidewidth\cr$#1-$\cr}%
}
\title{Parcial 2, Punto 5}
\author{Juan Manuel Young Hoyos}
\date{March 16, 2021}

\begin{document}

\maketitle

Sea A un subconjunto de un conjunto universal U. Construir una demostración algebraica para la siguiente identidad:
 $A \dotminus  U= A^c$
Justificar cada paso de la demostración indicando una propiedad del Teorema 6.2.2 del texto guía

\[ (A - U) \cup (U - A) = A^c \]

\[ (A \cap U^c) \cup (U \cap A^c) = A^c \]

\[ (A \cap \emptyset) \cup (U \cap A^c) = A^c \]

\[ \emptyset \cup (U \cap A^c) = A^c \]

\[ (\emptyset \cup U) \cap (\emptyset \cup A^c) = A^c \]

\[ U \cap  A^c = A^c \]

\[ A^c = A^c\]

\end{document}