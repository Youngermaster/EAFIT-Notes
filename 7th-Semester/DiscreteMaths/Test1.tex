\documentclass[12pt, letterpaper, twoside]{article}
\usepackage[utf8]{inputenc}
\title{Parcial 1}
\date{Febrero 16 de 2021}
\author{Juan Manuel Young Hoyos \thanks{201810117010}}

\begin{document}
    \maketitle

    \section{}

    ¿Es posible definir recursivamente el conjunto de los números reales no negativos? En
    caso afirmativo, presentar la definición. En caso negativo, justificar su respuesta.

    \section{}

    Un estudiante A dice que puede definir una función f de los enteros positivos a los 
    enteros positivos por las siguientes ecuaciones:

    \[ f(1)=2 \]
    \[ f(1) = 2 \]
    \[ f(n) = 1 + f(f(n-1),\:si \; n>=2 \]

    Un estudiante B dice que la función f no está bien definida. ¿Cuál estudiante tiene la razón?

    \begin{itemize}
        \item El estudiante B tiene la razón.
    \end{itemize}

    \section{}

    Supongamos que le solicitan demostrar por inducción matemática que $n = n + 1$, 
    para todo número natural $n >= 1$. Su demostración falla porque no es posible demostrar:

    \begin{itemize}
        \item Ni el paso base, ni el paso inductivo.
    \end{itemize}
 
    \section{}

    El conjunto potencia (o conjunto de partes) del conjunto vacío, denotado P($\emptyset$), es:

     \begin{itemize}
         \item  $\emptyset \cap \{\emptyset\} $
     \end{itemize}

    \section{}

    El argumento del elemento es un método de demostración empleado para demostrar que un conjunto
    es un subconjunto de otro conjunto. Este método se puede emplear para demostrar que dos conjuntos son iguales porque:

    \begin{itemize}
        \item si $ A = B $ implica que $ A \subseteq B $ y $ B \subseteq A $
    \end{itemize}

    \section{}

    Se define un conjunto S de forma recursiva como sigue:
    
    I. Base: $\in$ $\in$ S
    
    II. Recursión: Si s $\in$ S, entonces,
    
    a. 0s1 $\in$ S
    
    b. 1s0 $\in$ S
    
    III. Restricción: No hay nada en S que no sean objetos definidos en I y II.

    Demostrar por inducción estructural que cada cadena en S tiene el mismo número de ceros que de unos:

    \[ S = \{0, 1\}\]

    \[s \in S^*,\; entonces\; s0,\; s1 \in S^*\]

    \[n \in Nat\]

    \[PR1.\; s \in S^* \rightarrow 0s1 \in S^*\]

    \[P(s) \rightarrow P(0s1)\]

    \[PR2.\; s \in S^* \rightarrow 1s0 \in S^*\]

    \[P(s) \rightarrow P(1s0)\]

    \section{}

    La relación lógica entre los principios de inducción matemática e inducción matemática fuerte, 
    para los números naturales, implica que:

    \begin{itemize}
        \item Ninguna de las anteriores, un enunciado se puede demostrar con inducción matemática, si y sólo si, 
        el enunciado se puede demostrar con inducción matemática fuerte.
    \end{itemize}


\end{document}