\section{Montaje 1}

\renewcommand{\labelenumii}{\arabic{enumi}.\arabic{enumii}}
\renewcommand{\labelenumiii}{\arabic{enumi}.\arabic{enumii}.\arabic{enumiii}}
\renewcommand{\labelenumiv}
{\arabic{enumi}.\arabic{enumii}.\arabic{enumiii}.\arabic{enumiv}}

\begin{enumerate}
    \item En estos ejercicios es posible que debido a condiciones externas u
    otros factores, los materiales dados en el laboratorio para realizar los
    ejercicios de frotación con otros, no manifiesten el comportamiento
    esperado. En ese caso se apela a fomentar su instinto investigativo, su buen
    juicio y su capacidad para solución de problemas, agotando todos los
    recursos que encuentre a la mano para hacer que el ejercicio funcione.
    \begin{enumerate}
        \item Recorte pequeños trocitos de papel y reúnalos; luego tome la barra
        de vidrio y frótela con seda; acérquela a los trocitos de papel. Repita
        este proceso con una barra de plástico frotándola con piel. Describa lo
        que ocurre.\\
        Explique apoyado en la teoría sobre el comportamiento de LAS CARGAS EN
        EL INTERIOR DE materiales dieléctricos por qué los trocitos de papel son
        atraídos por la barra de plástico y también por la barra de vidrio.\\

        \textit{
            Los trocitos de papel son atraídos por ambas barras dado que al 
            frotar la barra de plástico con seda esta queda cargada
            negativamente y los trozos de papel se verán atraídos por ellas,
            gracias a que la misma barra induce cargas en el papel atrayendo sus
            protones a los electrones de esta misma, lo anterior también se ve
            reflejado con la barra de vidrio solo que en sentido contrario.
        }

        \item Frote una barra de vidrio con un pedazo de seda y colóquela en una
        base giratoria; trate que la barra se quede quieta. Luego frote otra
        barra de plástico con piel y acérquela a la barra de vidrio anterior, en
        el extremo que hizo el frotamiento. Describa lo que ocurre.\\

        \textit{
            Lo que ocurre entre la barra de vidrio y la de plástico, que se
            encuentran en la base giratoria es un proceso de atracción, a partir
            de esto inferimos que las cargas de estas barras son opuestas.\\
            
            En este experimento la barra de vidrio adquirió carga positiva y la
            barra de plástico adquirió carga negativa, es por esto por lo que se
            atrajeron entre sí. 
        }

        \item Repita el literal (1.2) con dos barras de plástico frotadas con
        piel. Describa lo que ocurre.

        \textit{
            Al frotar las dos barras de plástico con piel y al momento de
            colocarlas en la base giratoria las barras se alejan entre ellas, es
            decir, existe una fuerza de repulsión, ya que ambas barras poseen el
            mismo signo.
        }
        
        \item Explique o realice un diseño experimental sencillo que permita:
        \begin{enumerate}
            \item Identificar la existencia de carga eléctrica en un material.\\
            
            \textit{Experimento entre globos,} materiales a utilizar:
            \begin{itemize}
                \item Dos globos.
                \item Cabello Humano.
            \end{itemize}
            \textit{
                \\Nuestros globos se llamarán X y Y. En primera instancia, si
                acercamos el globo X al globo Y no ocurre absolutamente nada
                (ya sea atracción o repulsión) y es lo que se espera pese a que ambos
                están en estado neutro. En otra situación frotaremos el globo X
                en nuestro cabello por un lapsus de tiempo y seguidamente lo
                acercamos al globo Y, notaremos que ambos se atraen entre sí y
                esto se lleva a cabo porqué ambos elementos quedaron con cargas 
                opuestas. Y en un último caso frotaremos ambos globos tanto el
                globo X como el Globo Y en nuestro cabello y seguidamente
                acercamos ambos globos y notaremos que ambos globos se repelen
                entre sí, esto ocurre ya que ambos globos quedaron con carga de
                igual signo.
            }
            \item Verificar la existencia de dos tipos de carga eléctrica.\\
            \textit{
                En el experimento anterior (El de los globos) se logró demostrar
                la existencia de carga eléctrica en un material, más
                específicamente en los globos, gracias a esas cargas adquiridas
                en cada situación experimental logramos observar tanto el
                proceso de atracción ó repulsión. 
            }
        \end{enumerate}

        \item Indique y describa con detalle apoyándose con esquemas, cómo se
        pueden cargar dos esferas metálicas aisladas del mismo material y tamaño:

        \begin{enumerate}
            \item Con cargas de igual magnitud, pero con signos diferentes.
            \item Con cargas de igual magnitud e igual signo. Haga una secuencia
            de gráficos que relacione la situación correspondiente explicando
            cada uno de los pasos.
        \end{enumerate}

        \item Considere la siguiente situación:\\
        Se aproxima una carga negativa a un conductor aislado sin carga, el
        conductor se aterriza (se conecta a tierra) mientras la carga está cerca.

        \begin{enumerate}
            \item ¿El conductor se carga? (Sí o no) Explique.
            \item Ahora si se retira la carga y luego la conexión a tierra,
            ¿se carga el conductor? (Si o No) Explique.
            \item Si se suprime la conexión a tierra y luego se retira la carga
            externa, ¿se carga el conductor? (Sí o No) Explique.
            \item ¿Cómo es posible verificar el comportamiento de los materiales
            conductores y los no conductores?
        \end{enumerate}

        \item Dé una explicación basada en el comportamiento triboeléctrico
        (consultar sobre esto) de los materiales de ¿Por qué el vidrio se carga
        positivamente y el ámbar negativamente?

    \end{enumerate}
\end{enumerate}
