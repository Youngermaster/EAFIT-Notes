\section{Objetivos}

\subsection{Objetivo General}
Explicar y comprender experimentalmente cómo se puede determinar la relación carga-masa del electrón, la técnica de difracción para medir el grosor de un cabello, y el funcionamiento del interferómetro de Michelson-Morley, detallando cada proceso y su importancia en la física moderna.

\subsection{Objetivos Específicos}
\begin{itemize}
    \item Explicar la metodología para calcular la relación carga-masa del electrón usando un tubo de vacío esférico que contiene un cañón de electrones y placas deflectoras. Para ello, se detalla cómo un campo magnético uniforme creado por bobinas de Helmholtz influye en la trayectoria de los electrones y permite calcular dicha relación.
    \item Describir el uso de patrones de difracción para medir el grosor de un cabello humano, explicando cómo un haz de luz láser produce estos patrones al atravesar el cabello.
    \item Analizar el principio de interferencia y la sensibilidad del interferómetro de Michelson-Morley, explicando cómo los patrones de interferencia se afectan al cambiar la distancia en las trayectorias ópticas. Esto permite ilustrar el concepto de interferencia de ondas y la aplicación del interferómetro en experimentos de precisión.
\end{itemize}

\section{Marco Teórico}
La física moderna aborda conceptos fundamentales que explican los fenómenos a nivel microscópico y macroscópico, revelando la naturaleza ondulatoria y partícula de la materia y la energía. Los experimentos en este laboratorio son cruciales para entender teorías fundamentales como la cuántica y la relatividad especial.

\subsection{Relación Carga-Masa del Electrón}
El experimento de la relación carga-masa del electrón demuestra cómo los campos magnéticos afectan las trayectorias de partículas cargadas. Un campo magnético perpendicular a la velocidad de un electrón lo desvía en una trayectoria circular, según la fuerza de Lorentz:
\[
F = qvB = \frac{mv^2}{r}
\]
donde \(q\) es la carga del electrón, \(v\) su velocidad, \(B\) el campo magnético, y \(r\) el radio de la trayectoria circular.

\subsection{Emisión de Luz por Gases Ionizados}
La excitación de átomos de gases bajo un campo eléctrico y su posterior emisión de luz es un fenómeno que se estudia tanto en física atómica como en astrofísica, proporcionando una ventana a la composición y características de objetos astronómicos distantes.

\subsection{Difracción para Medir el Grosor de un Cabello}
La difracción ocurre cuando una onda de luz encuentra un obstáculo de dimensiones comparables a su longitud de onda. La fórmula para determinar el grosor del cabello a través de la difracción es:
\[
d = \frac{\lambda L}{\Delta y}
\]
donde \(\lambda\) es la longitud de onda del láser, \(L\) la distancia de la pantalla, y \(\Delta y\) el espaciamiento entre los máximos de interferencia.

\subsection{Interferómetro de Michelson-Morley}
Este dispositivo fue fundamental para refutar la existencia del éter y es esencial para experimentos que requieren alta precisión en la medición de distancias. El interferómetro divide un haz de luz en dos, enviando cada uno en direcciones perpendiculares y luego reuniéndolos para crear un patrón de interferencia que depende de las diferencias en el recorrido óptico:
\[
\Delta x = \frac{\lambda}{2\pi} \Delta \phi
\]
donde \(\Delta \phi\) es la diferencia de fase entre los dos haces.


\section{Experimentos}
\subsection{Experimento 1: Relación Carga-Masa del Electrón}
Descripción del montaje y resultados esperados.

\subsection{Experimento 2: Emisión de Luz por Gases Ionizados}
Descripción del montaje y resultados esperados.

\subsection{Experimento 3: Medición del Grosor de un Cabello}
En este experimento, aplicamos el principio de difracción de la luz para medir el grosor de un cabello humano. La luz láser, al pasar a través de un cabello, produce un patrón de difracción, y a partir de este patrón podemos calcular el diámetro del cabello utilizando la relación entre la longitud de onda del láser, la distancia entre las franjas claras de difracción y la distancia desde el cabello hasta la superficie donde se proyecta el patrón.

\subsubsection{Datos Experimentales}
\begin{itemize}
    \item Distancia del láser a la superficie \(d\): \(2.76 \, \text{m}\)
    \item Distancia entre las líneas claras observadas en el patrón de difracción \(d\): \(0.01 \, \text{m}\)
    \item Longitud de onda del láser \(\lambda\): \(518 \, \text{nm} = 0.000000518 \, \text{m}\)
\end{itemize}

\subsubsection{Cálculo del Grosor del Cabello}
Utilizando la fórmula para la difracción de la luz por un cabello:
\[
d = \frac{\lambda L}{\Delta y}
\]
donde:
\begin{itemize}
    \item \(L\) es la distancia desde el cabello hasta la superficie de proyección.
    \item \(\Delta y\) es la distancia entre las franjas de difracción.
    \item \(\lambda\) es la longitud de onda del láser.
\end{itemize}
Sustituyendo los valores obtenemos:
\[
W = \frac{0.000000518 \times 2.76}{0.01} \approx 0.000143 \, \text{m} = 143 \, \text{micrómetros}
\]

\subsubsection{Análisis de Error}
Considerando el rango típico para el grosor del cabello humano, calculamos los porcentajes de error:
\begin{itemize}
    \item Grosor máximo estimado: \(170 \, \text{micrómetros}\)
    \[
    \text{Error} = \left| \frac{143 - 170}{170} \right| \times 100\% \approx 15.88\%
    \]
    \item Grosor promedio del cuero cabelludo: \(110 \, \text{micrómetros}\)
    \[
    \text{Error} = \left| \frac{143 - 110}{110} \right| \times 100\% \approx 30\%
    \]
    \item Grosor típico en individuos asiáticos: \(120 \, \text{micrómetros}\)
    \[
    \text{Error} = \left| \frac{143 - 120}{120} \right| \times 100\% \approx 19.17\%
    \]
\end{itemize}

\subsubsection{Relación con la Física Moderna}
Este experimento demuestra la aplicación de la teoría de la difracción de ondas, un concepto fundamental en la física moderna que se extiende al estudio de la naturaleza ondulatoria de las partículas (dualidad onda-partícula) en la mecánica cuántica. Además, la capacidad de medir dimensiones a escalas microscópicas usando principios ópticos es esencial en tecnologías de nanofabricación y caracterización de materiales en física y ingeniería. Este análisis proporciona una base para entender cómo técnicas simples pueden aplicarse para estudiar y verificar fenómenos complejos y principios en física moderna, conectando la teoría con aplicaciones prácticas y experimentales.


\subsection{Experimento 4: Interferencia de Luz con el Láser}
Descripción del montaje y resultados esperados.

\section{Conclusiones}
Reflexiones sobre cómo cada uno de los experimentos contribuye al entendimiento de la física moderna y sus aplicaciones tecnológicas y científicas.
