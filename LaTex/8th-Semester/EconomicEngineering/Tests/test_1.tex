\documentclass[12pt, letterpaper, twoside]{article}
\usepackage[utf8]{inputenc}
\title{Parcial 1}
\date{Febrero 17 de 2021}
\author{Sebastian Villegas Valencia | Juan Manuel Young Hoyos}

\begin{document}
    \maketitle

    \section{}


    \[ (1 + i_1)^{m_1} =  (1 + i_2)^{m_2} \]

    \[ (1 + 0.18)^{1} =  (1 + i_2)^{12} \]

    \[ (1 + 0.18)^{1/12} =  1 + i_2 \]

    \[ (1 + 0.18)^{1/12} - 1 = i_2 \]

    \[ (1 + 0.18)^{1/12} - 1 = i_2 \]

    \[ (1.18)^{1/12} - 1 = i_2 \]

    \[ (1.013) - 1 = i_2 \]

    \[ 0.013 = i_2 \]

    \[ \rightarrow F_6 = 20000000 (1 + 0.013)^6 \]

    \[ F_6 = 21611587.41 \]
    
    \( \rightarrow \) a \( F_6 \) le restamos el primer pago y nos quedaría:

    \[ F_6 = 21611587.41 - 5000000 \]

    \[ F_6 = 16611587.41\]

    \[ \rightarrow F_6 = 16611587.41 (1 + 0.013)^6 \]

    \[ F_6 = 17950138.67\]

    \( \rightarrow \) a \( F_6 \) le restamos el segundo pago y nos quedaría:

    \[ F_6 = 17950138.67 - 7000000 \]

    \[ F_6 = 10950138.67 \]

    \[ 15500000 = 10950138.67 (1 + 0.013 * n) \]

    \[ n = 31.9 \; meses \]

\section{}

\[ i = \frac{J}{m} \]

\[ \rightarrow 17\% N.M. \]

\[ i = \frac{17}{12} = 1.4166666 \; E.M. \]

\[ 1.417 \; E.M. \]

\[ \rightarrow 21\% \; N.S. \]

\[ i = \frac{21}{2} = 10.5 \; E.S. \]

\[ 10.5\% \; E.S \]

\[ (1 + 0.105)^2 = (1 + i)^{12} \]

\[ (1.105)^2 = (1 + i)^{12} \]

\[ 1.221 = (1 + i)^{12} \]

\[ \sqrt[12]{1.221} = 1 + i \]

\[ \sqrt[12]{1.221} - 1 = i \]

\[ i = 0.01678 \]

\[ 1.678\% \; E.M. \]

\[ \rightarrow P_0 = 1800000 \]

\[ \rightarrow P_3 = \frac{300000}{(1 + 0.01417)^3} \]

\[ P_3 = 287600.0624 \]

\[ \rightarrow P_7 = \frac{500000}{(1 + 0.01678)^1} \]

\[ P_7 = 491748.4608 \]

\[ \rightarrow P_{12} = \frac{800000}{(1 + 0.01678)^6} \]

\[ P_{12} = 723982.4131 \]

\[ \rightarrow P_{15} = \frac{1200000}{(1 + 0.01678)^9} \]

\[ P_{15} = 1033090.315 \]

\[ \rightarrow F_6 = P_7 + P_{12} + P_{15} \]

\[ F_6 = 491748.4608 + 723982.4131 + 1033090.315 \]

\[ F_6 = 2248857.189 \]

\[ \rightarrow P_6 = \frac{2248857.189}{(1 + 0.01417)^6} \]

\[ P_6 = 2066794.605\]

\[ \rightarrow P = P_0 + P_3 + P_6 \]

\[ P = 1800000 + 287600.0624 + 2066794.605\]

\[ P = 4154394.667\]

\(\rightarrow \) Valor de contado.

\[ P_{contado} = P - 25\% \]

\[ P_{contado} = 4154394.667 - (4154394.667 \times 0.25) \]

\[ P_{contado} = 3115796 \]

El valor del artículo de tecnología es de \(\$4'154'394.667\), ahora bien, si se quiere
comprar de contado, este recibirá un descuento del 25\%, por lo que quedaría con un valor
de \(\$3'115'796\) 

\begin{itemize}
  \item Stuff
  \item More Stuff
  \item More Stuff
  \item More Stuff
\end{itemize}

\end{document}
