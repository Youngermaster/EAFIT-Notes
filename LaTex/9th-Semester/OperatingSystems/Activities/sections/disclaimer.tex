\section{Disclaimer}

A penetration test is considered a snapshot in time. The findings and recommendations reflects the
information gathered during the assessment and not any changes or modifications made outside of that
period. 

Time-limited engagements do not allow for a full evaluation of all security controls.
\textbf{\companyName} prioritized the assessment to identify the weakest security controls an
attacker would exploit. \textbf{\companyName} recommends conducting similar assessments on an annual
basis by internal or third-party assessors to ensure the continued success of the controls. 

\subsection{Confidentiality Statement}

This document is the exclusive property of \textbf{\companyName}. This
document contains proprietary and confidential information. Duplication, redistribution, or use, in
whole or in part, in any form, requires consent of \textbf{\companyName}. 

\subsection{Contact info}

\begin{tabular}{ |p{3cm}|p{3cm}|p{3cm}|p{3cm}|  }
\hline
\multicolumn{4}{|c|}{\textbf{Contact info.}} \\
\hline
	Name & Title & Email & Contact \\
\hline
    Juan & Cybersecurity Lead & juan@test & (99) 9999 9999 \\
    Manuel & Junior Pentester & manuel@test & (99) 9999 9999 \\
    Young & Junior Pentester & young@test & (99) 9999 9999 \\
\hline
\end{tabular}

\subsection{Assesment overview}
From \textit{Date 1} to \textit{Date 2}, \textbf{\companyName} engaged Penetration tests to evaluate
the security posture of its infrastructure compared to current industry best practices. All testing
performed is based on the NIST SP 800-115 Technical Guide to Information Security Testing and
Assessment, OWASP Testing Guide (v4), and customized testing frameworks.  

Phases conducted for penetration testing are the following:

\begin{itemize}
    \item Planning and preparation.
    \item Reconnaissance / Discovery.
    \item Vulnerability Enumeration / Analysis.
    \item Initial Exploitation.
    \item Expanding Foothold / Deeper Penetration.
    \item Cleanup.
    \item Report Generation.
\end{itemize}

\clearpage
