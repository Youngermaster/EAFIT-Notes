\documentclass{exam}
\usepackage[utf8]{inputenc}

\begin{document}

\begin{center}
\fbox{\fbox{\parbox{5.5in}{\centering
Escuela de Ingeniería
Departamento de Ingeniería de Sistemas
Parcial 1 - Análisis Numérico - ST0256 - G002}}}
\end{center}

\textbf{Nombre:} \underline{Juan Manuel Young Hoyos} \enspace\hrulefill

\textbf{Código:} \underline{201810117010} \enspace\hrulefill

\textbf{Nota:\enspace\hrulefill}

\begin{enumerate}

    \section*{Falso o Verdadero}

    \item (15 pts) Para los siguientes enunciados diga si este es Falso (F) o Verdadero (V). Escriba su respuesta sobre la línea que precede al enunciado. No se requiere justificación.
    
    \begin{enumerate}
        \item (2 pts) \textbf{\underline{V}} El método de Bisección es un método de búsqueda incremental que permite encontrar un valor faltante en una serie de datos, al que llamamos punto fijo.
        \item (2 pts) \textbf{\underline{V}} El método de Newton se basa en obtener la recta tangente que pasa exactamente por los puntos $ (x_{n-1}, f(x_{n-1})) $ y $ (x_n, f(x_n)) $.
        \item (2 pts) \textbf{\underline{V}} Una similitud entre los métodos de Newton y Bisección es que ambos encuentran la raíz de una función en una determinada cantidad de iteraciones que se pueden determinar a priori.
        \item (2 pts) \textbf{\underline{V}} El método de Bisección es un método de búsqueda incremental donde el intervalo se divide siempre en una cantidad impar de partes, siendo siempre una más grande que las otras, lo que permite aproximar más eficientemente a la raíz.
        \item (2 pts) \textbf{\underline{F}} Si un método numérico no converge, el resultado obtenido será la solución aproximada del problema.
        \item (2 pts) \textbf{\underline{F}} Sea $M$ el conjunto de todos los números que se pueden almacenar en cierta máquina. Dados $m_1, m_2 \in M$, siempre ocurre que $m_1 \times m_2 \in M$.
        \item (3 pts) \textbf{\underline{F}} Dado $x$, en la aproximación de $f(x)$ usando series de Taylor está involucrado el error de propagación, siempre y cuando se inicie en cero.
    \end{enumerate}

    \section*{Selección Múltiple}

    Para la siguiente pregunta encierre en un óvalo la respuesta correcta y no requiere justificación.

    \item (15 pts) La velocidad ascendente, $v$, de un cohete puede ser calculada por la siguiente expresión:
    
    \[ v = u \times Ln(\frac{m_o}{m_0-qt}) - gt \]

    En dónde u es la velocidad relativa a la que el combustible es expedido, $m_0$ es la masa inicial del cohete en el instante $t = 0$, $q$ 
    es la tasa de consumo de combustible y $g$ es la aceleración de la gravedad. Considerando $u = 2200m/s, g = 9.81m/s^2, m0 = 1.6 × 105kg \;y q = 2680kg/s $
    y empleando el método de Newton - Raphson, el tiempo para el cual el cohete alcanza la velocidad
    de $v = 1000m/s$, y con un valor inicial de $t_0 = 22s$, en la tercera iteración es:

    \( \textbf{D.}\;25.169875 \)

    \item (10 pts) Una mejora al método de Newton-Raphson está dada por:
    
    \[ x_1 + 1 = x_i - \frac{f(x_i) f'(x_i)}{(f'(x_i))^2 - f(x_i)f''(x_i) } \]
    
    Usando esta fórmula, una de las raíces de la ecuación $ f(x)=x^3-5x^2+7x-3 $ para $ x_0 = 0 $ en la segunda iteración $(x_2)$ es:

    \( \textbf{B.}\;1.003082 \)

    \section*{Respuesta Corta}

    Para la siguiente pregunta diligencie la respuesta correcta y no requiere justificación.

    \item (10 pts) El método de Newton-Raphson aplicado a la ecuación $x^k + A = 0$ produce el siguiente esquema de iteración:

    \begin{center}
        \begin{tabular}{ |c|c|c| } 
        \hline
        i & $x_i$ & $f(x_i)$ \\ 
        0 & 0 & 0 \\ 
        \hline
        \end{tabular}
    \end{center}

\end{enumerate}



\end{document}