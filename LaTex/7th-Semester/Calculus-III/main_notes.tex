\documentclass{article}
\usepackage[margin=0.25in]{geometry}
\usepackage{pgfplots}
\usepackage{pdftexcmds}
\pgfplotsset{width=10cm,compat=1.9}
\title{Calculus module III}
\author{Juan Manuel Young Hoyos}
\date{July 2021}
\begin{document}
\maketitle

\section*{Ecuaciones Paramétricas}
En matemáticas, un sistema de ecuaciones paramétricas permite representar una curva o superficie
en el plano o en el espacio, mediante valores que recorren un intervalo de números reales, mediante
una variable, llamada parámetro, considerando cada coordenada de un punto como una función dependiente
del parámetro.
Por ejemplo:

\[ x = f(t)\]
\[ y = g(t)\]

%Here begins the 2D plot
\begin{tikzpicture}
\begin{axis}
\addplot[color=red]{exp(x^2)};
\end{axis}
\end{tikzpicture}

\begin{tikzpicture}
\begin{axis}[
    title=Example using the mesh parameter,
    hide axis,
    colormap/cool,
]
\addplot3[
    mesh,
    samples=50,
    domain=-8:8,
]
{sin(deg(sqrt(x^2+y^2)))/sqrt(x^2+y^2)};
\addlegendentry{\[ sin(deg(sqrt(x^2+y^2)))/sqrt(x^2+y^2) \]}
\end{axis}
\end{tikzpicture}

\end{document}
