\section{Implementación}

\subsection{OSPF}

En este proyecto, uno de los aspectos fundamentales para garantizar la robustez
y la escalabilidad de nuestra infraestructura de red es la implementación de
una estrategia de enrutamiento eficiente. A lo largo de este trabajo, hemos
enfatizado la importancia del VLSM y el subnetting en el diseño de nuestra red.
Sin embargo, para añadir una capa adicional de eficiencia y flexibilidad, hemos
optado por implementar el Protocolo de Estado de Enlace Abierto
(Open Shortest Path First, OSPF).
\\

OSPF es un protocolo de enrutamiento de estado de enlace, que es especialmente
adecuado para redes de gran tamaño, como la que estamos diseñando para esta
empresa multinacional. Al combinarlo con nuestra estrategia de VLSM y
subnetting, y dada nuestra topología de red en estrella y el esperado alto
volumen de tráfico de la página web de la empresa, OSPF se presenta como una
solución ideal para gestionar eficazmente el tráfico de red y asegurar una
conectividad ininterrumpida.
\\

La importancia de OSPF en este proyecto radica en su capacidad para determinar
la ruta más corta y menos congestionada para el envío de paquetes de datos en
la red. Esto se logra mediante el algoritmo de Dijkstra, que OSPF utiliza para
calcular las rutas más eficientes. Esta capacidad es particularmente relevante
dada la estructura de nuestra red, que conecta todas las sedes de la empresa a
un nodo central, en este caso, la sede principal en Bogotá.
\\

Además, OSPF es un protocolo de enrutamiento dinámico, lo que significa que es
capaz de adaptarse rápidamente a los cambios en la red. En caso de un fallo de
red o un cambio en la topología de la red, OSPF puede ajustar rápidamente las
rutas de los paquetes de datos para evitar la interrupción de la conectividad.
Esta capacidad de recuperación y adaptabilidad, combinada con la flexibilidad
proporcionada por VLSM y subnetting, es esencial para mantener la disponibilidad
del sitio web de la empresa y garantizar un rendimiento óptimo,
independientemente del volumen de tráfico.
\\

En las siguientes secciones, detallaré cómo hemos implementado OSPF en nuestra
infraestructura de red, y cómo esta decisión, en combinación con el uso de VLSM
y subnetting, contribuye a los objetivos de robustez, escalabilidad y
eficiencia de nuestro diseño de red. Estamos convencidos de que, con la
implementación de OSPF, estaremos mejor equipados para enfrentar los desafíos
que presenta la gestión de una red de gran tamaño y alto tráfico.

\subsection{VLSM}

Ahora bien, antes de comenzar a desarrollar e implementar nuestra solución, es
esencial definir completamente la infraestructura. Para esto, utilizaremos
VLSM (Variable Length Subnet Mask) y técnicas de subnetting. En la siguiente
sección, nos centraremos en dos componentes críticos de la gestión de redes:
el Subneteo de Longitud de Máscara Variable
(Variable Length Subnet Masking, VLSM) y el Subnetting.
\\


En el contexto de nuestro proyecto, tanto VLSM como el Subnetting son esenciales
por diversas razones. Utilizamos VLSM principalmente para los routers. Esta
técnica nos permite maximizar el uso del espacio de direcciones IP asignado a
la empresa.
\\


Por otro lado, para el resto de la red, recurrimos al subnetting para crear
subredes que se ajusten a las necesidades específicas de cada sede o
departamento. Esta estrategia puede mejorar el rendimiento de la red al reducir
la cantidad de tráfico de enrutamiento innecesario, y al mismo tiempo, nos
permite aplicar políticas de seguridad más granulares, mejorando así la
seguridad de la red.
\\

Finalmente, la combinación de VLSM y subnetting es una herramienta poderosa que
nos permite diseñar e implementar una red que es eficiente, segura y capaz de
satisfacer las necesidades específicas de la empresa.
\\

Ahora bien, tenemos 7 redes de oficinas y 6 redes \textit{WAN}. En teoría se
podrían hacer con 8 subredes dejando la última subred con VLSM. Ahora bien, es
bueno tener en cuenta que:

\begin{itemize}
    \item \(2^3 = 8\), Red \(\rightarrow\) IP privada \textbf{clase B}.
    \item \(2^4 = 16\), máscara de 16.
    \item Haremos 8 subredes y la última le haremos VLSM.
\end{itemize}

En la siguiente tabla, \cref{table: subnetting 1}, \textbf{Usaremos la IP \(132.18.0.0/16\)}
para realizar el subnetting.

\begin{table}[ht]
    \rowcolors{2}{EAFIT-blue!10}{white}
    \centering
    \caption{Subnetting usando la IP \(132.18.0.0/16\).}
    \begin{tabular}[t]{ccccc}
        \toprule
        \color{EAFIT-blue}\textbf{Región} & \color{EAFIT-blue}\textbf{Subred} & \color{EAFIT-blue}\textbf{Dirección de Broadcast} & \color{EAFIT-blue}\textbf{Interavalo de direcciones} \\
        \midrule
        Bogotá                            & \(132.18.0.0\)                    & \(132.18.31.255\)                                 & \(132.18.0.1 - 132.18.31.254\)                       \\
        Medellín                          & \(132.18.32.0\)                   & \(132.18.63.255\)                                 & \(132.18.32.1 - 132.18.63.254\)                      \\
        Río Negro                         & \(132.18.64.0\)                   & \(132.18.95.255\)                                 & \(132.18.64.1 - 132.18.95.254\)                      \\
        Cali                              & \(132.18.96.0\)                   & \(132.18.127.255\)                                & \(132.18.86.1 - 132.18.127.254\)                     \\
        Popayán                           & \(132.18.128.0\)                  & \(132.18.159.255\)                                & \(132.18.128.1 - 132.18.159.254\)                    \\
        Cartagena                         & \(132.18.160.0\)                  & \(132.18.191.255\)                                & \(132.18.160.1 - 132.18.191.254\)                    \\
        Barranquilla                      & \(132.18.192.0\)                  & \(132.18.223.255\)                                & \(132.18.192.1 - 132.18.223.254\)                    \\
        Redes WAN                         & \(132.18.224.0\)                  & \(132.18.255.255\)                                & \(132.18.224.1 - 132.18.255.254\)                    \\
        \bottomrule
    \end{tabular}
    \label{table: subnetting 1}
\end{table}

En la siguiente tabla, \cref{table: VLSM WAN}, realizaremos el VLSM para las
redes WAN. Es bueno tener en cuenta que:

\begin{itemize}
    \item Máscara \(30 \rightarrow 255.255.255.252 \).
    \item IP \(\rightarrow 132.18.224.0 \).
\end{itemize}

\begin{table}[ht]
    \rowcolors{2}{EAFIT-blue!10}{white}
    \centering
    \caption{VLSM para las redes WAN con la IP \(132.18.224.0\).}
    \begin{tabular}[t]{ccccc}
        \toprule
        \color{EAFIT-blue}\textbf{Subred} & \color{EAFIT-blue}\textbf{Dirección de Broadcast} & \color{EAFIT-blue}\textbf{Interavalo de direcciones} \\
        \midrule
        \(132.18.224.0\)                  & \(132.18.224.3\)                                  & \(132.18.224.1 - 132.18.224.2\)                      \\
        \(132.18.224.4\)                  & \(132.18.224.7\)                                  & \(132.18.224.5 - 132.18.224.6\)                      \\
        \(132.18.224.8\)                  & \(132.18.224.11\)                                 & \(132.18.224.9 - 132.18.224.10\)                     \\
        \(132.18.224.12\)                 & \(132.18.224.15\)                                 & \(132.18.224.13 - 132.18.224.14\)                    \\
        \(132.18.224.16\)                 & \(132.18.224.19\)                                 & \(132.18.224.17 - 132.18.224.18\)                    \\
        \(132.18.224.20\)                 & \(132.18.224.23\)                                 & \(132.18.224.21 - 132.18.224.22\)                    \\
        \(132.18.224.24\)                 & \(132.18.224.27\)                                 & \(132.18.224.25 - 132.18.224.26\)                    \\
        \(132.18.224.28\)                 & \(132.18.224.31\)                                 & \(132.18.224.29 - 132.18.224.30\)                    \\
        \(132.18.224.32\)                 & \(132.18.224.35\)                                 & \(132.18.224.33 - 132.18.224.34\)                    \\
        \(132.18.224.36\)                 & \(132.18.224.39\)                                 & \(132.18.224.37 - 132.18.224.38\)                    \\
        \bottomrule
    \end{tabular}
    \label{table: VLSM WAN}
\end{table}

\newpage
\subsection{Desarrollo}

Como ya se sabe, nuestra infraestructura de red se basa en una topología de
estrella (más las dos arquitecturas adicionales añadidas se podría considerar
una híbrida, pero la llamaremos en estrella de este modo por ahora), donde la
ciudad de Bogotá actúa como el nodo central o router
principal. Desde Bogotá, las conexiones se extienden a través de cables seriales
hacia las demás ciudades, que incluyen Medellín, Barranquilla, Rionegro, Cali,
Popayán, entre otras, mientras que las conexiones internas se gestionan mediante
cables ethernet.
\\

Es importante destacar que, a pesar de estar interconectados, cada uno de los
routers tiene un direccionamiento de red único. Este diseño incorpora un total
de 7 switches, cada uno de los cuales está conectado a 3 clientes, lo que suma
un total de 21 clientes en toda la red.
\\

Para cada router, asignamos una dirección IP y su correspondiente máscara de
subred. Con estas configuraciones, la puerta de enlace predeterminada
corresponde a la dirección IP configurada en los routers. Los switches, por otro
lado, actúan como puntos de conexión y, por lo tanto, no requieren configuraciones adicionales.
\\

Realizamos la configuración de los routers, específicamente el modelo 1841, a
través de la terminal. Asimismo, fue necesario añadir dos interfaces seriales a
la configuración de la interfaz web para garantizar una comunicación efectiva en
toda la red.
\\

\subsection{Core, Distribución y Acceso}

Ahora bien, en comparación al trabajo pasado, este proyecto presenta una
evolución significativa, ya que se añade a la sede de Bogotá una arquitectura de
red de tres capas: \textit{Core, Distribución y Acceso}. Esta arquitectura es
una mejora fundamental para la infraestructura de red de la empresa, permitiendo
una mejor gestión del tráfico y mejorando la escalabilidad y la eficiencia de la
red.
\\

La capa Core, o núcleo, es la columna vertebral de esta arquitectura, encargada
de transportar grandes volúmenes de tráfico de forma rápida y sin
interrupciones. Su diseño está orientado a minimizar los retrasos y maximizar
la disponibilidad y la confiabilidad.
\\

La capa de Distribución actúa como intermediaria entre las capas de Core y
Acceso, administrando las políticas de red, el direccionamiento, la filtración
y el enrutamiento entre subredes. Esta capa tiene un papel crucial en el
control y segregación del tráfico de la red.
\\

Finalmente, la capa de Acceso es la que conecta los dispositivos de los
usuarios, como ordenadores y teléfonos, etc. a la red (en este caso a nuestros
servidores principales y de backup). Esta capa gestiona el control
de acceso y las políticas de seguridad, proporcionando conectividad y servicios
de red a los usuarios finales.
\\

La implementación de esta arquitectura de red de tres capas en la sede de
Bogotá permitirá a la empresa manejar de manera más eficiente el creciente
tráfico de la red, proporcionando un rendimiento de red más robusto y confiable.
\\

Entonces primero se añadió el switch 3560-24PS para empezar el core con la
dirección \textit{132.18.0.0} con máscara \textit{21}, después se añadió otro de estos switches para la capa de
distribución, y en la capa de acceso se añadieron 2 switches \textit{2950-24}, los
cuales tienen acceso a la \textbf{VLAN 10} y la \textbf{VLAN 20}, después se creó la \textbf{VLAN 10} que
está destinada para el servidor principal, y la \textbf{VLAN 20} está destinada para los
servidores de backup.
\\

\subsection{Core/Collapsed}

Al igual que en el caso de Bogotá, en Medellín se han añadido nuevos nodos a la
infraestructura de red. Sin embargo, la principal diferencia radica en el tipo de arquitectura que se implementará: en lugar de la arquitectura de tres capas que se utilizó en Bogotá, en Medellín se ha optado por la arquitectura de dos capas, específicamente, la arquitectura Core/collapsed.
\\

Esta arquitectura de red simplificada se caracteriza por su estructura en dos
niveles. Primero, se ha añadido un switch 3560-24PS para el Core/collapsed, que actuará como núcleo central o Core de la red. Este elemento se encargará de manejar el tráfico de alto volumen, garantizando una transmisión rápida y fiable de los datos a través de la red. Su objetivo es proporcionar un alto rendimiento y minimizar las latencias, esencial para la transmisión eficiente de datos.
\\

La segunda capa, denominada 'collapsed' en esta arquitectura, combina las
funciones de las capas de distribución y acceso de la arquitectura de tres capas. Esta capa maneja el control de acceso, la implementación de políticas de red y la conexión con los dispositivos finales, entre otras responsabilidades.
\\

El uso de la arquitectura de dos capas en la sede de Medellín proporciona una
solución más compacta y simplificada que la arquitectura de tres capas. Sin embargo, mantiene la eficiencia y la fiabilidad necesarias para manejar el tráfico de la red de la empresa, permitiendo así un rendimiento de red optimizado y confiable.


\subsection{NAT}

Para desarrollar esta sección, nos basamos un poco en la implementación de
\textit{ccnadesdecero} \cite{ccnaNAT}.

Para la implementación de la traducción de direcciones de red (NAT) de manera dinámica en Bogotá, se siguieron varios pasos:

\begin{enumerate}
    \item \textbf{Definición del conjunto de direcciones para la traducción}: Se determinó un conjunto de direcciones globales (direcciones IP públicas) para utilizar en la traducción.
          \begin{verbatim}
ip nat pool NAT-POOL1 172.16.0.0 172.16.255.254 netmask 255.255.0.0
\end{verbatim}

    \item \textbf{Configuración de la lista de acceso}: Se configuró una lista de acceso que permitiera las direcciones IP privadas de la red interna de Bogotá que requerían traducción.
          \begin{verbatim}
access-list 1 permit 132.18.224.8 0.0.0.3
access-list 1 permit 132.18.16.0 0.0.7.255
access-list 1 permit 132.18.8.0 0.0.7.255
access-list 1 permit 132.18.192.0 0.0.31.255
access-list 1 permit 132.18.160.0 0.0.31.255
access-list 1 permit 132.18.96.0 0.0.31.255
access-list 1 permit 132.18.128.0 0.0.31.255
access-list 1 permit 132.18.64.0 0.0.31.255
access-list 1 permit 132.18.40.0 0.0.7.255
access-list 1 permit 132.18.48.0 0.0.7.255
\end{verbatim}

    \item \textbf{Establecimiento de la traducción dinámica de origen}: Se especificó la lista de acceso y el conjunto de direcciones determinado en los pasos anteriores para establecer la traducción dinámica.
          \begin{verbatim}
ip nat inside source list 1 pool NAT-POOL1
\end{verbatim}

    \item \textbf{Identificación de la interfaz interna y externa}: Se definieron las interfaces interna y externa.
          \begin{verbatim}
interface Serial0/0/0
 ip address 128.96.224.2 255.255.255.252
 ip nat outside

interface Serial0/1/0
 ip address 132.18.224.5 255.255.255.252
 ip nat inside

interface Serial0/2/0
 ip address 132.18.224.9 255.255.255.252
 ip nat inside

interface Serial0/3/0
 ip address 132.18.224.2 255.255.255.252
 ip nat inside

\end{verbatim}

    \item \textbf{Verificación de la configuración de NAT dinámica}: Finalmente, se verificó la correcta configuración de la NAT dinámica utilizando comandos específicos.
          \begin{verbatim}
show ip nat translations
show ip nat statistics
\end{verbatim}
\end{enumerate}

Estos pasos simplifican el proceso técnico detallado pero brindan una visión clara de la implementación de la NAT dinámica en Bogotá.

