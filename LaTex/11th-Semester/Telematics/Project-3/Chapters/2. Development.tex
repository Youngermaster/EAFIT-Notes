\section{VLSM}


Ahora bien, antes de comenzar a desarrollar e implementar nuestra solución, es
esencial definir completamente la infraestructura. Para esto, utilizaremos
VLSM (Variable Length Subnet Mask) y técnicas de subnetting. En la siguiente
sección, nos centraremos en dos componentes críticos de la gestión de redes:
el Subneteo de Longitud de Máscara Variable
(Variable Length Subnet Masking, VLSM) y el Subnetting.
\\


En el contexto de nuestro proyecto, tanto VLSM como el Subnetting son esenciales
por diversas razones. Utilizamos VLSM principalmente para los routers. Esta
técnica nos permite maximizar el uso del espacio de direcciones IP asignado a
la empresa.
\\


Por otro lado, para el resto de la red, recurrimos al subnetting para crear
subredes que se ajusten a las necesidades específicas de cada sede o
departamento. Esta estrategia puede mejorar el rendimiento de la red al reducir
la cantidad de tráfico de enrutamiento innecesario, y al mismo tiempo, nos
permite aplicar políticas de seguridad más granulares, mejorando así la
seguridad de la red.
\\

Finalmente, la combinación de VLSM y subnetting es una herramienta poderosa que
nos permite diseñar e implementar una red que es eficiente, segura y capaz de
satisfacer las necesidades específicas de la empresa.
\\

Ahora bien, tenemos 7 redes de oficinas y 6 redes \textit{WAN}. En teoría se
podrían hacer con 8 subredes dejando la última subred con VLSM. Ahora bien, es
bueno tener en cuenta que:

\begin{itemize}
    \item \(2^3 = 8\), Red \(\rightarrow\) IP privada \textbf{clase B}.
    \item \(2^4 = 16\), máscara de 16.
    \item Haremos 8 subredes y la última le haremos VLSM.
\end{itemize}

En la siguiente tabla, \cref{table: subnetting 1}, \textbf{Usaremos la IP \(132.18.0.0/16\)}
para realizar el subnetting.

\begin{table}[ht]
    \rowcolors{2}{EAFIT-blue!10}{white}
    \centering
    \caption{Subnetting usando la IP \(132.18.0.0/16\).}
    \begin{tabular}[t]{ccccc}
        \toprule
        \color{EAFIT-blue}\textbf{Región} & \color{EAFIT-blue}\textbf{Subred} & \color{EAFIT-blue}\textbf{Dirección de Broadcast} & \color{EAFIT-blue}\textbf{Interavalo de direcciones} \\
        \midrule
        Bogotá                            & \(132.18.0.0\)                    & \(132.18.31.255\)                                 & \(132.18.0.1 - 132.18.31.254\)                       \\
        Medellín                          & \(132.18.32.0\)                   & \(132.18.63.255\)                                 & \(132.18.32.1 - 132.18.63.254\)                      \\
        Río Negro                         & \(132.18.64.0\)                   & \(132.18.95.255\)                                 & \(132.18.64.1 - 132.18.95.254\)                      \\
        Cali                              & \(132.18.96.0\)                   & \(132.18.127.255\)                                & \(132.18.86.1 - 132.18.127.254\)                     \\
        Popayán                           & \(132.18.128.0\)                  & \(132.18.159.255\)                                & \(132.18.128.1 - 132.18.159.254\)                    \\
        Cartagena                         & \(132.18.160.0\)                  & \(132.18.191.255\)                                & \(132.18.160.1 - 132.18.191.254\)                    \\
        Barranquilla                      & \(132.18.192.0\)                  & \(132.18.223.255\)                                & \(132.18.192.1 - 132.18.223.254\)                    \\
        Redes WAN                         & \(132.18.224.0\)                  & \(132.18.255.255\)                                & \(132.18.224.1 - 132.18.255.254\)                    \\
        \bottomrule
    \end{tabular}
    \label{table: subnetting 1}
\end{table}

En la siguiente tabla, \cref{table: VLSM WAN}, realizaremos el VLSM para las
redes WAN. Es bueno tener en cuenta que:

\begin{itemize}
    \item Máscara \(30 \rightarrow 255.255.255.252 \).
    \item IP \(\rightarrow 132.18.224.0 \).
\end{itemize}

\begin{table}[ht]
    \rowcolors{2}{EAFIT-blue!10}{white}
    \centering
    \caption{VLSM para las redes WAN con la IP \(132.18.224.0\).}
    \begin{tabular}[t]{ccccc}
        \toprule
        \color{EAFIT-blue}\textbf{Subred} & \color{EAFIT-blue}\textbf{Dirección de Broadcast} & \color{EAFIT-blue}\textbf{Interavalo de direcciones} \\
        \midrule
        \(132.18.224.0\)                  & \(132.18.224.3\)                                  & \(132.18.224.1 - 132.18.224.2\)                      \\
        \(132.18.224.4\)                  & \(132.18.224.7\)                                  & \(132.18.224.5 - 132.18.224.6\)                      \\
        \(132.18.224.8\)                  & \(132.18.224.11\)                                 & \(132.18.224.9 - 132.18.224.10\)                     \\
        \(132.18.224.12\)                 & \(132.18.224.15\)                                 & \(132.18.224.13 - 132.18.224.14\)                    \\
        \(132.18.224.16\)                 & \(132.18.224.19\)                                 & \(132.18.224.17 - 132.18.224.18\)                    \\
        \(132.18.224.20\)                 & \(132.18.224.23\)                                 & \(132.18.224.21 - 132.18.224.22\)                    \\
        \(132.18.224.24\)                 & \(132.18.224.27\)                                 & \(132.18.224.25 - 132.18.224.26\)                    \\
        \(132.18.224.28\)                 & \(132.18.224.31\)                                 & \(132.18.224.29 - 132.18.224.30\)                    \\
        \(132.18.224.32\)                 & \(132.18.224.35\)                                 & \(132.18.224.33 - 132.18.224.34\)                    \\
        \(132.18.224.36\)                 & \(132.18.224.39\)                                 & \(132.18.224.37 - 132.18.224.38\)                    \\
        \bottomrule
    \end{tabular}
    \label{table: VLSM WAN}
\end{table}

\newpage
\subsection{Implementación}

Ahora bien, para implementarlo como se mencionó anteriormente tenemos una
topología de estrella en donde nos encontramos con una ciudad o router
principal el cual es Bogotá, que de este se deriva por medio de un
\textbf{cable serial} las demás ciudades las cuales son Medellín, Barranquilla,
Río negro, Cali, Popayán y las demás están conectados por un
\textbf{cable ethernet}. Con esto también debemos aclarar que cada una de las
redes tiene un direccionamiento diferente a pesar de que los \textbf{routers}
estén unidos. Contamos con \textbf{7 switches} conectados a 3 clientes cada uno,
lo que nos da un total de 21 clientes en todo el esquema de red.
\\

Después lo que hicimos fue agregar el esquema a los \textbf{routers}, que en
este caso usamos el \textbf{2950}, también poner la \textbf{tarjeta ethernet},
y hacer este mismo proceso con todos los \textbf{routers}. Después de esto, se
tiene que conectar los \textbf{switches} por medio de \textbf{ethernet}, y el
resto por serial, y así mismo configurar cada uno, en este caso configuramos
los \textbf{routers} con \textbf{Ethernet/0}, y los \textbf{switches} en
\textbf{serial0/1}.

Después de esto lo que hacemos es que en cada \textbf{router} le asignamos las
rutas estáticas, ósea, asignarle la red remota y su dirección IP a la cual se le
enviaran los paquetes hacia otra red, y por último activar/encender el puerto
(\textbf{On}).
\\

Ahora bien, en comparación al trabajo pasado se añade a la sede de Bogotá una
arquitectura de tres capas. yada yada yada.
\\

Entonces primero se añadió el switch 3560-24PS para empezar el core con la
dirección X, después se añadió otro de estos switches para la capa de
distribución, y en la capa de acceso se añadieron 2 switches x, y, z, los
cuales tienen acceso a la VLAN 10 y la VLAN 20, después se creó la VLAN 10 que
está destinada para el servidor principal, y la VLAN 20 está destinada para los 
servidores de backup.
\\

Al igual que en la parte anterior con Bogotá, en Medellín se añadieron nuevos
nodos, sine embargo acá se usará la arquitectura de 2 capas. Entonces primero 
también se añadió el switch 3560-24PS para el Core/collapsed, 



\subsection{NAT}

