\section{Introducción}

En el siguiente proyecto, nos embarcaremos en la tarea de mejorar el diseño e
implementación de una infraestructura de red robusta y escalable para una empresa multinacional
que ha establecido su sede principal en la ciudad de Bogotá en la actividad
número 2 de la materia. Nuestro principal
objetivo será proporcionar una arquitectura de red que garantice un alto
rendimiento y una conectividad ininterrumpida, teniendo en cuenta la
expectativa de un alto volumen de tráfico para su página web usando 2
arquitecturas adicionales (\textit{Two tier} y \textit{three tier architecture})
y cómo estas impactan la arquitectura previa.
\\

Procederemos a diseñar una topología de red en estrella, una estructura que se
destaca por su confiabilidad y capacidad para administrar eficientemente el
tráfico de red. Con este enfoque, cada sede de la empresa estará conectada a un
nodo central, en este caso, la sede principal en Bogotá, permitiendo una
comunicación fluida entre todas las localidades.
En la sede de Medellín se implementará la arquitectura de red de
\textit{Core / Collapsed}, mientras que en la sede de Bogotá se incorporará la
arquitectura de red de tres capas \textit{(Core, Distribución y Acceso)}.
\\

La página web de la empresa será desplegada en un clúster de servidores web
para asegurar un rendimiento óptimo, incluso ante un alto tráfico. Esto no
solo mejorará la disponibilidad del sitio, sino que también ofrecerá una mayor
tolerancia a fallos y flexibilidad, ya que los recursos pueden ser ajustados de
acuerdo con la demanda.
\\

Además, como líderes del equipo de infraestructura, estaré al frente de la
supervisión del despliegue, asegurándome de que todas las partes del sistema
funcionen en armonía y cumplan con las expectativas de la empresa.
Aquí radica la importancia de las arquitecturas \textit{Core / Collapsed} y de
tres capas, que permitirán optimizar la gestión del tráfico, garantizar la
robustez de la red y ofrecer escalabilidad para el crecimiento futuro de la
empresa.
\\

En el transcurso de este proyecto, describiré en detalle el proceso de diseño
de la topología de red, la implementación del clúster de servidores web, así
como las medidas que se tomarán para garantizar la seguridad, la escalabilidad
y la eficiencia de la infraestructura de red.
\\

Estoy convencido de que, al final de este trabajo, habremos creado una
infraestructura sólida y confiable que permitirá a la empresa operar su página
web de manera eficiente y sin interrupciones, sin importar la cantidad de
tráfico que reciba.
