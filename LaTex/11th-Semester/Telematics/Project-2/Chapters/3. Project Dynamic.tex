\section{Proyecto Dinámico}

\subsection{OSPF}

En este proyecto, uno de los aspectos fundamentales para garantizar la robustez
y la escalabilidad de nuestra infraestructura de red es la implementación de
una estrategia de enrutamiento eficiente. A lo largo de este trabajo, hemos
enfatizado la importancia del VLSM y el subnetting en el diseño de nuestra red.
Sin embargo, para añadir una capa adicional de eficiencia y flexibilidad, hemos
optado por implementar el Protocolo de Estado de Enlace Abierto
(Open Shortest Path First, OSPF).
\\

OSPF es un protocolo de enrutamiento de estado de enlace, que es especialmente
adecuado para redes de gran tamaño, como la que estamos diseñando para esta
empresa multinacional. Al combinarlo con nuestra estrategia de VLSM y
subnetting, y dada nuestra topología de red en estrella y el esperado alto
volumen de tráfico de la página web de la empresa, OSPF se presenta como una
solución ideal para gestionar eficazmente el tráfico de red y asegurar una
conectividad ininterrumpida.
\\

La importancia de OSPF en este proyecto radica en su capacidad para determinar
la ruta más corta y menos congestionada para el envío de paquetes de datos en
la red. Esto se logra mediante el algoritmo de Dijkstra, que OSPF utiliza para
calcular las rutas más eficientes. Esta capacidad es particularmente relevante
dada la estructura de nuestra red, que conecta todas las sedes de la empresa a
un nodo central, en este caso, la sede principal en Bogotá.
\\

Además, OSPF es un protocolo de enrutamiento dinámico, lo que significa que es
capaz de adaptarse rápidamente a los cambios en la red. En caso de un fallo de
red o un cambio en la topología de la red, OSPF puede ajustar rápidamente las
rutas de los paquetes de datos para evitar la interrupción de la conectividad.
Esta capacidad de recuperación y adaptabilidad, combinada con la flexibilidad
proporcionada por VLSM y subnetting, es esencial para mantener la disponibilidad
del sitio web de la empresa y garantizar un rendimiento óptimo,
independientemente del volumen de tráfico.
\\

En las siguientes secciones, detallaré cómo hemos implementado OSPF en nuestra
infraestructura de red, y cómo esta decisión, en combinación con el uso de VLSM
y subnetting, contribuye a los objetivos de robustez, escalabilidad y
eficiencia de nuestro diseño de red. Estamos convencidos de que, con la
implementación de OSPF, estaremos mejor equipados para enfrentar los desafíos
que presenta la gestión de una red de gran tamaño y alto tráfico.

\subsection{Implementación}

Como ya se sabe, nuestra infraestructura de red se basa en una topología de
estrella, donde la ciudad de Bogotá actúa como el nodo central o router
principal. Desde Bogotá, las conexiones se extienden a través de cables seriales
hacia las demás ciudades, que incluyen Medellín, Barranquilla, Rionegro, Cali,
Popayán, entre otras, mientras que las conexiones internas se gestionan mediante
cables ethernet.
\\

Es importante destacar que, a pesar de estar interconectados, cada uno de los
routers tiene un direccionamiento de red único. Este diseño incorpora un total
de 7 switches, cada uno de los cuales está conectado a 3 clientes, lo que suma
un total de 21 clientes en toda la red.
\\

Para cada router, asignamos una dirección IP y su correspondiente máscara de
subred. Con estas configuraciones, la puerta de enlace predeterminada
corresponde a la dirección IP configurada en los routers. Los switches, por otro
lado, actúan como puntos de conexión y, por lo tanto, no requieren configuraciones adicionales.
\\

Realizamos la configuración de los routers, específicamente el modelo 1841, a
través de la terminal. Asimismo, fue necesario añadir dos interfaces seriales a
la configuración de la interfaz web para garantizar una comunicación efectiva en
toda la red.
