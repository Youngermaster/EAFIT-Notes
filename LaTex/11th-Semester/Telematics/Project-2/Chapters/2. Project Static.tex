\section{Proyecto Estático}

\subsection{VLSM}

Ahora bien, antes de comenzar a desarrollar e implementar nuestra solución,
lo mejor es definir completamente la infraestructura y primero usaremos
VLSM \textit{(Variable Length Subnet Mask)}. Entonces en la siguiente sección,
nos centraremos en un componente crítico de la gestión de redes: el Subneteo de
Longitud de Máscara Variable \textit{(Variable Length Subnet Masking, VLSM)}.
\\

VLSM es una técnica que permite dividir eficientemente el espacio de
direcciones IP en subredes de diferentes tamaños, según las necesidades
específicas de cada subred. Es una mejora significativa respecto al subneteo de
longitud de máscara fija, que obliga a todas las subredes a ser del mismo
tamaño, lo que puede resultar en una utilización ineficiente del espacio de
direcciones IP.
\\

En el contexto de nuestro proyecto, VLSM es esencial por varias razones.
Primero, nos permite maximizar el uso del espacio de direcciones IP asignado a
la empresa. Dado que las diferentes sedes y departamentos de la empresa pueden
tener diferentes necesidades en términos de cantidad de hosts requeridos, el uso
de VLSM nos permite asignar el espacio de direcciones de manera eficiente, 
minimizando el desperdicio.
\\

En segundo lugar, VLSM puede mejorar el rendimiento de la red al reducir la
cantidad de tráfico de enrutamiento innecesario. Al diseñar subredes que se
ajusten a las necesidades específicas de cada sede o departamento, podemos
minimizar la cantidad de tráfico de red que necesita ser enrutado a través de la
red principal, lo que a su vez puede mejorar el rendimiento general de la red.
\\

Finalmente, el uso de VLSM puede ayudar a mejorar la seguridad de la red. Al
segmentar la red en subredes de diferentes tamaños, podemos aplicar políticas
de seguridad más granulares, lo que nos permite tener un control más preciso
sobre quién puede acceder a qué partes de la red.
\\

En resumen, VLSM es una herramienta poderosa que nos permite diseñar e
implementar una red que es eficiente, segura y capaz de satisfacer las
necesidades específicas de la empresa (del ejercicio). En las siguientes
páginas, detallaremos cómo implementaremos VLSM en el diseño de nuestra red.
\\

Ahora bien, tenemos 7 redes de oficinas

\subsection{Implementación}

Ahora bien, para implementarlo realizamos lo siguiente: