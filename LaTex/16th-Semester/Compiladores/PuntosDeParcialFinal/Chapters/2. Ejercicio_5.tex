\section{[20 pts] Ejercicio 5}

\subsection{Enunciado}
Considere el siguiente lenguaje fuente $L$ sobre el alfabeto $\Sigma = \{a, b\}$: todas las cadenas que contienen una o más repeticiones del patrón $ab$.

\begin{enumerate}[label=\alph*.]
    \item Diseñe una gramática libre de contexto $G$ que genere $L$.
    \item Extienda dicha gramática para construir una gramática de traducción $G_{\tau}$ que, por cada aparición del patrón $ab$, produzca como salida el símbolo $X$; es decir, si la entrada es $ababab$, entonces la salida es $XXX$.
    \item Escriba una expresión de traducción regular equivalente a la gramática $G_{\tau}$.
\end{enumerate}

\subsection{a. Gramática libre de contexto $G$}
Una gramática libre de contexto $G = (\Sigma, N, P, S)$ que genera $L$ es:

\[
P: \begin{cases} 
S \rightarrow aS \mid bS \mid abF \\
F \rightarrow aF \mid bF \mid abF \mid \epsilon
\end{cases}
\]

Con símbolos no terminales $N=\{S, F\}$. 
\textbf{Justificación:} La producción $S$ consume cualquier prefijo hasta encontrar el primer patrón obligatorio $ab$, momento en el cual transiciona a $F$.
El no terminal $F$ permite generar repeticiones adicionales del patrón o terminar la cadena con $\epsilon$, asegurando así la condición de una o más repeticiones \cite{crespi2019formal}.

\subsection{b. Gramática de traducción $G_{\tau}$}
Extendemos $G$ para definir la gramática de traducción $G_{\tau}=(V, \Sigma, \Delta, P_{\tau}, S)$ definida a través de las producciones:

\[
P_{\tau}: \begin{cases} 
S \rightarrow \frac{a}{\epsilon}S \mid \frac{b}{\epsilon}S \mid \frac{ab}{X}F \\
F \rightarrow \frac{a}{\epsilon}F \mid \frac{b}{\epsilon}F \mid \frac{ab}{X}F \mid \epsilon
\end{cases}
\]

Con símbolos no terminales $V=\{S, F\}$, símbolos terminales de destino $\Delta=\{X\}$ y el alfabeto terminal de pares $C = \{ \frac{a}{\epsilon}, \frac{b}{\epsilon}, \frac{ab}{X} \} \subseteq \Sigma^* \times \Delta^*$.

\subsubsection*{Esquema de Traducción}
A partir de $G_{\tau}$ obtenemos el siguiente esquema, separando la sintaxis de origen y la semántica de destino \cite{aho2007compilers}:

\begin{table}[H]
\centering
\begin{tabular}{l|l}
\toprule
\textbf{Gramática de Origen} & \textbf{Gramática de Destino} \\
\midrule
$S \rightarrow aS \mid bS \mid abF$ & $S \rightarrow \epsilon S \mid \epsilon S \mid XF$ \\
$F \rightarrow aF \mid bF \mid abF \mid \epsilon$ & $F \rightarrow \epsilon F \mid \epsilon F \mid XF \mid \epsilon$ \\
\bottomrule
\end{tabular}
\caption{Esquema de traducción asociado a $G_{\tau}$}
\end{table}

\subsubsection*{Verificación de la Traducción}
Procesemos la traducción de la cadena de entrada $w = ababab$. Realizamos la derivación utilizando los pares de traducción:

\[
\begin{aligned}
S &\rightarrow \frac{ab}{X}F \\
  &\rightarrow \frac{ab}{X} \frac{ab}{X}F \\
  &\rightarrow \frac{ab}{X} \frac{ab}{X} \frac{ab}{X}F \\
  &\rightarrow \frac{ab}{X} \frac{ab}{X} \frac{ab}{X} \epsilon \\
  &= \frac{ababab}{XXX} \equiv z
\end{aligned}
\]

Sean $h_{\Sigma}$ y $h_{\Delta}$ los homomorfismos alfabéticos que proyectan sobre los alfabetos de origen y destino respectivamente:
\[ h_{\Sigma}(z) = ababab, \quad h_{\Delta}(z) = XXX \]
De esta manera, $(ababab, XXX) \in \rho_{\tau}$, confirmando que la traducción es correcta.

\subsection{c. Expresión de traducción regular ($e_{\tau}$)}
Primero, determinamos una expresión regular capaz de generar el lenguaje fuente $L(G)$:
\[ (a|b)^* ab (a|b)^* \]

Una expresión regular de traducción $e_{\tau}$ equivalente es:
\[
e_{\tau} = \underbrace{\left(\frac{a}{\epsilon} \mid \frac{b}{\epsilon}\right)^*}_{\text{Prefijo}} \cdot \frac{ab}{X} \cdot \underbrace{\left[ \left(\frac{a}{\epsilon} \mid \frac{b}{\epsilon}\right)^* \frac{ab}{X} \right]^*}_{\text{Repeticiones}} \cdot \underbrace{\left(\frac{a}{\epsilon} \mid \frac{b}{\epsilon}\right)^*}_{\text{Sufijo}}
\]
Esta estructura asegura que el primer $ab$ se traduce, y cualquier $ab$ subsecuente también se traduce a $X$.
