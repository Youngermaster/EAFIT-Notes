\section{FASE 2: Formulación y evaluación ambiental del proyecto}

\subsection{Estudio sectorial y de mercado}
\begin{itemize}
  \item \textbf{Qué vende el proyecto y a quién.} El Parque Solar Castilla es una instalación fotovoltaica de aproximadamente 21 MWp destinada a la autogeneración a gran escala, suministrando energía al propio consumo del complejo industrial Castilla mediante un contrato PPA interno Ecopetrol-AES.  
  El objetivo principal es reducir el costo energético, la exposición a la volatilidad del mercado eléctrico y la huella de carbono de las operaciones.

  \item \textbf{Demanda y competencia.} En Colombia, la penetración de fuentes no convencionales de energía renovable (FNCE) como la solar y eólica ha crecido sostenidamente.  
  La energía fotovoltaica presenta un costo nivelado de generación (LCOE) altamente competitivo y tiempos de construcción reducidos, lo que la posiciona como una alternativa atractiva, especialmente para proyectos con demanda anclada (autoconsumo industrial).

  \item \textbf{Ingresos y esquema comercial.} El proyecto opera bajo un esquema bilateral (PPA) de largo plazo, sin depender de subastas públicas.  
  El flujo de ingresos es estable y predecible, con bajo riesgo frente a fluctuaciones del mercado eléctrico, al tratarse de un modelo \textit{behind-the-meter}.

  \item \textbf{Riesgos sectoriales.} Se consideran la variabilidad de irradiancia, degradación de módulos (aprox.\ 0.5\% anual), posibles restricciones operativas (\textit{curtailment}) por red interna y cambios regulatorios asociados a medición, acceso o autogeneración.

  \item \textbf{Fuentes técnicas típicas.}  
  Atlas de Radiación Solar (IDEAM/UPME), bases de datos horarios de irradiancia, curvas de producción fotovoltaica, disponibilidad y capacidad de red interna, Plan de Ordenamiento Territorial (POT) municipal y planes energéticos corporativos.

  \item \textbf{Base normativa. \cite{Ley1715_2014,CREG030_2018,D2041_2014}} Ley 1715 de 2014 (promoción de FNCE y autogeneración), CREG 030 de 2018 (autogeneración), Ley 142 de 1994 (servicios públicos), y Decreto 2041 de 2014 (licenciamiento ambiental).
\end{itemize}

\subsection{Estudio técnico y de localización}

\textbf{Criterios de localización.}
\begin{itemize}
  \item Alta irradiancia y baja nubosidad estacional.
  \item Predio ubicado dentro del complejo industrial existente, minimizando conflictos de uso del suelo.
  \item Topografía suave y suelos con adecuada capacidad portante (según estudios geotécnicos).
  \item Conectividad eléctrica cercana (barra/subestación interna) con pérdidas mínimas.
  \item Evitar rondas hídricas, humedales y fragmentos de bosque; establecer franjas de manejo y pasos de fauna.
\end{itemize}

\textbf{Diseño y tecnología.}
\begin{itemize}
  \item Módulos fotovoltaicos mono-PERC o bifaciales con estructuras fijas o seguidores de un eje (según evaluación LCOE).
  \item Inversores tipo string o central, transformadores elevadores y sistema SCADA para control y despacho.
  \item Drenajes perimetrales, control de escorrentía, manejo de polvo y vías internas en recebo.
  \item Cerramiento perimetral, control de acceso y plan de prevención de incendios (eléctricos y vegetativos).
\end{itemize}

\textbf{Construcción (metodología).}
\begin{itemize}
  \item Desbroce controlado del terreno y rescate de epífitas o individuos protegidos.
  \item Rescate y reubicación de fauna antes del hincado de estructuras.
  \item Zonas de acopio temporal impermeabilizadas y manejo de residuos bajo PGIRS y RESPEL (si aplica).
  \item Implementación del programa de manejo arqueológico y protocolo de hallazgo fortuito.
\end{itemize}

\textbf{Operación.}
\begin{itemize}
  \item Limpieza de paneles con consumo racional de agua o técnicas en seco (preferencia: \textit{cero vertimiento}).
  \item Monitoreo de generación e indisponibilidades.
  \item Mantenimiento preventivo y control periódico de vegetación.
\end{itemize}

\textbf{Base normativa. \cite{D1076_2015,Res631_2015,D1791_1996,D1608_1978,ICANH_fortuito,NSR10_2010}.}  
Decreto 1076 de 2015 (ocupación de cauce, vertimientos, aprovechamiento forestal, fauna),  
Resolución 631 de 2015 (parámetros de vertimiento),  
Decreto 1791 de 1996 (aprovechamiento forestal),  
Decreto 1608 de 1978 (fauna),  
Ley 1185 de 2008 e ICANH (patrimonio arqueológico),  
Norma NSR-10 (diseño estructural), POT municipal (uso del suelo).


\subsection{Estudio administrativo}

\textbf{Gobernanza del proyecto.}
\begin{itemize}
  \item \textbf{Estructura:} Propietario/ancla de demanda (Ecopetrol) y desarrollador/operador (AES).
  \item \textbf{Roles ambientales:} Titular de la licencia o permisos, interventoría ambiental, coordinación HSE en obra y operación, y gestores autorizados de residuos.
  \item \textbf{Stakeholders:} Alcaldía, comunidad local, inspectores de trabajo, autoridad ambiental (ANLA o CORMACARENA), ICANH, bomberos y gestores de residuos.
\end{itemize}

\textbf{Plan de relación con grupos de interés.}
\begin{itemize}
  \item Socialización previa del proyecto y mecanismos de atención de quejas y reclamos.
  \item Priorización de mano de obra y compras locales.
  \item Programas de educación ambiental y seguridad eléctrica comunitaria.
\end{itemize}

\textbf{Cronograma administrativo típico.}
\begin{itemize}
  \item Prefactibilidad: 6-9 meses (recurso solar, terreno, conexión, línea base ambiental).
  \item Factibilidad y licenciamiento: 6-12+ meses (EIA/EMAS, permisos específicos, concertaciones).
  \item Construcción: 6-12 meses.
  \item Operación: 15-30 años.
  \item Cierre/abandono: 3-12 meses.
\end{itemize}

\textbf{Base normativa. \cite{Ley99_1993,D2041_2014}}  
Ley 99 de 1993 (SINA y competencias),  
Decreto 2041 de 2014 (trámite de licencias),  
Ley 80 de 1993 (contratación pública, si aplica),  
Decreto 1072 de 2015 (Sistema de Gestión de Seguridad y Salud en el Trabajo).


\subsection{Estudio legal}

\textbf{Requerimiento de licencia ambiental.}  
Dado que la capacidad del parque supera los 10 MW, se modela bajo el régimen de \textbf{Licencia Ambiental Global}, de conformidad con el Decreto 2041 de 2014 y su incorporación al Decreto 1076 de 2015.  
Este instrumento integra los permisos asociados y facilita la gestión unificada del Plan de Manejo Ambiental (PMA).

\textbf{Permisos integrados o asociados (según diseño final).}
\begin{itemize}
  \item \textbf{Aprovechamiento forestal:} Decreto 1791 de 1996.
  \item \textbf{Fauna (rescate y reubicación):} Decreto 1608 de 1978.
  \item \textbf{Vertimientos:} Decreto 3930 de 2010 y Resolución 631 de 2015 (si hay descargas a cuerpos de agua o suelo).
  \item \textbf{Concesión de aguas:} Decreto 1541 de 1978 (si se capta agua para limpieza).
  \item \textbf{Ocupación de cauce:} Decreto 1076 de 2015, Art. 2.2.3.2.9.1.
  \item \textbf{Ruido:} cumplimiento de Resolución 627 de 2006.
  \item \textbf{Aire:} Permisos sólo si hay fuentes fijas (p. ej., generadores diésel permanentes); Decreto 948 de 1995.
  \item \textbf{Patrimonio arqueológico:} PMA-ICANH y protocolo de hallazgo fortuito; Ley 1185 de 2008.
  \item \textbf{Residuos:} PGIRS y, si aplica, registro RESPEL (Decreto 4741 de 2005).
\end{itemize}

\textbf{Autoridad competente.}  
Por su alcance y tipología, la autoridad principal sería la ANLA; si se determina competencia regional, aplicaría CORMACARENA (Meta).

\textbf{Obligaciones típicas en la resolución.}
\begin{itemize}
  \item Ejecución integral del PMA (suelo, agua, aire, flora, fauna, social, arqueología y residuos).
  \item Reportes de monitoreo y cumplimiento según la frecuencia establecida (ICA, parámetros, etc.).
  \item Cumplimiento de compensaciones (forestal, biodiversidad o social).
  \item Aprobación y ejecución del Plan de Abandono antes del cierre definitivo.
\end{itemize}

\textbf{Norma base. \cite{D2041_2014,D1076_2015,D1541_1978,D3930_2010,Res631_2015,D1791_1996,D1608_1978,Res627_2006,D948_1995,ICANH_fortuito}} D.~2041/2014; D.~1076/2015; D.~1541/1978; D.~3930/2010; Res.~631/2015; D.~1791/1996; D.~1608/1978; Res.~627/2006; D.~948/1995; Ley 1185/2008; D.~4741/2005.

\subsection{Estudio económico}

Los costos ambientales del \textbf{Parque Solar Castilla} se estructuran en función del ciclo de vida del proyecto (preparación, construcción, operación y cierre) y de lo que exige la normatividad colombiana para proyectos que requieren licencia ambiental (Ley 99 de 1993, Decreto 2041 de 2014 y Decreto 1076 de 2015). A continuación se detallan los costos estimados reportados en el ejercicio:

\begin{table}[h]
\centering
\caption{Costos ambientales estimados del proyecto}
\begin{tabular}{|p{10cm}|r|}
\hline
\textbf{Costo ambiental} & \textbf{Precio estimado (COP)} \\ \hline
Elaboración del Estudio de Impacto Ambiental (EIA) & \$ 250{,}000{,}000 \\ \hline
Gestión de permisos ambientales (ANLA, CORMACARENA, ICA, ICANH, etc.) & \$ 80{,}000{,}000 \\ \hline
Consultorías técnicas (suelo, biodiversidad, uso del territorio) & \$ 120{,}000{,}000 \\ \hline
Capacitaciones ambientales iniciales al personal de obra & \$ 20{,}000{,}000 \\ \hline
Medidas de prevención y mitigación (control de polvo, ruido, vertimientos, residuos) & \$ 1{,}700{,}000{,}000 \\ \hline
Adecuación de zonas verdes y revegetalización & \$ 250{,}000{,}000 \\ \hline
Sistemas de drenaje y control de escorrentías & \$ 180{,}000{,}000 \\ \hline
Equipos para monitoreo de agua, aire y suelo & \$ 150{,}000{,}000 \\ \hline
Monitoreo ambiental continuo (ruido, aire, agua, suelo) & \$ 900{,}000{,}000 \\ \hline
Personal ambiental (ingeniero ambiental y técnico de planta) & \$ 300{,}000{,}000 \\ \hline
Gestión integral de residuos y mantenimiento sostenible & \$ 200{,}000{,}000 \\ \hline
Reportes y auditorías ambientales & \$ 50{,}000{,}000 \\ \hline
Siembra de árboles / revegetalización compensatoria & \$ 100{,}000{,}000 \\ \hline
Programas comunitarios y educativos & \$ 50{,}000{,}000 \\ \hline
Contribuciones a fondos ambientales regionales & \$ 30{,}000{,}000 \\ \hline
Desmontaje y disposición final de paneles y estructuras & \$ 600{,}000{,}000 \\ \hline
Gestión de residuos electrónicos & \$ 100{,}000{,}000 \\ \hline
Restauración del terreno & \$ 200{,}000{,}000 \\ \hline
Evaluación final ambiental & \textit{Sin información} \\ \hline
\rowcolor[HTML]{FFF2CC}
\textbf{Total estimado de costos ambientales} & \textbf{\$ 5{,}280{,}000{,}000} \\ \hline
\end{tabular}
\end{table}

El valor total del proyecto tiene aproximadamente un valor de USD 20 millones, convertido a pesos colombianos con una tasa de $1$ USD $=$ $3,910{.}85$ COP es de:

\begin{center}
\textbf{Valor total del proyecto: \$ 77{,}495{,}600{,}000}
\end{center}

Por lo tanto, la participación de los costos ambientales dentro del costo total del proyecto es:

\begin{center}
\[
\frac{5{,}280{,}000{,}000}{77{,}495{,}600{,}000} \approx 0.068 \; \Rightarrow \; \textbf{6.8\%}
\]
\end{center}

Esto quiere decir que cerca del \textbf{7\% del CAPEX} está asociado directamente a obligaciones ambientales (estudios, permisos, monitoreo, compensaciones y cierre). Ese porcentaje está dentro del rango que normalmente se ve para proyectos de energía renovable en Colombia cuando hay que hacer revegetalización, manejo de escorrentías y, sobre todo, cuando el plan de cierre se calcula desde el principio (como lo exige el art. 2.2.2.3.6. del Decreto 1076 de 2015 para proyectos con licenciamiento).

\paragraph{Relación con la norma.} Estos rubros aparecen porque:
\begin{itemize}
  \item el EIA y las consultorías técnicas son requisito del \textbf{Decreto 2041 de 2014};
  \item las medidas de prevención, monitoreo y reportes se derivan del \textbf{Plan de Manejo Ambiental} que forma parte de la resolución de licencia (Ley 99 de 1993, art. 57);
  \item las compensaciones forestales y los programas comunitarios responden al principio de participación y al deber de compensar la afectación de cobertura (Decreto 1791 de 1996 y Ley 99 de 1993);
  \item los costos de desmontaje y restauración son coherentes con la obligación de incluir el \emph{plan de abandono} en el trámite de licencia (Decreto 2041 de 2014, art. 12).
\end{itemize}

\subsection{Estudio financiero}

Para efectos del trabajo académico se mantiene el mismo esquema financiero que veníamos usando: una inversión total de USD 20 millones (CAPEX), una vida útil de 15 años, un ingreso inicial de \$3{,}000{,}000 y un OPEX inicial de \$500{,}000 con crecimientos moderados. A esto se le suma, al final, el costo de abandono que ya está cuantificado en \$1{,}000{,}000 dentro del modelo.

\subsubsection*{Supuestos financieros principales}
\begin{table}[h]
\centering
\begin{tabular}{|l|r|p{6cm}|}
\hline
\textbf{Parámetro} & \textbf{Valor} & \textbf{Justificación} \\ \hline
Inversión total del proyecto & \$ 20{,}000{,}000 & CAPEX típico de parque FV de \(\sim\)21 MWp con conexión interna. \\ \hline
Vida útil / horizonte de evaluación & 15 años & Coincide con el horizonte del PPA. \\ \hline
Ingreso año 1 & \$ 3{,}000{,}000 & Venta de energía al autoproductor (Ecopetrol) a una tarifa estable. \\ \hline
Crecimiento anual de ingresos & 2\% & Indexación por IPC / actualización de tarifa del contrato. \\ \hline
Costos operativos iniciales (OPEX) & \$ 500{,}000 & Operación, mantenimiento, personal y seguimiento ambiental. \\ \hline
Crecimiento anual de costos & 3\% & Mantenimiento creciente y actualización salarial. \\ \hline
Degradación de paneles & -0.5\% anual & Pérdida de eficiencia típica de módulos FV. \\ \hline
Plan de abandono (año 15) & \$ 1{,}000{,}000 & Desmontaje, gestión de RAEE y restauración del terreno. \\ \hline
\end{tabular}
\end{table}

\subsubsection*{Flujo de caja referencial (años 0--15)}

\begin{table}[h]
\centering
\small
\begin{tabular}{|c|l|r|r|r|r|}
\hline
\textbf{Año} & \textbf{Concepto} & \textbf{Entradas} & \textbf{Salidas} & \textbf{Flujo} & \textbf{Flujo acumulado} \\ \hline
0  & Inversión inicial y estudios & 0 & \$ 20{,}000{,}000 & -\$ 20{,}000{,}000 & -\$ 20{,}000{,}000 \\ \hline
1  & EIA e inicio de operación    & \$ 3{,}000{,}000 & \$ 700{,}000  & \$ 2{,}300{,}000 & -\$ 17{,}700{,}000 \\ \hline
2  & Operación                    & \$ 3{,}060{,}000 & \$ 515{,}000  & \$ 2{,}545{,}000 & -\$ 15{,}155{,}000 \\ \hline
3  & Operación                    & \$ 3{,}121{,}200 & \$ 530{,}450  & \$ 2{,}590{,}750 & -\$ 12{,}564{,}250 \\ \hline
4  & Operación                    & \$ 3{,}183{,}624 & \$ 546{,}364  & \$ 2{,}637{,}261 & -\$ 9{,}926{,}990 \\ \hline
5  & Operación                    & \$ 3{,}247{,}296 & \$ 562{,}754  & \$ 2{,}684{,}542 & -\$ 7{,}242{,}447 \\ \hline
6  & Operación                    & \$ 3{,}312{,}242 & \$ 579{,}637  & \$ 2{,}732{,}605 & -\$ 4{,}509{,}842 \\ \hline
7  & Operación                    & \$ 3{,}378{,}487 & \$ 597{,}026  & \$ 2{,}781{,}461 & -\$ 1{,}728{,}381 \\ \hline
8  & Operación                    & \$ 3{,}446{,}057 & \$ 614{,}937  & \$ 2{,}831{,}120 & \$ 1{,}102{,}739 \\ \hline
9  & Operación                    & \$ 3{,}514{,}978 & \$ 633{,}385  & \$ 2{,}881{,}593 & \$ 3{,}984{,}332 \\ \hline
10 & Operación                    & \$ 3{,}585{,}278 & \$ 652{,}387  & \$ 2{,}932{,}891 & \$ 6{,}917{,}223 \\ \hline
11 & Operación                    & \$ 3{,}656{,}983 & \$ 671{,}958  & \$ 2{,}985{,}025 & \$ 9{,}902{,}248 \\ \hline
12 & Operación                    & \$ 3{,}730{,}123 & \$ 692{,}117  & \$ 3{,}038{,}006 & \$ 12{,}940{,}254 \\ \hline
13 & Operación                    & \$ 3{,}804{,}725 & \$ 712{,}880  & \$ 3{,}091{,}845 & \$ 16{,}032{,}099 \\ \hline
14 & Operación                    & \$ 3{,}880{,}820 & \$ 734{,}267  & \$ 3{,}146{,}553 & \$ 19{,}178{,}652 \\ \hline
15 & Operación + plan de abandono & \$ 3{,}958{,}436 & \$ 1{,}756{,}295  & \$ 2{,}202{,}141 & \$ 21{,}380{,}794 \\ \hline
\end{tabular}
\end{table}

\paragraph{Comentario financiero.} Con estos supuestos el proyecto logra recuperar la inversión inicial hacia el año 8--9 (cuando el flujo acumulado pasa a ser positivo), lo cual es coherente con un parque solar con PPA estable. Si al modelo se le descuenta el componente ambiental calculado (5.28 billones COP \(\approx\) 6.8\% del valor del proyecto), se ve que el proyecto sigue siendo viable, lo que demuestra que \textbf{incorporar la variable ambiental desde la formulación no lo vuelve inviable}, sólo obliga a programar esos costos desde el año 0.

