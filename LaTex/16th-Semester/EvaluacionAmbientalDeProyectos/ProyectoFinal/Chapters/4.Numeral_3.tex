\section{FASE 2: Formulación y evaluación ambiental del proyecto}

\subsection{Estudio sectorial y de mercado}
\begin{itemize}
  \item \textbf{Producto y cliente.} Autogeneración FV a gran escala para el complejo Castilla mediante \emph{PPA} interno; reducción de compras a red y de exposición a precios.
  \item \textbf{Demanda/competencia.} Alta irradiancia regional (\emph{IDEAM/UPME}); tendencia nacional de FNCE (Ley 1715/2014); LCOE competitivo; construcción rápida.
  \item \textbf{Riesgos sectoriales.} Variabilidad de irradiancia, degradación de módulos ($\sim$0.5\% anual), restricciones operativas (curtailment), cambios regulatorios y de conexión.
  \item \textbf{Inteligencia de entorno.} Revisión de POT, disponibilidad de predios, vías de acceso, ofertas de conexión, inventarios de biodiversidad, fuentes hídricas y comunidades.
\end{itemize}

\medskip
\textbf{Normativa guía.} Ley 1715/2014, Ley 142/1994 (servicios), circulares CREG aplicables; Decreto 2041/2014 (licenciamiento).

\subsection{Estudio técnico y de localización}
\begin{itemize}
  \item \textbf{Criterios de emplazamiento.} Irradiancia, pendiente baja, suelos estables, evitar rondas/fragmentos, cercanía a barras de conexión, servidumbres claras.
  \item \textbf{Diseño.} Módulos mono-PERC/bifaciales; estructuras (fijas/seguidor); inversores; trafo elevador; SCADA; drenajes y control de escorrentías; cortafuegos.
  \item \textbf{Construcción.} Desbroce selectivo, control de polvo y sedimentos, acopios impermeabilizados, PGIRS/RESPEL, protocolo arqueológico, señalización HSE.
  \item \textbf{Operación.} Limpieza en seco o con uso eficiente de agua; \textit{cero vertimiento}; mantenimiento preventivo; control de vegetación.
\end{itemize}
\textbf{Norma base.} D.~1076/2015; Res.~631/2015 (si hay vertimientos); D.~1791/1996 (forestal); D.~1608/1978 (fauna); Ley 1185/2008 (ICANH).

\subsection{Estudio administrativo}
\begin{itemize}
  \item \textbf{Gobernanza.} Titular; operador con rol HSE; interventoría ambiental; gestores autorizados; articulación con Alcaldía, comunidad, ANLA/CORMACARENA e ICANH.
  \item \textbf{Participación.} Socialización temprana; \emph{PQRS}; empleo y compras locales; educación ambiental.
  \item \textbf{Cronograma de referencia.} Prefactibilidad (6-9 m), factibilidad/licencia (6-12+ m), construcción (6-12 m), operación (15-30 a), abandono (3-12 m).
\end{itemize}
\textbf{Norma guía.} Ley 99/1993; D.~2041/2014; D.~1072/2015 (SG-SST).

\subsection{Estudio legal}
\begin{itemize}
  \item \textbf{Instrumento principal.} Licencia Ambiental (licencia global) integrando permisos/autoridades asociadas (forestal, fauna, cauce, concesión/vertimientos si aplica, aire si hay fuentes fijas, arqueología, residuos).
  \item \textbf{Competencia.} ANLA (por alcance/typología); si fuese regional en Meta: CORMACARENA.
  \item \textbf{Obligaciones típicas.} PMA, monitoreos/ICA, compensaciones, plan de abandono, pólizas y garantías.
\end{itemize}

\textbf{Norma base.} D.~2041/2014; D.~1076/2015; D.~1541/1978; D.~3930/2010; Res.~631/2015; D.~1791/1996; D.~1608/1978; Res.~627/2006; D.~948/1995; Ley 1185/2008; D.~4741/2005.

\subsection{Estudio económico: costos ambientales}
\begin{itemize}
\item \textbf{Planeación y licencias:} EIA/EMAS, trámites, estudios (suelo, biodiversidad, arqueología), capacitaciones HSE.
\item \textbf{Implementación:} control de polvo/ruido, manejo de sedimentos, revegetalización, drenajes, equipos de monitoreo.
\item \textbf{Operación:} monitoreo continuo (agua/ruido/aire), O\&M ambiental, personal, reportes e ICA.
\item \textbf{Compensaciones:} forestal y programas comunitarios/educación.
\item \textbf{Abandono:} desmantelamiento, \textsc{Raee}/\textsc{Respel}, restauración y evaluación final.
\end{itemize}

\subsection{Estudio financiero: supuestos y flujo}

\subsubsection*{Parámetros clave}
\begin{table}[H]
\centering
\begin{tabular}{|l|r|l|}
\hline
\textbf{Parámetro} & \textbf{Valor} & \textbf{Justificación} \\ \hline
Inversión total & \$\,20{,}000{,}000 & CAPEX estimado del parque y conexión interna \\ \hline
Vida útil & 15 años & Horizonte contractual y vida útil de equipos principales \\ \hline
Ingreso año 1 & \$\,3{,}000{,}000 & Producción esperada $\times$ tarifa PPA \\ \hline
Crecimiento ingresos & 2\% anual & Indexación/IPC \\ \hline
OPEX año 1 & \$\,500{,}000 & O\&M, personal, seguros, monitoreo \\ \hline
Crecimiento OPEX & 3\% anual & Inflación y mantenimiento \\ \hline
Degradación módulos & $-0.5$\% anual & Pérdida de performance \\ \hline
Plan de abandono & \$\,1{,}000{,}000 & Desmantelamiento y restauración \\ \hline
\end{tabular}
\end{table}

\subsubsection*{Flujo de caja simplificado (años 0-15)}
\begin{table}[H]
\centering
\small
\begin{tabular}{|c|l|r|r|r|r|}
\hline
\textbf{Año} & \textbf{Concepto} & \textbf{Entradas} & \textbf{Salidas} & \textbf{Flujo} & \textbf{Acumulado}\\ \hline
0  & Inversión inicial y estudio & 0 & \$\,20{,}000{,}000 & -\$\,20{,}000{,}000 & -\$\,20{,}000{,}000\\
1  & EIA e inicio operación      & \$\,3{,}000{,}000 & \$\,700{,}000 & \$\,2{,}300{,}000 & -\$\,17{,}700{,}000\\
2  & Operación                   & \$\,3{,}060{,}000 & \$\,515{,}000 & \$\,2{,}545{,}000 & -\$\,15{,}155{,}000\\
3  & Operación                   & \$\,3{,}121{,}200 & \$\,530{,}450 & \$\,2{,}590{,}750 & -\$\,12{,}564{,}250\\
4  & Operación                   & \$\,3{,}183{,}624 & \$\,546{,}364 & \$\,2{,}637{,}261 & -\$\,9{,}926{,}990\\
5  & Operación                   & \$\,3{,}247{,}296 & \$\,562{,}754 & \$\,2{,}684{,}542 & -\$\,7{,}242{,}447\\
6  & Operación                   & \$\,3{,}312{,}242 & \$\,579{,}637 & \$\,2{,}732{,}605 & -\$\,4{,}509{,}842\\
7  & Operación                   & \$\,3{,}378{,}487 & \$\,597{,}026 & \$\,2{,}781{,}461 & -\$\,1{,}728{,}381\\
8  & Operación                   & \$\,3{,}446{,}057 & \$\,614{,}937 & \$\,2{,}831{,}120 & \$\,1{,}102{,}739\\
9  & Operación                   & \$\,3{,}514{,}978 & \$\,633{,}385 & \$\,2{,}881{,}593 & \$\,3{,}984{,}332\\
10 & Operación                   & \$\,3{,}585{,}278 & \$\,652{,}387 & \$\,2{,}932{,}891 & \$\,6{,}917{,}223\\
11 & Operación                   & \$\,3{,}656{,}983 & \$\,671{,}958 & \$\,2{,}985{,}025 & \$\,9{,}902{,}248\\
12 & Operación                   & \$\,3{,}730{,}123 & \$\,692{,}117 & \$\,3{,}038{,}006 & \$\,12{,}940{,}254\\
13 & Operación                   & \$\,3{,}804{,}725 & \$\,712{,}880 & \$\,3{,}091{,}845 & \$\,16{,}032{,}099\\
14 & Operación                   & \$\,3{,}880{,}820 & \$\,734{,}267 & \$\,3{,}146{,}553 & \$\,19{,}178{,}652\\
15 & Operación + Abandono        & \$\,3{,}958{,}436 & \$\,1{,}756{,}295 & \$\,2{,}202{,}141 & \$\,21{,}380{,}794\\
\hline
\end{tabular}
\end{table}

\textit{Nota:} La inversión de impacto ambiental sobre la inversión total es del orden del 5-10\%.
