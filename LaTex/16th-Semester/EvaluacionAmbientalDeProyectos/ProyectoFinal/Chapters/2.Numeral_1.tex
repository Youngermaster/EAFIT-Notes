\section{FASE 1: Preparación -- Evaluación Ambiental}
\subsection{Identificación de impactos (Tabla A)}
\begin{table}[h]
\centering
\caption{Tabla A. Identificación de impactos ambientales}
\begin{tabular}{|l|p{10cm}|c|}
\hline
\textbf{Componente} & \textbf{Descripción de impactos ambientales} & \textbf{Calif.}\\ \hline
\multirow{3}{*}{Biótico} 
& Reducción puntual de cobertura vegetal por implantación de estructuras. & $-$\\
& Desplazamiento temporal de fauna durante construcción; rescate y reubicación. & $-$\\
& Microhábitats de sombra que favorecen especies pequeñas en operación. & $+$\\ \hline
\multirow{3}{*}{Atmosférico} 
& Reducción de emisiones de CO$_2$ por sustitución de generación fósil. & $+$\\
& Disminución de gases de combustión locales en operación. & $+$\\
& Polvo y ruido temporales por obra y tránsito de maquinaria. & $-$\\ \hline
\multirow{3}{*}{Hidrosférico} 
& Bajo consumo de agua comparado con otras tecnologías. & $+$\\
& Riesgo de arrastre de sedimentos por movimientos de tierra (obra). & $-$\\
& Mejora indirecta de calidad del agua por menor huella global de emisiones. & $+$\\ \hline
\multirow{3}{*}{Socioeconómico} 
& Generación de empleo directo e inclusión de mano de obra local. & $+$\\
& Programas con instituciones educativas (sensibilización/energía limpia). & $+$\\
& Tránsito y ruido temporal en comunidades aledañas durante obra. & $-$\\ \hline
\multirow{3}{*}{Cultural} 
& Sin afectación prevista a vestigios; protocolo de hallazgo fortuito. & $+$\\
& Promoción de cultura ambiental y tecnológica. & $+$\\
& Posible alteración menor por presencia de maquinaria. & $-$\\ \hline
\multirow{3}{*}{Paisaje} 
& Estructuras visibles que modifican la escena rural. & $-$\\
& Percepción positiva asociada a energías limpias. & $+$\\
& Integración con suelo productivo al no generar emisiones/olores. & $+$\\ \hline
\multirow{3}{*}{Geosférico} 
& Movimiento de tierras y compactación en frentes de obra. & $-$\\
& Protección frente a erosión por cobertura parcial y control de escorrentías. & $+$\\
& Alteración local por fundaciones y redes soterradas. & $-$\\ \hline
\end{tabular}
\end{table}

\subsubsection*{Sustento normativo}
Decreto 2041/2014 (licencias), Decreto 1076/2015 (gestión de recursos; programas de manejo), Resolución 631/2015 (vertimientos), Resolución 627/2006 (ruido), Decreto 1791/1996 (forestal), Decreto 1608/1978 (fauna), Ley 1185/2008 (patrimonio).
