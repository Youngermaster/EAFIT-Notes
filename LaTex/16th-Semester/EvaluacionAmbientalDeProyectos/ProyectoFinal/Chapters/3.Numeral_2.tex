\subsection{Uso de recursos naturales}

\subsubsection{Permisos, concesiones y autorizaciones (Tabla B)}

\begin{table}[H]
\centering
\caption{Tabla B. Uso de recursos naturales}
\begin{tabular}{|p{2.5cm}|p{5cm}|p{8cm}|}
\hline
\textbf{Componente ambiental} & \textbf{Permiso, concesión o autorización} & \textbf{Restricción}\\
\hline
Hidrosférico (extracción) &
Permiso de uso de aguas superficiales o subterráneas otorgado por la autoridad ambiental (\textbf{Corporinoquia \cite{Corporinoquia_permisos}}, en el Meta). &
Solo se autoriza si el uso del agua no afecta el abastecimiento local ni los ecosistemas hídricos; se deben respetar los caudales ecológicos.\\
\hline
Hidrosférico (disposición de agua residual) &
Permiso de vertimientos si se generan aguas residuales durante la fase de construcción (lavado de maquinaria o paneles). &
Las aguas residuales deben cumplir con los límites de calidad establecidos en la \textbf{Resolución 0631 de 2015} antes de ser vertidas.\\
\hline
Hidrosférico (retiros a fuentes de agua) &
Autorización para captación o desvío temporal de cauces (en caso de requerirse para obras). &
No se permite alterar el cauce natural ni afectar la fauna acuática; debe garantizarse la restitución del terreno al finalizar las obras.\\
\hline
Atmosférico &
Permiso de emisiones atmosféricas para maquinaria o generadores usados en la construcción. &
Se deben cumplir los estándares de emisión y ruido del \textbf{Decreto 1076 de 2015}; las emisiones deben ser temporales y controladas.\\
\hline
Biótico (Fauna) &
Plan de manejo de fauna silvestre aprobado por la autoridad ambiental. &
No se puede cazar, capturar ni desplazar fauna sin autorización; se debe realizar rescate y reubicación de especies antes de intervenir el terreno.\\
\hline
Biótico (Flora: árboles y arbustos) &
Permiso de aprovechamiento forestal para retirar vegetación o árboles en el área del proyecto. &
Se debe compensar con reforestación equivalente o superior a lo talado, cumpliendo la normatividad ambiental vigente.\\
\hline
Vestigios prehispánicos &
Autorización del \textbf{ICANH} (Instituto Colombiano de Antropología e Historia) en caso de hallazgos arqueológicos. &
Las obras deben detenerse de inmediato si se encuentran restos arqueológicos hasta obtener la evaluación del ICANH.\\
\hline
Socioeconómico &
Licencia ambiental general del proyecto (gestionada por \textbf{Ecopetrol} ante la autoridad ambiental). &
Se deben cumplir compromisos de empleo local, manejo de residuos, programas sociales y mitigación de impactos comunitarios.\\
\hline
\end{tabular}
\end{table}

\textit{Sustento normativo:} \cite{D1541_1978,D3930_2010,Res631_2015,D1076_2015,D948_1995,D1791_1996,D1608_1978,ICANH_fortuito}.
