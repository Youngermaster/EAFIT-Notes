\section{Título del proyecto}
\textbf{Parque Solar Castilla (FV $\sim$21 MWp) — Planificación Ambiental en Preparación, Operación y Abandono}

\section{Resumen del proyecto}

El \textbf{Parque Solar Castilla} es una instalación fotovoltaica ubicada en el departamento del Meta, concebida para abastecer principalmente el \textit{Complejo Castilla} mediante un esquema de autogeneración con contrato interno (\emph{PPA}) entre el ancla de demanda y el operador. La capacidad instalada es del orden de \textbf{21 MWp}, con producción anual estimada acorde a la irradiancia del piedemonte llanero. El proyecto prioriza tiempos de construcción cortos y una reducción sustantiva de la huella de carbono frente a fuentes fósiles.

\medskip

\textbf{Alcance técnico:} módulos FV mono-PERC/bifaciales, estructuras fijas o seguidor (según optimización de LCOE), inversores \emph{string} o \emph{central}, subestación elevadora y SCADA para operación segura. Obras complementarias: cerramiento, vías internas, drenajes y gestión de escorrentía. En operación se privilegia \textit{uso eficiente del agua} o limpieza en seco; meta operativa de \textbf{cero vertimiento}.

\medskip

\textbf{Gestión ambiental:} el predio se selecciona evitando rondas hídricas, humedales y parches boscosos; se implementan medidas de \emph{control de erosión y sedimentos}, \emph{rescate y reubicación de fauna}, \emph{compensación forestal} cuando aplique, protocolo de \emph{hallazgo arqueológico}, y \emph{gestión integral de residuos} (ordinarios, \textsc{Raee} y \textsc{Respel}). El \textbf{Plan de Manejo Ambiental (PMA)} consolida programas de suelo, agua, aire/ruido, flora, fauna, social y arqueología, con indicadores e \textit{Informes de Cumplimiento Ambiental} (ICA) a la autoridad.

\medskip

\textbf{Competencia y permisos:} el instrumento rector es la \textbf{Licencia Ambiental} (\emph{licencia global}) que integra permisos asociados (aprovechamiento forestal, fauna, ocupación de cauce; concesión/vertimientos si aplica; emisiones sólo si hay fuentes fijas; plan arqueológico ante ICANH; residuos \textsc{Respel}/\textsc{Raee}). La autoridad competente será la \textbf{ANLA} por alcance/typología; si fuese de competencia regional en Meta, la autoridad sería \textbf{CORMACARENA}.

\medskip

\textbf{Cierre y abandono:} contempla retiro de módulos y estructuras, gestión de \textsc{Raee}/\textsc{Respel} con trazabilidad, reconformación y revegetalización de suelos, cumplimiento final de compensaciones y actos de terminación/cancelación de instrumentos (concesiones, vertimientos), hasta el \textit{paz y salvo} ambiental.

\medskip

\textbf{Marco normativo base (síntesis):} \textit{Ley 99 de 1993} (SINA; participación; inversión del 1\%), \textit{Decreto 2041 de 2014} (licenciamiento; incluye abandono), \textit{Decreto 1076 de 2015} (uso de recursos naturales; vertimientos; cauces; flora y fauna), \textit{Decreto 1541 de 1978} (concesiones de aguas), \textit{Decreto 3930 de 2010} y \textit{Resolución 631 de 2015} (vertimientos), \textit{Decreto 1791 de 1996} (forestal), \textit{Decreto 1608 de 1978} (fauna), \textit{Resolución 627 de 2006} (ruido), \textit{Decreto 948 de 1995} (aire), \textit{Ley 1185 de 2008}/ICANH (patrimonio), \textit{Decreto 4741 de 2005} (RESPEL), \textit{Ley 1715 de 2014} (FNCE).
