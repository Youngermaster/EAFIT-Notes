\section{Título del proyecto}
\textbf{Parque Solar Castilla (FV $\sim$21 MWp) -- Planificación Ambiental en Preparación, Operación y Abandono}

\section{Resumen del proyecto}
El Parque Solar Castilla es una instalación fotovoltaica de aproximadamente 21 MWp ubicada en el departamento del Meta, concebida para el abastecimiento eléctrico del complejo industrial Castilla bajo un esquema de autogeneración con contrato PPA interno entre el ancla de demanda y el operador. El proyecto aprovecha la alta irradiancia regional, minimiza la huella de carbono y reduce la exposición a la volatilidad del mercado eléctrico, con tiempos de construcción relativamente cortos frente a otras tecnologías de generación.

Desde el punto de vista ambiental, el proyecto se ubica en un predio con vocación ya intervenida, busca evitar rondas hídricas, humedales y fragmentos de bosque, e incorpora medidas de manejo de suelos (control de erosión y escorrentías), biodiversidad (rescate y reubicación de fauna; compensación forestal si hay tala puntual), patrimonio (protocolo de hallazgo arqueológico) y gestión integral de residuos (ordinarios, peligrosos y RAEE). En operación, el consumo de agua es bajo y se privilegia un esquema de limpieza de paneles con uso eficiente o técnicas en seco, con meta de \textit{cero vertimiento}. El diseño eléctrico contempla inversores, transformadores elevadores y SCADA para operación segura y eficiente.

En términos de viabilidad normativa, el proyecto se modela con \textbf{Licencia Ambiental} como instrumento principal (licencia global) que integra los permisos y autorizaciones necesarios (\emph{aprovechamiento forestal}, \emph{fauna}, \emph{ocupación de cauce}, \emph{concesión de aguas} y/o \emph{vertimientos} si aplican, \emph{aire} únicamente si hubiere fuentes fijas, \emph{ruido} por cumplimiento de límites, \emph{PMA arqueológico} ante el ICANH y gestión de \emph{residuos} incluidos RAEE y RESPEL). La autoridad competente podrá ser la \textbf{ANLA} por alcance/typología; si fuese competencia regional en Meta, sería \textbf{CORMACARENA}.

El \textbf{Plan de Manejo Ambiental (PMA)} estructura programas de suelo, agua, aire/ruido, flora, fauna, social y arqueología, con indicadores e informes periódicos a la autoridad. En abandono se ejecuta el \textbf{Plan de Cierre}: retiro de módulos/estructuras, gestión de RAEE/RESPEL con trazabilidad, restitución de suelos y coberturas, cierre de concesiones/vertimientos, y verificación de éxito de compensaciones hasta el acto de \textit{paz y salvo} ambiental.

\textbf{Bases normativas clave (síntesis):} Ley 99/1993 (SINA, participación, inversión del 1\%), Decreto 2041/2014 (licenciamiento; incluye abandono), Decreto 1076/2015 (recursos naturales, vertimientos, cauces, fauna, flora), Decreto 1541/1978 (concesiones de aguas), Decreto 3930/2010 y Resolución 631/2015 (vertimientos), Decreto 1791/1996 (aprovechamiento forestal), Decreto 1608/1978 (fauna), Resolución 627/2006 (ruido), Decreto 948/1995 (aire), Ley 1185/2008 e ICANH (patrimonio), Decreto 4741/2005 (RESPEL), Ley 1715/2014 (FNCE).
