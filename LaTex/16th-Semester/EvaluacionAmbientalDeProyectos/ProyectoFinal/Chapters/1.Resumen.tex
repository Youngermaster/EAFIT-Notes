\section{Título del proyecto}
\textbf{Parque Solar Castilla (FV $\sim$21\,MWp) — Planificación Ambiental en Preparación, Operación y Abandono}

\section{Resumen del proyecto}

El \textbf{Parque Solar Castilla} es una instalación fotovoltaica ubicada en el municipio de Castilla La Nueva (Meta, Colombia), concebida para el \emph{autoconsumo industrial} del \textit{Complejo Castilla} mediante un contrato bilateral de compra de energía (\emph{PPA}) entre el ancla de demanda (Ecopetrol) y el operador (AES Colombia), con \textbf{capacidad instalada del orden de 21\,MWp}. El proyecto fue puesto en operación en 2019 y se integra a la infraestructura eléctrica del complejo para reducir costos, volatilidad de mercado y huella de carbono, aprovechando la alta irradiancia del piedemonte llanero.\footnote{Antecedente sectorial y comunicado: \cite{Ecopetrol2019}. Marco normativo general: \cite{Ley99_1993,D2041_2014,D1076_2015}.}

\medskip

\textbf{Contexto y objetivos.} Castilla es uno de los complejos de producción más relevantes de Ecopetrol. La estrategia de \emph{electrificación limpia} busca: (i) disminuir la intensidad de emisiones corporativas, (ii) estabilizar el costo de la energía con un activo de generación \emph{behind-the-meter} y (iii) robustecer la seguridad energética del sitio. Bajo esta lógica, el parque solar actúa como respaldo parcial de la demanda interna, con curvas de generación diurnas que calzan con consumos operativos del complejo.

\medskip

\textbf{Alcance técnico (síntesis).} El diseño contempla módulos FV mono-PERC/bifaciales, estructuras a \emph{fijo} o seguidor de un eje (según optimización de LCOE), inversores tipo \emph{string} o \emph{central}, celdas de media tensión, transformadores elevadores y sistema \textsc{scada} para operación y despacho seguro. Obras complementarias: cerramiento, vías internas y \emph{drenajes perimetrales} para control de escorrentía. En operación se privilegia \emph{limpieza en seco} o \emph{uso eficiente del agua} con meta de \textbf{cero vertimiento} siempre que el esquema técnico lo permita.

\medskip

\textbf{Dimensionamiento y desempeño esperado.} Para llanos orientales, un factor de planta de referencia entre 18--22\,\% permite estimar una producción anual en el rango de $33$--$41$\,GWh para $\sim$21\,MWp. Con esa magnitud, y suponiendo reemplazo marginal de generación fósil en la red interna, la reducción agregada de emisiones reportada para el proyecto es del orden de \textbf{$\sim$154{,}000\,t de CO$_2$ en 15 años} (promedio $\sim$10{,}000\,t\,CO$_2$/año), consistente con metas corporativas de descarbonización.\cite{Ecopetrol2019}

\medskip

\textbf{Integración y esquema comercial.} El parque opera \emph{detrás del medidor} conectado a barras internas del complejo, bajo un \textbf{PPA intragrupo} de largo plazo. Este esquema mitiga exposición a precios de bolsa/contratos del \emph{Sistema Interconectado Nacional}, mejora la previsibilidad del flujo de caja y habilita beneficios operacionales (menores pérdidas, control de calidad de energía y mantenimiento planificado).

\medskip

\textbf{Gestión ambiental por ciclo de vida.} Desde la \emph{preparación}, el predio se selecciona evitando rondas hídricas, humedales y coberturas sensibles; el \textbf{PMA} integra programas de suelo (control de erosión/sedimentos), agua (uso eficiente; sólo captación/vertimiento si es técnicamente indispensable), aire/ruido (control de polvo; cumplimiento Res.~627/2006), flora (aprovechamiento y compensación forestal) y fauna (rescate/reubicación), además de \emph{hallazgo arqueológico} (ICANH) y \emph{residuos} (\textsc{pgirs}, \textsc{respel}, \textsc{raee}). En \emph{operación}, se reportan \textbf{ICA} con indicadores de desempeño; en \emph{abandono} se desmantelan módulos/estructuras, se gestionan \textsc{raee}/\textsc{respel} con trazabilidad y se ejecuta restauración morfológica y revegetalización hasta el \emph{paz y salvo} ambiental.\cite{D2041_2014,D1076_2015,D3930_2010,Res631_2015,D1791_1996,D1608_1978,Res627_2006,D948_1995,ICANH_fortuito}

\medskip

\textbf{Riesgos críticos y mitigación (operativo-ambiental).} Variabilidad interanual de irradiancia (\emph{mitigación:} coberturas financieras y reservas de mantenimiento), degradación de módulos ($\sim$0.5\,\%/año; \emph{mitigación:} \emph{O\&M} preventivo y reposición planificada), \emph{soiling} estacional (\emph{mitigación:} limpieza racional y control de polvo), eventos extremos de lluvia (\emph{mitigación:} drenajes, disipadores, \emph{franjas de manejo}), y cumplimiento de compensaciones (\emph{mitigación:} contratos de mantenimiento de siembras y monitoreo con umbrales de supervivencia $\geq 80\%$).

\medskip

\textbf{Valor público y encadenamientos.} Además de la reducción de GEI, el proyecto promueve empleo local, compras en territorio y \emph{programas de educación ambiental y seguridad eléctrica}, con mecanismos de \textsc{pqrs} y socialización permanente, en línea con la \textit{Ley 99 de 1993} y la participación informada en licenciamiento.\cite{Ley99_1993,D2041_2014}

\medskip

\textbf{Marco normativo base (síntesis).} \textit{Ley 99 de 1993} (SINA, licenciamiento e inversión del 1\,\%) \cite{Ley99_1993}; \textit{Decreto 2041 de 2014} (trámite de licencias, seguimiento y cierre) \cite{D2041_2014}; \textit{Decreto 1076 de 2015} (uso de recursos; ocupación de cauce; vertimientos; fauna/flora) \cite{D1076_2015}; \textit{Decreto 1541 de 1978} (concesiones de agua) \cite{D1541_1978}; \textit{Decreto 3930 de 2010} y \textit{Resolución 631 de 2015} (parámetros y límites de vertimientos) \cite{D3930_2010,Res631_2015}; \textit{Decreto 1791 de 1996} (aprovechamiento forestal) \cite{D1791_1996}; \textit{Decreto 1608 de 1978} (fauna) \cite{D1608_1978}; \textit{Resolución 627 de 2006} (ruido) \cite{Res627_2006}; \textit{Decreto 948 de 1995} (aire) \cite{D948_1995}; \textit{Ley 1185 de 2008}/ICANH (patrimonio arqueológico y hallazgo fortuito) \cite{ICANH_fortuito}; \textit{Decreto 4741 de 2005} (residuos peligrosos) y \textit{Ley 1715 de 2014} (promoción de FNCE y autogeneración) \cite{Ley1715_2014}. Antecedente sectorial del proyecto: \cite{Ecopetrol2019}.

\medskip

\textbf{Límites y supuestos.} Cuando se presentan cifras de desempeño anual (GWh/año) o factores de planta, se ofrecen como \emph{referencias técnicas} coherentes con la radiación local y bibliografía del sector; las magnitudes exactas dependen del \emph{as-built}, disponibilidad, pérdidas del \emph{BOS}, \emph{curtailment} y perfil de demanda interna.

\bigskip

\noindent\textbf{Palabras clave:} autogeneración, PPA, FNCE, licenciamiento ambiental, PMA, \textsc{raee}, \textsc{respel}, control de erosión y sedimentos, drenajes, revegetalización, ICANH, \textsc{pqrs}.
