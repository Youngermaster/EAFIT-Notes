\section{FASE 3: Operación del proyecto}
En caso de que se obtenga la licencia ambiental, durante la fase de operación del \textbf{Parque Solar Castilla} el titular deberá cumplir todas las obligaciones que queden consignadas en la resolución de licencia ambiental y en los permisos, concesiones o autorizaciones asociadas. A continuación se listan las obligaciones típicas para este tipo de proyecto en Colombia.

\subsection{1) Generales de la licencia}
\begin{itemize}
  \item Ejecutar el \textbf{Plan de Manejo Ambiental (PMA)} aprobado, incluyendo programas, metas, indicadores y presupuesto, tanto para construcción como para operación y abandono.
  \item Actualizar el PMA cuando haya cambios en el diseño, en la huella intervenida o en el cronograma, y solicitar modificación de la licencia cuando corresponda.
  \item Constituir y mantener vigentes las garantías o pólizas de cumplimiento ambiental exigidas en la resolución.
  \item Llevar archivo y trazabilidad de permisos, actas de visita, monitoreos, reportes e informes de cumplimiento ambiental (ICA).
\end{itemize}
\textbf{Base legal \cite{D2041_2014,Ley99_1993}:} Decreto 2041 de 2014 (procedimiento y obligaciones de la licencia); Ley 99 de 1993.

\subsection{2) Suelo y geosfera}
\begin{itemize}
  \item Delimitar y señalizar áreas de obra, vías internas y zonas de acopio; prohibir disposición de escombros o RCD por fuera de las áreas autorizadas.
  \item Aplicar medidas de control de erosión y sedimentación (cunetas, filtros, revegetalización inmediata de taludes).
  \item Restaurar o descompactar suelos afectados al cierre de frentes de obra o en el desmantelamiento.
\end{itemize}
\textbf{Base legal \cite{D1076_2015}:} Decreto 1076 de 2015 (gestión de suelos y protección); PMA aprobado.

\subsection{3) Agua (captación, vertimientos y escorrentía)}
\begin{itemize}
  \item Si se capta agua para limpieza de paneles u operación: contar con \textbf{concesión de aguas}, instalar macromedidor y respetar el caudal ecológico; implementar uso eficiente y, de ser posible, recirculación.
  \item Si se generan descargas: tramitar \textbf{permiso de vertimientos}, implementar tratamiento y cumplir con la Resolución 631 de 2015 (DBO, DQO, SST, aceites y grasas, pH, caudal).
  \item Manejar la escorrentía superficial con zanjas perimetrales, disipadores y estructuras de retención para evitar que llegue turbia a los drenajes naturales.
  \item Presentar los informes de monitoreo con la periodicidad que fije la autoridad (trimestral o semestral).
\end{itemize}
\textbf{Base legal \cite{D1541_1978,D3930_2010,Res631_2015,D1076_2015}:} Decreto 1541 de 1978 (concesiones), Decreto 3930 de 2010 y Resolución 631 de 2015 (vertimientos), Decreto 1076 de 2015.

\subsection{4) Aire y ruido}
\begin{itemize}
  \item Si existen fuentes fijas (grupo electrógeno permanente, compresores): contar con permiso de emisiones y darle mantenimiento preventivo.
  \item Controlar el polvo en vías internas (humectación, velocidad máxima, cubrimiento de acopios).
  \item Cumplir los niveles de ruido de la Resolución 627 de 2006 (diurno/nocturno) y ajustar horarios si hay receptores sensibles cercanos.
  \item Realizar campañas de ruido con sonómetro clase 1 cuando lo pida la autoridad.
\end{itemize}
\textbf{Base legal \cite{D948_1995,Res627_2006,D1076_2015}:} Decreto 948 de 1995 (aire), Resolución 627 de 2006 (ruido), Decreto 1076 de 2015.

\subsection{5) Flora y compensación forestal}
\begin{itemize}
  \item Realizar cualquier tala o retiro de árboles solamente con \textbf{permiso de aprovechamiento forestal} e inventario previo.
  \item Ejecutar la \textbf{compensación forestal} en el sitio, especie y factor que señale el acto administrativo, con mantenimiento mínimo hasta lograr el porcentaje de supervivencia (usualmente $\geq 80\%$).
  \item Reportar a la autoridad los avances de la compensación con evidencias fotográficas y de georreferenciación.
\end{itemize}
\textbf{Base legal \cite{D1791_1996,D1076_2015}:} Decreto 1791 de 1996; Decreto 1076 de 2015.

\subsection{6) Fauna silvestre}
\begin{itemize}
  \item Implementar el \textbf{Plan de Manejo de Fauna} aprobado: pre-rescate, ahuyentamiento, captura y reubicación en sitios autorizados.
  \item Prohibir la caza, tenencia o comercialización de fauna por parte de contratistas y trabajadores; realizar capacitaciones periódicas.
  \item Mantener registros de rescate y de mortalidad (meta: mortalidad cero).
\end{itemize}
\textbf{Base legal \cite{D1608_1978,D1076_2015}:} Decreto 1608 de 1978 (fauna silvestre); Decreto 1076 de 2015.

\subsection{7) Patrimonio arqueológico}
\begin{itemize}
  \item Cumplir el \textbf{Programa de Manejo Arqueológico (PMA-ICANH)} y el Protocolo de Hallazgo Fortuito.
  \item Ante un hallazgo: parar la obra, aislar el área, informar al ICANH y sólo reanudar con su concepto.
  \item Capacitar a las cuadrillas para reconocer material arqueológico.
\end{itemize}
\textbf{Base legal \cite{ICANH_fortuito}:} Ley 1185 de 2008; lineamientos del ICANH.

\subsection{8) Residuos (ordinarios, peligrosos, RAEE)}
\begin{itemize}
  \item Ejecutar el \textbf{PGIRS} del proyecto: separación en la fuente, almacenamiento temporal en área impermeabilizada y entrega a gestor autorizado.
  \item Para \textbf{RESPEL} (aceites usados, trapos contaminados, baterías): registro como generador y disposición con gestor licenciado, con manifiestos.
  \item Para \textbf{RAEE} (inversores, equipos, cables al final de su vida): devolución al productor o entrega a gestor autorizado.
  \item Presentar reportes semestrales o anuales de generación y disposición.
\end{itemize}
\textbf{Base legal \cite{D1076_2015}:} Decreto 1076 de 2015; Decreto 4741 de 2005 (residuos peligrosos); normativa RAEE vigente; Ley 1259 de 2008 (comparendo ambiental, si aplica municipio).

\subsection{9) Componente social y relacionamiento}
\begin{itemize}
  \item Ejecutar el \textbf{plan de participación y relacionamiento} aprobado: socialización, mecanismo de PQRS, contratación de mano de obra local cuando sea posible, compras locales y programas de educación ambiental.
  \item Implementar plan de manejo de tránsito y seguridad vial durante movimientos de maquinaria o reposición de paneles.
  \item Mantener registros de reuniones con la comunidad y de atención a quejas.
\end{itemize}
\textbf{Base legal \cite{Ley99_1993,D2041_2014}:} Ley 99 de 1993 (participación, art. 69); Decreto 2041 de 2014 (participación en licenciamiento).

\subsection{10) Riesgo y contingencias}
\begin{itemize}
  \item Mantener vigente un \textbf{plan de contingencias} para incendio eléctrico, incendio de vegetación, derrames de combustibles y eventos climáticos fuertes.
  \item Realizar simulacros periódicos y coordinar con bomberos y autoridades municipales.
  \item Cumplir con el \textbf{SG-SST} durante la operación.
\end{itemize}
\textbf{Base legal \cite{D1072_2015?,Ley99_1993}:} Ley 1523 de 2012 (gestión del riesgo); Decreto 1072 de 2015 (SG-SST).

\subsection{11) Monitoreo, reporte y auditoría}
\begin{itemize}
  \item Presentar a la autoridad los \textbf{Informes de Cumplimiento Ambiental (ICA)} con la periodicidad establecida (trimestral o semestral).
  \item Incluir en los informes los resultados de monitoreo de agua (si hay), ruido, polvo, residuos, flora y fauna, componente social y cumplimiento de compensaciones.
  \item Implementar auditorías internas o externas y planes de mejora cuando se detecten no conformidades.
\end{itemize}
\textbf{Base legal:} Decreto 2041 de 2014 (seguimiento); Decreto 1076 de 2015.

\subsection{12) Ordenamiento territorial y servidumbres}
\begin{itemize}
  \item Mantener la \textbf{compatibilidad con el POT} o EOT del municipio (uso de suelo rural con actividad energética/autogeneración).
  \item Gestionar y formalizar las \textbf{servidumbres} que sean necesarias para accesos o líneas internas antes de intervenir los predios.
\end{itemize}
\textbf{Base legal \cite{Ley99_1993}:} Ley 388 de 1997 (ordenamiento territorial); Código Civil (servidumbres).

\subsection{13) Inversión ambiental del 1\%}
\begin{itemize}
  \item Si el proyecto resulta ser \textbf{usuario de agua de fuente natural} mediante concesión y construye obras de captación, deberá invertir no menos del 1\% del costo del proyecto en la cuenca abastecedora, conforme lo apruebe la autoridad ambiental.
\end{itemize}
\textbf{Base legal \cite{Ley99_1993}:} Ley 99 de 1993, art. 43.

\subsection{14) Abandono y cierre}
\begin{itemize}
  \item Presentar y ejecutar el \textbf{Plan de Abandono} antes del cese definitivo: desmantelar módulos y estructuras, gestionar RAEE y residuos peligrosos, restaurar el terreno y hacer monitoreo post-cierre.
  \item Obtener el acto de terminación o paz y salvo ambiental de la autoridad.
\end{itemize}
\textbf{Base legal:} Decreto 2041 de 2014 (la licencia incluye abandono); Decreto 1076 de 2015.
