\section{FASE 4: Abandono del proyecto}
En Colombia el abandono no es automático: la licencia ambiental incluye la fase de cierre y el titular sigue obligado hasta que la autoridad expida el acto de terminación o paz y salvo ambiental.

\subsection{1) Plan de abandono y desmontaje total}
\begin{itemize}
  \item Ejecutar el \textbf{Plan de Abandono} aprobado: retiro de paneles/módulos, estructuras de soporte, inversores, cableado, cerramientos, cuartos eléctricos y obras temporales.
  \item Retirar o tratar las fundaciones según lo defina la autoridad (retirar, seccionar o cubrir si no representan riesgo).
  \item Entregar a la autoridad un \textbf{informe de cierre} con memoria fotográfica, planos “as-left” y relación de equipos efectivamente desmontados.
\end{itemize}
\textbf{Por qué + norma \cite{D2041_2014,D1076_2015}:} la licencia ambiental comprende construcción, operación y abandono; el cierre es una obligación expresa. Decreto 2041 de 2014 (licencias, seguimiento y cierre); Decreto 1076 de 2015.

\subsection{2) Gestión de residuos del cierre (RAEE, RESPEL y ordinarios)}
\begin{itemize}
  \item Gestionar los \textbf{RAEE} del proyecto (paneles, inversores, tableros, cableado) con productor o gestor autorizado, con certificados de valorización o reciclaje y trazabilidad completa.
  \item Gestionar los \textbf{RESPEL} que queden al final (aceites, trapos contaminados, baterías, solventes) manteniendo activo el registro como generador hasta agotar inventarios y presentando el informe final con manifiestos.
  \item Disponer los residuos ordinarios y los RCD en sitios autorizados por la autoridad ambiental o el municipio.
\end{itemize}
\textbf{Por qué + norma \cite{D1076_2015}:} el titular responde hasta la disposición final acreditada. Decreto 4741 de 2005 (residuos peligrosos); lineamientos RAEE; Decreto 1076 de 2015 (PGIRS).

\subsection{3) Suelo y restauración morfológica/vegetal}
\begin{itemize}
  \item Descompactar y reconformar los suelos donde hubo tránsito, fundaciones o acopios.
  \item Restituir los microdrenajes o canales que se hubieran modificado durante la operación.
  \item Realizar \textbf{revegetalización} con especies nativas y mantenerlas hasta alcanzar la cobertura y la supervivencia que exija la resolución.
  \item Ejecutar monitoreo post-cierre de erosión y estabilidad y remitir los reportes hasta que la autoridad dé por cumplidas las metas.
\end{itemize}
\textbf{Por qué + norma \cite{D1076_2015,D2041_2014}:} el PMA se mantiene hasta que la autoridad verifique que el área quedó en condiciones aceptables. Decreto 1076 de 2015; Decreto 2041 de 2014.

\subsection{4) Agua (captación, vertimientos y escorrentía)}
\begin{itemize}
  \item Si el proyecto tenía \textbf{concesión de aguas}, debe solicitar el \textbf{acto de terminación} de la concesión, presentar las lecturas finales y retirar o asegurar los puntos de captación.
  \item Si tenía \textbf{permiso de vertimientos}, debe hacer un muestreo final de cumplimiento, desmontar las estructuras asociadas y declarar \textbf{cero descarga} a partir del cese.
  \item Mantener el control de escorrentías en el predio mientras duren las obras de restauración.
\end{itemize}
\textbf{Por qué + norma \cite{D1541_1978,D3930_2010,Res631_2015,D1076_2015}:} el uso del agua y el vertimiento sólo terminan con acto expreso de la autoridad. Decreto 1541 de 1978 (concesiones); Decreto 3930 de 2010 y Resolución 631 de 2015 (vertimientos); Decreto 1076 de 2015.

\subsection{5) Flora y fauna — obligaciones remanentes}
\begin{itemize}
  \item Cumplir la \textbf{compensación forestal} y su mantenimiento hasta lograr el porcentaje de supervivencia exigido (usualmente $\geq 80\%$) y obtener el certificado de cumplimiento.
  \item Cerrar el \textbf{Plan de Manejo de Fauna}: última evaluación de hábitat, consolidación de registros de rescate/avistamientos y entrega del informe final a la autoridad.
\end{itemize}
\textbf{Por qué + norma \cite{D1791_1996,D1608_1978,D1076_2015}:} las compensaciones y planes de fauna no se extinguen con el cierre operativo; se cierran por verificación de metas. Decreto 1791 de 1996 (forestal); Decreto 1608 de 1978 (fauna); Decreto 1076 de 2015.

\subsection{6) Patrimonio arqueológico}
\begin{itemize}
  \item Mantener vigente el \textbf{protocolo de hallazgo fortuito} durante todo el desmontaje.
  \item Si aparece material arqueológico, aplicar el PMA arqueológico y remitir al ICANH el informe de cierre.
\end{itemize}
\textbf{Por qué + norma \cite{ICANH_fortuito}:} el patrimonio arqueológico se protege en todas las fases del proyecto. Ley 1185 de 2008; lineamientos del ICANH.

\subsection{7) Componente social y relacionamiento}
\begin{itemize}
  \item Mantener el mecanismo de \textbf{quejas y reclamos} y la divulgación del cronograma de cierre hasta terminar las obras.
  \item Realizar una \textbf{socialización de cierre} con la comunidad y con las autoridades locales, dejando acta de los compromisos.
\end{itemize}
\textbf{Por qué + norma \cite{Ley99_1993,D2041_2014}:} la participación y el acceso a la información son obligaciones hasta el archivo del expediente. Ley 99 de 1993; Decreto 2041 de 2014.

\subsection{8) Riesgo y contingencias en el cierre}
\begin{itemize}
  \item Mantener \textbf{activo} el plan de contingencias durante el desmontaje (incendio eléctrico, manejo de combustibles, clima, seguridad de personal).
  \item Realizar simulacros si la autoridad lo exige y conservar los EPP y el SG-SST hasta el último frente de trabajo.
\end{itemize}
\textbf{Por qué + norma:} hay obligación de prevención y respuesta mientras existan frentes de obra. Ley 1523 de 2012 (gestión del riesgo); Decreto 1072 de 2015 (SG-SST).

\subsection{9) Garantías financieras y pólizas}
\begin{itemize}
  \item Mantener vigentes las \textbf{pólizas o garantías} de cumplimiento ambiental hasta que la autoridad expida el acto de cierre.
  \item Ampliar la vigencia si el cierre tarda más de lo previsto.
\end{itemize}
\textbf{Por qué + norma \cite{D2041_2014}:} la autoridad puede hacer efectivas las garantías por incumplimientos en el abandono. Decreto 2041 de 2014.

\subsection{10) Ordenamiento territorial y servidumbres}
\begin{itemize}
  \item Tramitar la \textbf{cancelación o actualización} de las servidumbres técnicas (vías, líneas, accesos) que ya no se requieran.
  \item Restituir las vías o accesos temporales que se hubieran abierto sólo para el proyecto.
\end{itemize}
\textbf{Por qué + norma:} las afectaciones al uso del suelo deben regularizarse al terminar el proyecto. Ley 388 de 1997 (ordenamiento); Código Civil (servidumbres).

\subsection{11) Inversión del 1\% (si aplicó)}
\begin{itemize}
  \item Si el proyecto tuvo obligación de \textbf{inversión del 1\%} por uso de agua, debe cerrar y certificar esa inversión con actas de ejecución y entrega a la autoridad y al administrador de la cuenca.
\end{itemize}
\textbf{Por qué + norma \cite{Ley99_1993}:} la obligación subsiste hasta su ejecución y certificación. Ley 99 de 1993, art. 43.

\subsection{12) Informes de cumplimiento y acto final}
\begin{itemize}
  \item Presentar el \textbf{Informe Final de Cumplimiento Ambiental (IFCA)} integrando todos los componentes: suelo, agua, aire, flora, fauna, arqueología, social y residuos.
  \item Solicitar la \textbf{visita de verificación} a la autoridad ambiental.
  \item La autoridad expedirá el acto administrativo de terminación o \textit{paz y salvo} ambiental; hasta ese momento el titular sigue obligado a reportar y mantener los controles.
\end{itemize}
\textbf{Por qué + norma \cite{D2041_2014}:} el cierre es un acto expreso; no aplica silencio administrativo positivo. Decreto 2041 de 2014; Ley 1437 de 2011, art. 84.

\subsubsection*{Entregables típicos del cierre}
\begin{itemize}
  \item Plan de Abandono ejecutado e Informe Final de Cierre.
  \item Manifiestos y certificados de disposición/valorización de RAEE, RESPEL y RCD.
  \item Actas de cumplimiento de compensación forestal y de socialización de cierre.
  \item Actos de terminación de concesión y de permiso de vertimientos (si existían).
  \item Paz y salvo ambiental expedido por la autoridad.
\end{itemize}
