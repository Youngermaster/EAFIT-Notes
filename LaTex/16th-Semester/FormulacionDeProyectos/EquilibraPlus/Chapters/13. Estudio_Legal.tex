% ==========================================
% Chapters/13. Estudio_Legal.tex
% ==========================================
\section{Estudio Legal}

\subsection{Tipo de organización y constitución legal}

Para el proyecto \emph{Equilibrado} se propone la creación de una \textbf{Sociedad por Acciones Simplificada (S.A.S.)}, de acuerdo con la Ley 1258 de 2008. Este tipo de sociedad es el más utilizado por emprendimientos de base tecnológica en Colombia, ya que combina flexibilidad con protección patrimonial.

\subsubsection*{Tipo de sociedad recomendada: S.A.S.}

\begin{itemize}
    \item \textbf{Flexibilidad:} se adapta muy bien a emprendimientos innovadores y de base tecnológica como \emph{Equilibrado}, permitiendo pactar estatutos a la medida.
    \item \textbf{Número de socios:} permite constituirse con uno o varios socios, lo que facilita iniciar con un socio fundador y sumar aliados más adelante.
    \item \textbf{Responsabilidad limitada:} los socios responden únicamente hasta el monto de sus aportes, protegiendo su patrimonio personal.
    \item \textbf{Facilidad de gestión:} no requiere junta directiva de manera obligatoria, lo que simplifica la toma de decisiones y reduce trámites internos.
    \item \textbf{Escalabilidad:} facilita la entrada de inversionistas a futuro mediante la emisión de nuevas acciones o la cesión de las existentes.
\end{itemize}

\subsubsection*{Requisitos básicos de constitución}

Para constituir formalmente la empresa se deben cumplir, como mínimo, los siguientes pasos:

\begin{itemize}
    \item Registro de la sociedad ante la \textbf{Cámara de Comercio de Medellín}.
    \item Elaboración y firma del \textbf{documento de constitución} (acta o minuta) y de los \textbf{estatutos sociales}.
    \item Designación de un \textbf{representante legal} y, si se requiere, de suplentes.
    \item Definición de un \textbf{objeto social} que incluya el desarrollo de software, servicios tecnológicos y dispositivos aplicados a la salud.
    \item Aporte inicial de capital, en dinero o en especie (por ejemplo, desarrollo tecnológico ya realizado).
\end{itemize}

En resumen, la figura de S.A.S. ofrece una estructura jurídica sencilla pero robusta, alineada con la naturaleza escalable y tecnológica del proyecto.

\subsection{Requerimientos legales}

A nivel operativo, \emph{Equilibrado} debe cumplir con varios requisitos legales y registros ante entidades colombianas. A continuación se resumen los más relevantes.

\subsubsection*{Cámara de Comercio}

\begin{itemize}
    \item Registro mercantil de la S.A.S. como persona jurídica.
    \item Diligenciamiento de formularios, acta de constitución, estatutos y pago de derechos.
    \item Verificación de homonimia para el nombre de la sociedad. Actualmente las marcas \emph{Equilibrado}, \emph{Equilibra+} y \emph{Equilibrio+} no aparecen registradas, por lo que no se prevén conflictos de nombre en esta fase.
\end{itemize}

\subsubsection*{DIAN}

\begin{itemize}
    \item Inscripción en el \textbf{Registro Único Tributario (RUT)}.
    \item Obtención del \textbf{Número de Identificación Tributaria (NIT)}.
    \item Habilitación y uso de \textbf{facturación electrónica}.
    \item Definición del régimen tributario (simple o régimen ordinario), de acuerdo con los ingresos y proyecciones del proyecto.
\end{itemize}

\subsubsection*{Propiedad industrial y derechos de autor}

\begin{itemize}
    \item Registro de la \textbf{marca} (nombre, logotipo) ante la \textbf{Superintendencia de Industria y Comercio (SIC)}.
    \item Evaluar una posible \textbf{patente de modelo de utilidad} para el dispositivo o el conjunto de hardware, si presenta una innovación técnica relevante.
    \item Registro del \textbf{software} como obra protegida ante la \textbf{Dirección Nacional de Derechos de Autor (DNDA)}.
\end{itemize}

\subsubsection*{Aspectos sanitarios (INVIMA, si aplica)}

Si el dispositivo es clasificado como dispositivo médico (por contacto con el cuerpo o uso clínico), se debe:

\begin{itemize}
    \item Realizar una \textbf{consulta previa} con un ingeniero biomédico y revisar la normativa para determinar si aplica como dispositivo de riesgo \textit{clase I o II}.
    \item Tramitar el \textbf{registro sanitario} correspondiente ante el INVIMA, cumpliendo los requisitos técnicos, de etiquetado y documentación.
\end{itemize}

\subsubsection*{Normas locales}

\begin{itemize}
    \item Cumplimiento del \textbf{Plan de Ordenamiento Territorial (POT)} de Medellín para el uso del suelo de la sede administrativa o centro de operaciones.
    \item Trámite de licencias o permisos municipales en caso de tener atención al público en un espacio físico.
\end{itemize}

\subsubsection*{Costos asociados al estudio legal}

A continuación se resumen los costos estimados de los principales trámites de constitución y operación legal. Los valores se expresan en pesos colombianos (COP) y pueden actualizarse consultando el archivo de Excel base.

\begin{table}[H]
\centering
\small
\renewcommand{\arraystretch}{1.2}
\begin{tabular}{p{6cm} c r}
\toprule
\multicolumn{3}{c}{\textbf{Fase de constitución}} \\
\midrule
\textbf{Concepto} & \textbf{Cantidad} & \textbf{Costo (COP)} \\
\midrule
Certificado de homonimia & 1 & \$11\,600,00 \\
Escritura pública / minuta / acta de inicio & 1 & \$2\,000\,000,00 \\
Inscripción en Cámara de Comercio & 1 & \$8\,100,00 \\
Apertura de cuenta bancaria (Bancolombia) & 1 & \$36\,000,00 \\
RUT y/o NIT (DIAN) & 1 & \$0,00 \\
Registro mercantil & 1 & \$69\,000,00 \\
Pago impuestos de rentas departamentales & -- & -- \\
\midrule
\textbf{Valor total estimado} & & \textbf{\$2\,124\,700,00} \\
\bottomrule
\end{tabular}
\caption{Costos legales en fase de constitución de \emph{Equilibrado}.}
\label{tab:costos-constitucion-legal}
\end{table}

\begin{table}[H]
\centering
\small
\renewcommand{\arraystretch}{1.2}
\begin{tabular}{p{6cm} c r}
\toprule
\multicolumn{3}{c}{\textbf{Fase de operación}} \\
\midrule
\textbf{Concepto} & \textbf{Cantidad} & \textbf{Costo (COP)} \\
\midrule
Registros de libros contables & 1 & \$23\,100,00 \\
Licencia de bomberos & -- & \$0,00 \\
Avisos y tableros & 1 & \$50\,000,00 \\
Licencia ambiental & -- & \$0,00 \\
Licencia de funcionamiento & -- & \$0,00 \\
Permiso de uso del suelo (planeación municipal) & -- & \$0,00 \\
Registro INVIMA & 1 & \$1\,500\,000,00 \\
Permiso ICA & -- & \$0,00 \\
Sayco y Acinpro & 1 & \$1\,600\,000,00 \\
\midrule
\textbf{Valor total estimado} & & \textbf{\$1\,600\,000,00} \\
\bottomrule
\end{tabular}
\caption{Costos legales en la fase de operación de \emph{Equilibrado}.}
\label{tab:costos-operacion-legal}
\end{table}

Estos montos dan una idea de la inversión mínima necesaria para operar dentro del marco legal. La revisión anual de tarifas y obligaciones es recomendable para mantener la empresa al día.

\subsection{Manejo de contratos}

El proyecto requiere una estructura contractual clara con los distintos actores involucrados, tanto internos como externos.

\subsubsection*{A. Contratos de desarrollo y tecnología}

\begin{itemize}
    \item Contratos de prestación de servicios tecnológicos con desarrolladores de software (app móvil, plataforma web y backend).
    \item Contratos de hosting y servicios en la nube con proveedores como AWS, Azure u otros equivalentes.
    \item Acuerdos de confidencialidad (\textit{Non Disclosure Agreements -- NDA}) con proveedores, programadores y personal que tenga acceso a información sensible o código fuente.
\end{itemize}

\subsubsection*{B. Contratos de compra y suministro}

\begin{itemize}
    \item Contratos de compra de dispositivos electrónicos (sensores, carcasas, placas, baterías) para el wearable.
    \item Contratos con proveedores de ensamblaje y logística, que especifiquen condiciones de calidad, tiempos de entrega y garantías.
    \item Acuerdos de suministro de componentes en caso de producción local, para asegurar continuidad y precios estables.
\end{itemize}

\subsubsection*{C. Contratos de transporte y distribución}

\begin{itemize}
    \item Convenios con empresas de logística para el envío de dispositivos a clientes finales o instituciones de salud.
    \item Definición de condiciones de embalaje, seguros, trazabilidad y tiempos máximos de entrega.
\end{itemize}

\subsubsection*{D. Contratos de asesoría y consultoría}

\begin{itemize}
    \item Contratos con consultores legales, financieros, contables, de marketing y de salud.
    \item Cada contrato debe definir con claridad: objeto del servicio, duración, entregables, honorarios y propiedad de los productos intelectuales generados.
\end{itemize}

\subsubsection*{E. Contratos laborales}

\begin{itemize}
    \item Contratos de trabajo (a término fijo o indefinido) para personal administrativo, atención al cliente y soporte técnico.
    \item Cumplimiento de las obligaciones laborales: afiliación a EPS, ARL, fondos de pensión, caja de compensación y pago completo de prestaciones sociales.
\end{itemize}

Una adecuada gestión contractual reduce conflictos futuros y respalda la operación formal del proyecto ante autoridades y aliados.

\subsection{Normatividad legal aplicable}

A continuación se resume la normativa clave que impacta a \emph{Equilibrado} desde diferentes frentes: comercial, tributario, sanitario, laboral, de propiedad intelectual y de protección de datos.

\subsubsection*{A. Normatividad comercial y societaria}

\begin{itemize}
    \item \textbf{Ley 1258 de 2008:} crea y regula la Sociedad por Acciones Simplificada (S.A.S.).
    \item \textbf{Código de Comercio Colombiano:} contiene disposiciones generales sobre sociedades comerciales, contratos y obligaciones mercantiles.
    \item \textbf{Ley 1480 de 2011 (Estatuto del Consumidor):} regula la protección al consumidor, garantías y publicidad.
\end{itemize}

\subsubsection*{B. Normatividad tributaria y fiscal}

\begin{itemize}
    \item \textbf{Estatuto Tributario:} establece las obligaciones fiscales, tipos de regímenes (simple y ordinario) e impuestos aplicables.
    \item Resoluciones de la DIAN sobre \textbf{facturación electrónica} y reportes de IVA, retención en la fuente y renta.
    \item \textbf{Ley 2010 de 2019} (Ley de Crecimiento Económico): introduce beneficios y reglas específicas para empresas emergentes y contribuyentes del régimen simple.
\end{itemize}

\subsubsection*{C. Normatividad sanitaria}

Aplica en caso de que el wearable sea considerado dispositivo médico:

\begin{itemize}
    \item \textbf{Decreto 4725 de 2005:} regula los dispositivos médicos en Colombia.
    \item \textbf{Decreto 2078 de 2012:} establece la clasificación de dispositivos según su nivel de riesgo (clases I a IV).
    \item Protocolos del \textbf{INVIMA}: requisitos para registro sanitario, etiquetado, empaques y vigilancia poscomercialización.
\end{itemize}

\subsubsection*{D. Normatividad laboral}

\begin{itemize}
    \item \textbf{Código Sustantivo del Trabajo (CST):} regula los contratos laborales, derechos y deberes de empleadores y trabajadores.
    \item \textbf{Ley 100 de 1993:} crea el Sistema de Seguridad Social Integral (salud, pensión y riesgos profesionales).
    \item Normas sobre contratación por prestación de servicios para casos puntuales, evitando desnaturalizar la relación laboral.
\end{itemize}

\subsubsection*{E. Propiedad intelectual y protección de software}

\begin{itemize}
    \item \textbf{Ley 23 de 1982:} régimen general de derechos de autor.
    \item \textbf{Decisión Andina 351} y \textbf{Decisión 486}: regulan la propiedad intelectual y el régimen de marcas en la Comunidad Andina.
    \item Registro de la app y del código fuente ante la \textbf{DNDA} como obra de software.
    \item Registro de la marca \emph{Equilibrado} y sus variantes ante la \textbf{SIC}.
\end{itemize}

\subsubsection*{F. Protección de datos personales}

\begin{itemize}
    \item \textbf{Ley 1581 de 2012:} establece el régimen de protección de datos personales en Colombia.
    \item \textbf{Decreto 1377 de 2013:} reglamenta aspectos relacionados con bases de datos y autorización de los titulares.
\end{itemize}

Dado que \emph{Equilibrado} recopila datos de salud y hábitos de adultos mayores, es crítico contar con políticas de tratamiento de datos, avisos de privacidad y esquemas de seguridad robustos.

\subsection{Identificación de posibles riesgos legales}

Finalmente, se identifican los principales riesgos legales que podrían afectar el proyecto si no se gestionan adecuadamente.

\subsubsection*{1. Riesgos por incumplimiento normativo}

\begin{itemize}
    \item Falta de registro INVIMA en caso de que el dispositivo se clasifique como médico.
    \item Incumplimiento de la normativa de protección de datos personales (Ley 1581 de 2012 y Decreto 1377 de 2013).
    \item No registrar la marca o el software, exponiéndose a conflictos por uso no autorizado o plagio.
    \item Omisión de obligaciones tributarias o laborales, con posibles sanciones por parte de la DIAN o del Ministerio de Trabajo.
\end{itemize}

\subsubsection*{2. Riesgos por fallas contractuales}

\begin{itemize}
    \item Contratos incompletos con desarrolladores, proveedores o consultores que no definan con precisión la propiedad intelectual, los entregables y las responsabilidades.
    \item Ausencia de cláusulas de confidencialidad (NDA) y de cesión de derechos de autor en contratos de desarrollo tecnológico.
\end{itemize}

\subsubsection*{3. Riesgos laborales}

\begin{itemize}
    \item Vínculos informales o sin contratos escritos con personal de soporte, atención al cliente o desarrollo.
    \item No afiliación del personal al sistema de seguridad social ni pago de prestaciones legales.
\end{itemize}

\subsubsection*{4. Riesgos en la relación con usuarios y consumidores}

\begin{itemize}
    \item Reclamaciones por publicidad engañosa o por promesas de resultados que el sistema no pueda garantizar.
    \item Demandas por fallas del dispositivo o de la app que puedan estar relacionadas con accidentes o caídas no detectadas.
    \item Falta de términos y condiciones de uso, políticas de privacidad y cláusulas de limitación de responsabilidad dentro de la app y la página web.
\end{itemize}

\subsubsection*{5. Riesgos de propiedad intelectual}

\begin{itemize}
    \item Uso de tecnologías, librerías o componentes preexistentes sin licencias adecuadas.
    \item No proteger el nombre \emph{Equilibrado}, el logotipo o el diseño de la interfaz, permitiendo que terceros se apropien de elementos clave de la marca.
\end{itemize}

En conclusión, el estudio legal muestra que el proyecto es viable dentro del marco normativo colombiano, siempre que se gestionen de forma proactiva los registros, contratos y obligaciones descritos. Esto reduce la exposición a sanciones y fortalece la credibilidad de \emph{Equilibrado} frente a usuarios, aliados e inversionistas.```
::contentReference[oaicite:0]{index=0}
