\section{Microentorno - Mercado, FO-DA y CAME}
\subsection{Tamaño y demanda}
TAM Colombia $\sim$ USD 500M; SAM urbano $\sim$ USD 180M; SOM a 5 años $\sim$ USD 10M. Meta: 10\,000 usuarios (24m), crecimiento por alianzas EPS y clínicas.

\subsection{Competencia (oferta)}
\begin{itemize}
\item \textbf{Apps solo software:} sin métricas físicas ni adherencia sostenida.
\item \textbf{Plataformas clínicas premium:} $>\$5\,000$, orientadas a hospitales.
\item \textbf{Wearables genéricos:} no miden sway ni entrenan reflejos guiados.
\end{itemize}

\subsection{FO-DA (síntesis)}
\textbf{F:} costo bajo, modularidad HW, analítica clínica, UX lúdica. 
\textbf{O:} envejecimiento, EPS buscando reducción de siniestralidad, tele-rehab en auge. 
\textbf{D:} certificaciones, confianza en datos, adopción tecnológica en seniors. 
\textbf{A:} entrada de grandes marcas; cambios regulatorios; competidores con bundles.

\subsection{CAME}
\textbf{Corregir} (D): acompañamiento y tutoriales senior-friendly; evidencia clínica local.\\
\textbf{Afrontar} (A): compliance y seguridad; contratos marco con EPS.\\
\textbf{Mantener} (F): bajo costo, gamificación, KPIs clínicos.\\
\textbf{Explotar} (O): programas municipales y capitas EPS.

\subsection{Demanda cuantificada}
Escenarios: base, optimista, conservador. Definir ventas mensuales por canal (B2C, clínicas, EPS) y elasticidad precio-adopción.
