% ==========================================
% Chapters/14. Estudio_Financiero.tex
% ==========================================
\section{Estudio financiero del proyecto}

En este capítulo se evalúa la viabilidad financiera de \emph{Equilibrado} a partir de los
supuestos de crecimiento de usuarios, estructura de costos e inversiones descritos en
los estudios previos. El detalle numérico se presenta en el archivo de Excel adjunto,
mientras que aquí se resumen los principales resultados e interpretaciones.

\subsection{Supuestos del modelo financiero}

\subsubsection*{Crecimiento de servicios y comportamiento de precios}

El modelo considera los siguientes supuestos base:

\begin{itemize}
    \item \textbf{Crecimiento de servicios (usuarios/suscripciones):} incremento de la demanda de
    $6\%$ en el año 2, $7\%$ en el año 3, \(8\%\) en el año 4 y \(9\%\) en el año 5.
    \item \textbf{Inflación y costos:} los costos operativos crecen a una tasa de
    \(\text{inflación} + 3\) puntos porcentuales. En el modelo se parametriza con
    incrementos anuales que oscilan entre \(5{,}5\%\) y \(6{,}4\%\).
    \item \textbf{Precio del servicio:} el precio de la suscripción sólo se ajusta con la
    inflación, sin recargos adicionales por margen en los primeros años, dado que el
    objetivo es consolidar la base de usuarios.
    \item \textbf{Precio base de la suscripción:} de acuerdo con el estudio de costos, el
    precio anual inicial se estima en \(\$1{,}091{,}815\) por suscripción.
    \item \textbf{Escala de usuarios:} se plantea una entrada conservadora al mercado. En el
    año 1 se supone una base moderada de usuarios, con crecimiento según los
    porcentajes anteriores. En términos de la modelación de flujos, esto se representa
    como un número de servicios adquiridos que crece de forma progresiva.
\end{itemize}

Dado que una parte importante de la estructura de costos es fija (infraestructura
tecnológica, nómina, arriendo, etc.), a medida que aumenta la cantidad de usuarios el
costo medio por suscripción disminuye, lo que hace al modelo especialmente atractivo
para esquemas de contratación con EPS o entidades públicas.

\subsubsection*{Supuestos de impuestos y estructura de capital}

\begin{itemize}
    \item \textbf{Tasa de impuesto de renta:} \(30\%\) sobre la utilidad operacional.
    \item \textbf{Horizonte de evaluación:} 5 años.
    \item \textbf{Inversión inicial:} \(\$208{,}071{,}268\), que incluye activos fijos,
    activos intangibles, capital de trabajo y costos legales.
    \item \textbf{Financiación:} aproximadamente \(\$70{,}0\) millones de la inversión
    inicial se financian mediante un préstamo bancario y el resto proviene de recursos
    propios de los socios.
\end{itemize}

\subsection{Proyección de ingresos y costos}

La Tabla~\ref{tab:ingresos-costos} resume los ingresos, costos y utilidades proyectadas
para los cinco años de análisis (valores aproximados, en pesos colombianos).

\begin{table}[H]
\centering
\small
\renewcommand{\arraystretch}{1.15}
\begin{tabular}{c r r r}
\toprule
\textbf{Año} &
\textbf{Ingresos totales} &
\textbf{Costos operacionales totales} &
\textbf{Utilidad operativa} \\
\midrule
1 & \$575\,986\,962 & \$566\,986\,271 & \$9\,000\,691 \\
2 & \$644\,187\,274 & \$598\,227\,215 & \$45\,960\,059 \\
3 & \$727\,259\,732 & \$631\,189\,534 & \$96\,070\,198 \\
4 & \$828\,718\,283 & \$665\,968\,077 & \$162\,750\,205 \\
5 & \$953\,074\,920 & \$702\,662\,918 & \$250\,412\,001 \\
\bottomrule
\end{tabular}
\caption{Resumen de ingresos, costos y utilidad operativa proyectada.}
\label{tab:ingresos-costos}
\end{table}

A partir de estas utilidades se calcula la \textbf{utilidad operativa después de
impuestos} (UODI), sobre la cual se construyen los \emph{flujos de caja de la
operación} añadiendo nuevamente los cargos no desembolsables por depreciación y
amortización.

\subsection{Flujos de caja del proyecto}

\subsubsection*{Flujo de caja operativo}

El flujo de caja operativo (FCO) resulta de sumar a la UODI la depreciación y
amortización de activos. En el modelo se obtiene el siguiente perfil:

\begin{table}[H]
\centering
\small
\renewcommand{\arraystretch}{1.15}
\begin{tabular}{c r}
\toprule
\textbf{Año} & \textbf{Flujo de caja operativo (FCO)} \\
\midrule
0 & \$0 \\
1 & \(-\$519\,517\) \\
2 & \$25\,352\,042 \\
3 & \$60\,429\,139 \\
4 & \$107\,105\,144 \\
5 & \$168\,468\,401 \\
\bottomrule
\end{tabular}
\caption{Flujo de caja operativo proyectado.}
\label{tab:fco}
\end{table}

El valor ligeramente negativo del año 1 refleja la etapa de puesta en marcha y
consolidación comercial.

\subsubsection*{Flujo de caja de la inversión}

El \textbf{flujo de caja de la inversión} incluye:

\begin{itemize}
    \item Inversión en activos fijos (\(\sim\$37{,}6\) millones).
    \item Inversión en activos intangibles (\(\sim\$25\) millones).
    \item Costos legales y de constitución (\(\sim\$3{,}7\) millones).
    \item Inversión en capital de trabajo inicial y sus variaciones año a año.
    \item Valor de desecho de activos al finalizar el año 5.
\end{itemize}

El resultado agregado por año se resume en la Tabla~\ref{tab:fci}.

\begin{table}[H]
\centering
\small
\renewcommand{\arraystretch}{1.15}
\begin{tabular}{c r}
\toprule
\textbf{Año} & \textbf{Flujo de caja de inversión (FCI)} \\
\midrule
0 & \(-\$208\,071\,268\) \\
1 & \(-\$7\,810\,236\) \\
2 & \(-\$8\,240\,580\) \\
3 & \(-\$8\,694\,636\) \\
4 & \(-\$9\,173\,710\) \\
5 & \$172\,165\,730 \\
\bottomrule
\end{tabular}
\caption{Flujo de caja de la inversión.}
\label{tab:fci}
\end{table}

\subsubsection*{Flujo de caja del proyecto}

El \textbf{flujo de caja libre del proyecto} se calcula como la suma de FCO y FCI:

\begin{table}[H]
\centering
\small
\renewcommand{\arraystretch}{1.15}
\begin{tabular}{c r}
\toprule
\textbf{Año} & \textbf{Flujo de caja del proyecto} \\
\midrule
0 & \(-\$208\,071\,268\) \\
1 & \(-\$8\,329\,752\) \\
2 & \$17\,111\,462 \\
3 & \$51\,734\,503 \\
4 & \$97\,931\,434 \\
5 & \$340\,634\,130 \\
\bottomrule
\end{tabular}
\caption{Flujo de caja libre del proyecto \emph{Equilibrado}.}
\label{tab:fcp}
\end{table}

Estos flujos permiten evaluar la conveniencia del proyecto antes de considerar la
estructura específica de financiación.

\subsection{Estructura de financiación y costo de capital}

\subsubsection*{Deuda bancaria}

Para financiar \emph{Equilibrado} se plantea un crédito con BBVA por un valor cercano a
\(\$70{,}9\) millones, con las siguientes condiciones:

\begin{itemize}
    \item \textbf{Monto del préstamo:} \(\$70{,}873{,}284\) aproximadamente.
    \item \textbf{Tasa de interés efectiva anual (E.A.):} \(19\%\).
    \item \textbf{Plazo:} 5 años, con cuota fija anual.
    \item \textbf{Cuota anual estimada:} \(\sim\$23{,}179{,}119\).
\end{itemize}

La tabla de amortización considera el beneficio tributario generado por los intereses
deducibles de renta.

\subsubsection*{Costo de capital (CAPM) y WACC}

El costo de los recursos propios (\(K_e\)) se estimó mediante el modelo CAPM:

\[
K_e = R_f + \beta \,(E_m - R_f),
\]

donde:

\begin{itemize}
    \item \(R_f\): tasa libre de riesgo (rendimiento de TES a un año, aproximadamente
    \(9\%\)).
    \item \(E_m\): rendimiento esperado del mercado (\(\sim 35\%\)).
    \item \(\beta\): medida del riesgo sistemático de la inversión, asumida en \(2\) para
    reflejar el alto riesgo de un emprendimiento tecnológico en etapa temprana.
\end{itemize}

Con estos valores se obtiene un \(K_e\) alto (del orden de \(60\%\) anual), coherente con
las expectativas de retorno para \emph{startups} de base tecnológica.

El \textbf{costo promedio ponderado de capital (WACC)} se calcula combinando la deuda y
el patrimonio:

\[
\text{WACC} = w_D\,K_d\,(1 - T) + w_E\,K_e,
\]

donde \(w_D\) y \(w_E\) son las participaciones de deuda y patrimonio en la estructura de
capital, \(K_d\) es el costo de la deuda antes de impuestos y \(T\) la tasa impositiva.
Con la proporción aproximada de \(34\%\) deuda y \(66\%\) recursos propios se obtiene
un WACC elevado (del orden de \(40{-}50\%\) anual). El detalle numérico se presenta en
el archivo de Excel.

\subsection{Indicadores de evaluación}

\subsubsection*{Valor Presente Neto (VPN) y TIR}

Usando el WACC estimado como tasa de descuento, el proyecto arroja los siguientes
indicadores:

\begin{table}[H]
\centering
\small
\renewcommand{\arraystretch}{1.15}
\begin{tabular}{l r}
\toprule
\textbf{Indicador} & \textbf{Valor} \\
\midrule
VPN del proyecto & \$127\,669\,212{,}9 \\
TIR del proyecto & \(21{,}38\%\) \\
VPN del inversionista & \$105\,613\,971{,}1 \\
TIR del inversionista & \(23{,}40\%\) \\
\bottomrule
\end{tabular}
\caption{Indicadores principales de rentabilidad.}
\label{tab:indicadores-financieros}
\end{table}

El \textbf{VPN positivo} indica que, bajo los supuestos de flujos y tasa de descuento
utilizados, el proyecto crea valor económico. La \textbf{TIR} se sitúa alrededor del
\(21{-}23\%\), lo cual es atractivo frente a inversiones tradicionales, aunque puede
resultar ajustado si los inversionistas exigen retornos \(\geq 25{-}30\%\) típicos de
negocios tecnológicos en etapa temprana.

\subsubsection*{Periodo de recuperación (Payback)}

El periodo de recuperación simple o \emph{Payback} se calcula a partir de los flujos de
caja del proyecto. La inversión inicial se recupera en aproximadamente:

\[
\text{PRI} \approx 4{,}15 \text{ años},
\]

es decir, \textbf{4 años, 1 mes y 17 días}. A partir de ese momento, los flujos de caja
adicionales representan beneficios netos para los inversionistas.

\subsubsection*{Flujo de caja del inversionista}

Al incorporar el efecto de la deuda (entrada de recursos en el año 0 y pago de cuotas e
intereses en los años siguientes), se obtiene el \textbf{flujo de caja del
inversionista}, que refleja lo que reciben efectivamente los socios:

\begin{table}[H]
\centering
\small
\renewcommand{\arraystretch}{1.15}
\begin{tabular}{c r}
\toprule
\textbf{Año} & \textbf{Flujo de caja del inversionista} \\
\midrule
0 & \(-\$137\,197\,984\) \\
1 & \(-\$27\,469\,095\) \\
2 & \(-\$2\,581\,533\) \\
3 & \$31\,382\,663 \\
4 & \$76\,795\,567 \\
5 & \$318\,565\,272 \\
\bottomrule
\end{tabular}
\caption{Flujo de caja del inversionista después de financiación.}
\label{tab:fci-inversionista}
\end{table}

Este flujo de caja es el insumo para calcular la TIR y el VPN desde la perspectiva de
los accionistas.

\subsection{Análisis e interpretación}

\subsubsection*{Fortalezas financieras del proyecto}

\begin{itemize}
    \item VPN positivo y TIR por encima de la rentabilidad de instrumentos tradicionales,
    lo que sugiere que el proyecto es financieramente atractivo bajo el escenario base.
    \item Alta proporción de costos fijos, lo que genera economías de escala a medida que
    se incrementa la base de usuarios.
    \item Posibilidad de mejorar los indicadores financieros si se cierran contratos
    B2B2C con EPS o entidades gubernamentales que aceleren la adopción.
\end{itemize}

\subsubsection*{Aspectos a vigilar}

\begin{itemize}
    \item El costo de capital estimado (\(K_e\) y WACC) es elevado, reflejando un riesgo
    significativo. Para que el proyecto sea competitivo frente a otras oportunidades de
    inversión de alto riesgo, sería deseable alcanzar TIR cercanas o superiores al
    \(25{-}30\%\).
    \item La sensibilidad del modelo a variables clave como la \textbf{tasa de adopción de
    usuarios}, el \textbf{precio promedio} y la \textbf{estructura de costos} sugiere
    que reducciones en costos fijos o incrementos moderados en el precio podrían mejorar
    significativamente el VPN.
    \item La dependencia de deuda bancaria al \(19\%\) E.A. obliga a mantener flujos
    operativos suficientes para cumplir cómodamente las cuotas sin comprometer la
    inversión en crecimiento.
\end{itemize}

\subsubsection*{Conclusión financiera}

En síntesis, la evaluación financiera muestra que \emph{Equilibrado} es \textbf{viable}
en el escenario base, con recuperación de la inversión en poco más de cuatro años y
un VPN positivo. No obstante, al tratarse de un emprendimiento tecnológico de alto
riesgo, los socios e inversionistas deben:

\begin{enumerate}
    \item Buscar estrategias que incrementen el número de usuarios (alianzas con EPS,
    municipios y programas de envejecimiento activo).
    \item Optimizar la estructura de costos, especialmente la nómina y gastos fijos
    recurrentes.
    \item Revisar periódicamente los supuestos de crecimiento, inflación y tasas de
    interés para actualizar el modelo financiero.
\end{enumerate}

Con estas acciones, el proyecto puede fortalecer su atractivo financiero y alinearse
mejor con las expectativas de retorno de inversionistas especializados en soluciones
tecnológicas para el sector salud.
