% ==========================================
% Chapters/12. Estudio_Ambiental.tex
% ==========================================
\section{Estudio Ambiental}

\subsection{Identificación de posibles impactos ambientales}

Para el proyecto \emph{Equilibrado} se consideraron las etapas de desarrollo tecnológico, operación continua y actividades comerciales y de soporte. A partir de estas fases se identificaron los principales impactos ambientales, su tipo y una valoración cualitativa de su importancia.

\begin{table}[H]
\centering
\small
\renewcommand{\arraystretch}{1.2}
\begin{tabular}{p{3cm} p{3cm} p{4.2cm} p{3cm} p{2.2cm}}
\toprule
\textbf{Etapa del Proyecto} & \textbf{Actividad} & \textbf{Impacto Ambiental} & \textbf{Tipo} & \textbf{Valoración} \\
\midrule
Desarrollo tecnológico &
Uso intensivo de computadores &
Alto consumo energético &
Negativo -- Indirecto &
Bajo impacto \\
%
Desarrollo tecnológico &
Sustitución de hardware &
Generación de residuos electrónicos &
Negativo -- Indirecto &
Moderado \\
%
Operación continua &
Uso de hosting y servidores &
Emisión de CO\textsubscript{2} (huella digital) &
Negativo -- Indirecto &
Moderado \\
%
Actividades administrativas &
Consumo de papel, agua y energía &
Uso de recursos naturales &
Negativo -- Directo &
Moderado \\
%
Promoción y difusión &
Producción de materiales impresos (si se usan) &
Residuos sólidos &
Negativo -- Directo &
Bajo \\
%
Soporte al cliente &
Uso de plataformas de atención y CRM &
Aumento de tráfico digital &
Negativo -- Indirecto &
Bajo \\
%
Alternativa digital &
Reducción de desplazamientos físicos &
Disminución de emisiones y consumo de combustible &
Positivo -- Indirecto &
Alto impacto positivo \\
\bottomrule
\end{tabular}
\caption{Impactos ambientales identificados por etapa del proyecto.}
\label{tab:impactos-ambientales-equilibrio}
\end{table}

En términos generales, los impactos negativos se concentran en el consumo de energía y la generación de residuos electrónicos, mientras que el proyecto aporta beneficios ambientales al sustituir desplazamientos físicos por interacciones digitales.

\subsection{Descripción del ambiente afectado}

\subsubsection*{Ámbito del proyecto \emph{Equilibrado}}

\emph{Equilibrado} es una solución tecnológica orientada a la prevención de caídas en personas mayores, basada en una aplicación móvil, sensores y reportes automáticos. Su implementación inicial se plantea para zonas urbanas del Valle de Aburrá, especialmente Medellín, con potencial de expansión a otras regiones del país.

\paragraph{Entorno físico}
\begin{itemize}
    \item \textbf{Ubicación principal:} Medellín, Antioquia, en el área metropolitana.
    \item \textbf{Infraestructura asociada:} oficinas administrativas, estaciones de trabajo para el equipo de desarrollo y operación, y uso de centros de datos de terceros (servidores en la nube).
    \item \textbf{Ecosistema físico impactado:} muy bajo o prácticamente nulo, dado que las actividades se realizan en infraestructura urbana ya existente.
    \item \textbf{Requerimientos de espacio:} no se proyecta construcción de nuevas edificaciones; se utilizan oficinas y espacios de coworking arrendados.
\end{itemize}

\paragraph{Entorno social y humano}
\begin{itemize}
    \item \textbf{Población objetivo:} personas mayores de $60$ años y sus cuidadores o familiares.
    \item \textbf{Número potencial de usuarios:} más de $100\,000$ usuarios en la región a mediano plazo, considerando la población adulta mayor de Medellín y su área de influencia.
    \item \textbf{Beneficio directo:} mejora de la calidad de vida, reducción de riesgo de caídas, apoyo a la autonomía y mayor tranquilidad para familias y cuidadores.
    \item \textbf{Posible afectación:} no se identifican afectaciones negativas relevantes; el impacto social es principalmente positivo.
\end{itemize}

\paragraph{Entorno ambiental}
\begin{itemize}
    \item \textbf{Uso de recursos naturales:} indirecto, principalmente asociado al consumo de energía eléctrica en oficinas y centros de datos.
    \item \textbf{Generación de residuos:} residuos de equipos de cómputo, dispositivos de prueba y periféricos cuando se renueven o descarten.
    \item \textbf{Consumo energético:} uso continuo de servidores en la nube y de dispositivos móviles en manos de usuarios y equipo técnico.
\end{itemize}

\begin{table}[H]
\centering
\small
\renewcommand{\arraystretch}{1.2}
\begin{tabular}{p{3.5cm} p{3.5cm} p{3.5cm}}
\toprule
\textbf{Componente} & \textbf{Nivel de afectación} & \textbf{Tipo de impacto} \\
\midrule
Físico & Bajo / Nulo & Indirecto \\
Biótico & Nulo & Ninguno \\
Socioeconómico & Positivo & Directo \\
Cultural & Positivo & Directo \\
\bottomrule
\end{tabular}
\caption{Resumen del ambiente potencialmente afectado por el proyecto.}
\label{tab:resumen-ambiente-equilibrio}
\end{table}

En síntesis, la intervención se concentra en el entorno urbano y digital, con impactos físicos limitados y un efecto social favorable al promover envejecimiento activo y prevención.

\subsection{Plan de Manejo Ambiental (PMA)}

El objetivo del PMA es definir acciones de prevención, mitigación, corrección y compensación para los impactos ambientales asociados al ciclo de vida de \emph{Equilibrado}, manteniendo el enfoque de sostenibilidad y uso responsable de recursos.

\subsubsection*{Medidas de prevención}

\begin{table}[H]
\centering
\small
\renewcommand{\arraystretch}{1.2}
\begin{tabular}{p{4.2cm} p{8.3cm}}
\toprule
\textbf{Impacto potencial} & \textbf{Medida preventiva} \\
\midrule
Consumo de energía en servidores y dispositivos &
Seleccionar proveedores de nube con políticas de eficiencia energética y, cuando sea posible, con energía renovable; configurar servidores con escalado automático para evitar sobrecarga innecesaria. \\
%
Generación de residuos electrónicos &
Definir ciclos de renovación responsables, priorizar la reparación sobre el reemplazo y vincularse a programas de recolección y reciclaje de RAEE. \\
%
Saturación de datos personales sin control &
Implementar buenas prácticas de seguridad y privacidad digital (políticas claras de retención, anonimización de datos clínicos y cumplimiento de normativas de protección de datos). \\
\bottomrule
\end{tabular}
\caption{Medidas preventivas del PMA.}
\label{tab:prevencion-pma}
\end{table}

\subsubsection*{Medidas de mitigación}

\begin{table}[H]
\centering
\small
\renewcommand{\arraystretch}{1.2}
\begin{tabular}{p{4.2cm} p{8.3cm}}
\toprule
\textbf{Impacto potencial} & \textbf{Medida de mitigación} \\
\midrule
Uso constante de dispositivos móviles &
Desarrollar una interfaz que favorezca sesiones cortas y eficientes; incluir recomendaciones de pausas activas y de uso responsable de la tecnología. \\
%
Posible exclusión de usuarios sin acceso tecnológico &
Diseñar versiones \emph{lite} para dispositivos de gama baja y evaluar esquemas de préstamo de dispositivos en alianza con EPS o programas públicos. \\
%
Brecha digital en adultos mayores y cuidadores &
Realizar talleres básicos de alfabetización digital y videos explicativos dentro de la app, en lenguaje sencillo y con soporte telefónico. \\
\bottomrule
\end{tabular}
\caption{Medidas de mitigación de impactos.}
\label{tab:mitigacion-pma}
\end{table}

\subsubsection*{Medidas de corrección}

\begin{table}[H]
\centering
\small
\renewcommand{\arraystretch}{1.2}
\begin{tabular}{p{4.2cm} p{8.3cm}}
\toprule
\textbf{Impacto detectado} & \textbf{Medida correctiva} \\
\midrule
Fallos técnicos que generen mal funcionamiento o frustración en el usuario &
Habilitar canales de soporte técnico accesible (línea telefónica y chat), registrar incidencias y aplicar ciclos de mejora continua en el software. \\
%
Desactualización tecnológica de la solución &
Diseñar un plan de actualizaciones periódicas, tanto de la app como del backend, y programar revisión anual de la arquitectura técnica. \\
\bottomrule
\end{tabular}
\caption{Medidas correctivas previstas.}
\label{tab:correccion-pma}
\end{table}

\subsection{Análisis de costos ambientales}

El análisis de costos ambientales busca identificar y estimar los recursos económicos asociados a las medidas del PMA en las diferentes fases del proyecto. En el caso de \emph{Equilibrado}, estos costos se relacionan con el diseño responsable de la solución, la operación en la nube y las acciones de compensación a mediano y largo plazo.

\begin{table}[H]
\centering
\small
\renewcommand{\arraystretch}{1.2}
\begin{tabular}{p{3cm} p{5.5cm} p{3cm} p{2.7cm}}
\toprule
\textbf{Fase del proyecto} & \textbf{Actividad o necesidad ambiental} & \textbf{Tipo de costo} & \textbf{Valor estimado (COP)} \\
\midrule
Construcción (diseño y desarrollo) &
Diseño eficiente y ecológico de la app (optimización de datos y energía) &
Prevención &
\$3.000.000 \\
%
Construcción &
Formación del equipo en prácticas de ecodiseño digital &
Prevención / Capacitación &
\$1.500.000 \\
%
Operación &
Uso de hosting en servidores sostenibles o carbono neutral &
Mitigación &
\$5.000.000/año \\
%
Operación &
Desarrollo de material educativo sobre uso responsable de la app &
Mitigación &
\$2.000.000 \\
%
Operación &
Soporte técnico para resolver fallas rápidamente &
Corrección &
\$3.000.000 \\
%
Largo plazo &
Campañas de reciclaje de dispositivos con aliados estratégicos &
Compensación &
\$4.000.000 \\
%
Largo plazo &
Donaciones o alianzas en programas de reforestación / compensación de huella digital &
Compensación &
\$2.500.000 \\
\bottomrule
\end{tabular}
\caption{Costos ambientales estimados por fase del proyecto.}
\label{tab:costos-ambientales-equilibrio}
\end{table}

Estos valores sirven como referencia para incorporar los costos ambientales dentro del presupuesto total del proyecto y asegurar que las acciones de sostenibilidad estén financiadas desde la fase de formulación.

\subsection{Plan de seguimiento y monitoreo ambiental}

El objetivo del plan de seguimiento es verificar que las medidas definidas en el PMA se cumplan y que los impactos ambientales se mantengan dentro de niveles aceptables, introduciendo mejoras cuando sea necesario.

\subsubsection*{Fases y aspectos a monitorear}

\begin{table}[H]
\centering
\small
\renewcommand{\arraystretch}{1.2}
\begin{tabular}{p{2.5cm} p{4.3cm} p{2.3cm} p{4.1cm} p{2.5cm}}
\toprule
\textbf{Fase} & \textbf{Aspecto a monitorear} & \textbf{Frecuencia} & \textbf{Método de verificación} & \textbf{Responsable} \\
\midrule
Desarrollo &
Consumo energético del desarrollo tecnológico &
Mensual durante el diseño &
Medición del consumo en kWh en equipos y oficina &
Coordinador técnico \\
%
Desarrollo &
Cumplimiento de prácticas de ecodiseño y eficiencia &
Trimestral &
Listas de chequeo y auditoría interna del código y la infraestructura &
Líder ambiental \\
%
Operación &
Uso energético de servidores y servicios de nube &
Semestral &
Revisión de facturas, métricas de consumo del proveedor de hosting y reportes de eficiencia &
Administrador TI \\
%
Operación &
Contenido educativo actualizado en la app &
Trimestral &
Verificación de módulos en línea y registro de contenidos nuevos &
Área pedagógica / comunicaciones \\
%
Operación &
Gestión de fallas técnicas y tiempos de respuesta &
Continuo &
Reportes del sistema de soporte y tiempo promedio de resolución &
Equipo de soporte técnico \\
%
Operación / largo plazo &
Actividades de compensación implementadas (reciclaje, campañas, alianzas) &
Anual &
Convenios firmados, informes de impacto y evidencias fotográficas o documentales &
Coordinador de alianzas \\
%
Operación &
Retroalimentación de usuarios sobre usabilidad y percepción ambiental &
Semestral &
Encuestas o formularios dentro de la app y entrevistas breves &
Área de comunicaciones \\
\bottomrule
\end{tabular}
\caption{Plan de seguimiento y monitoreo ambiental.}
\label{tab:seguimiento-ambiental}
\end{table}

\subsubsection*{Indicadores clave de monitoreo}

\begin{table}[H]
\centering
\small
\renewcommand{\arraystretch}{1.2}
\begin{tabular}{p{7.2cm} p{4cm}}
\toprule
\textbf{Indicador} & \textbf{Meta anual} \\
\midrule
Porcentaje de cumplimiento de prácticas ecológicas definidas en el PMA &
$\geq 90\%$ \\
Reducción del consumo energético de oficinas y servidores frente al año anterior &
$\geq 10\%$ \\
Tasa de resolución de fallas técnicas en menos de $48$ horas &
$\geq 95\%$ \\
Número de contenidos educativos sobre uso responsable y sostenibilidad publicados en la app &
$\geq 6$ al año \\
Número de usuarios sensibilizados en sostenibilidad digital mediante la app o campañas &
$\geq 3\,000$ usuarios \\
\bottomrule
\end{tabular}
\caption{Indicadores de desempeño ambiental para \emph{Equilibrado}.}
\label{tab:indicadores-ambientales}
\end{table}

Los resultados de estos indicadores se revisarán semestralmente y permitirán ajustar tanto el PMA como las prácticas operativas del proyecto.

\subsection{Matriz de impactos ambientales y acciones}

Para integrar la información anterior se aplica una matriz de impactos donde se cruzan las etapas del proyecto con los factores ambientales afectados y las acciones planteadas.

\begin{table}[H]
\centering
\small
\renewcommand{\arraystretch}{1.2}
\begin{tabular}{p{1.8cm} p{2.5cm} p{3cm} p{1.8cm} p{1.8cm} p{3.5cm}}
\toprule
\textbf{Etapa del proyecto} & \textbf{Actividad} & \textbf{Factor ambiental afectado} & \textbf{Tipo de impacto} & \textbf{Valoración} & \textbf{Acción propuesta} \\
\midrule
Diseño / Desarrollo &
Uso intensivo de computadores y software &
Consumo energético / huella digital &
Negativo -- Indirecto &
Medio &
Aplicar criterios de ecodiseño, optimización de código y eficiencia energética en equipos. \\
%
Desarrollo &
Reemplazo de equipos de cómputo &
Residuos electrónicos (RAEE) &
Negativo -- Indirecto &
Bajo &
Renovación responsable, programas de reciclaje y acuerdos con gestores autorizados. \\
%
Operación &
Uso continuo de servidores y transmisión de datos &
Emisiones asociadas al consumo eléctrico &
Negativo -- Indirecto &
Medio &
Contratar hosting sostenible, monitorear consumo y ajustar capacidad según demanda real. \\
%
Comercialización &
Publicidad digital y posible impresión de materiales &
Consumo de recursos (papel, energía) &
Negativo -- Directo &
Bajo &
Priorizar campañas digitales, minimizar impresos y usar papel certificado cuando sea necesario. \\
%
Soporte técnico &
Uso de plataformas de atención y CRM &
Consumo de energía y recursos TI &
Negativo -- Indirecto &
Bajo &
Monitorear uso, consolidar herramientas y fomentar buenas prácticas de eficiencia. \\
%
Alternativa digital &
Reducción de desplazamientos físicos para consultas y seguimiento &
Emisiones evitadas por transporte &
Positivo -- Indirecto &
Alto &
Potenciar la estrategia digital, promover teleconsulta y seguimiento remoto mediante la app. \\
%
Dimensión social &
Accesibilidad y alfabetización digital para usuarios mayores &
Bienestar y equidad social &
Positivo -- Directo &
Alto &
Desarrollar módulos de capacitación y acompañamiento, garantizando inclusión tecnológica. \\
\bottomrule
\end{tabular}
\caption{Matriz de impactos ambientales y acciones asociadas.}
\label{tab:matriz-impacto-ambiental}
\end{table}

Con esta matriz se evidencia que los impactos negativos del proyecto son manejables mediante buenas prácticas de diseño y operación, mientras que los impactos positivos --especialmente en reducción de desplazamientos y bienestar social-- son significativos y coherentes con el enfoque de sostenibilidad del proyecto.
