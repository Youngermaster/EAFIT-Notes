% ==========================================
% Chapters/16. Conclusiones.tex
% ==========================================
\section{Conclusiones y recomendaciones finales}

\subsection{Análisis de viabilidad integral del proyecto}

A partir de los estudios sectorial--estratégico, de mercado, técnico, ambiental,
administrativo, legal y financiero, se pueden obtener las siguientes conclusiones
sobre la viabilidad de \emph{Equilibrado}:

\begin{enumerate}
    \item \textbf{Viabilidad estratégica y de entorno.}  
    El análisis PESTEL muestra un contexto favorable para soluciones de
    envejecimiento activo y telemonitoreo: envejecimiento acelerado de la población,
    incremento del gasto en salud, interés de gobiernos locales y EPS en prevención,
    y alta penetración de smartphones en la región.  
    Estos factores configuran un \emph{product--market fit} potencialmente sólido,
    especialmente bajo un modelo \emph{B2B2C} (EPS/IPS $\rightarrow$ familias).

    \item \textbf{Viabilidad de mercado.}  
    El tamaño del mercado en Medellín (más de 300~000 personas de 60+ y entre
    27~000--30~000 adultos mayores con alta prioridad por riesgo de caídas) y la
    disposición a pagar identificada en las encuestas sugieren una demanda suficiente
    para un piloto escalable.  
    La competencia presenta soluciones parciales (relojes inteligentes,
    colgantes SOS, plataformas de telerehabilitación de alto costo), lo que deja un
    espacio claro para una solución integrada como \emph{Equilibrado}. No obstante,
    la adopción en B2C puede ser lenta y dependerá de la percepción de valor,
    facilidad de uso y respaldo de entidades de salud.

    \item \textbf{Viabilidad técnica.}  
    Desde el punto de vista técnico, el proyecto es factible con tecnologías
    disponibles: sensores inerciales de bajo costo, desarrollo móvil y
    plataformas cloud. El estudio técnico plantea una capacidad de prestación de
    servicio escalable basada en infraestructura en la nube y un equipo de desarrollo
    compacto.  
    A futuro será clave profundizar en:
    \begin{itemize}
        \item la robustez de los algoritmos de detección de caídas y equilibrio;
        \item la interoperabilidad con historias clínicas electrónicas y plataformas
        de telemedicina;
        \item el diseño para fabricación y mantenimiento de los dispositivos
        físicos (wearables).
    \end{itemize}

    \item \textbf{Viabilidad ambiental.}  
    El proyecto presenta impactos ambientales directos relativamente bajos,
    concentrados en consumo energético (equipos y servidores) y generación de
    residuos electrónicos.  
    El Plan de Manejo Ambiental y el plan de seguimiento propuestos incluyen
    medidas razonables de prevención, mitigación y compensación (ecodiseño, hosting
    sostenible, campañas de reciclaje, contenidos educativos), por lo que la
    viabilidad ambiental es \textbf{alta}, siempre que se implementen los
    mecanismos de monitoreo planteados.

    \item \textbf{Viabilidad legal y regulatoria.}  
    La figura societaria de S.A.S. es adecuada para un emprendimiento tecnológico
    escalable y permite la entrada de inversionistas.  
    Existen, sin embargo, \textbf{puntos críticos de cumplimiento}:
    \begin{itemize}
        \item confirmación del nivel de riesgo del dispositivo y necesidad de
        registro sanitario ante Invima;
        \item implementación estricta de la normatividad de protección de datos
        personales (Ley 1581 de 2012 y normas complementarias);
        \item registro de marca y protección del software como activos
        estratégicos de propiedad intelectual.
    \end{itemize}
    Con una gestión jurídica adecuada, estos requisitos son abordables, por lo que la
    viabilidad legal es \textbf{media--alta}, condicionada a gestionar de forma
    temprana la ruta regulatoria de dispositivo médico.

    \item \textbf{Viabilidad organizacional y administrativa.}  
    La estructura organizacional definida (dirección, tecnología, marketing,
    administración y soporte) es coherente con el tamaño del proyecto y con el
    modelo de negocio propuesto.  
    El dimensionamiento del equipo, los perfiles y salarios están alineados con el
    estudio técnico y financiero, aunque la carga gerencial y técnica será alta en
    las primeras etapas. Es recomendable revisar periódicamente el organigrama a
    medida que aumente la base de usuarios y se generen nuevas líneas de servicio.

    \item \textbf{Viabilidad financiera.}  
    Bajo las hipótesis planteadas (10~000 usuarios el primer año, crecimiento
    moderado de suscripciones y una estructura de costos dominada por costos fijos),
    el proyecto arroja un VPN positivo y una TIR aproximada de \(21{,}38\%\), con
    recuperación de la inversión en alrededor de 4 años.  
    Esto indica que el proyecto \textbf{es rentable} frente a la tasa de descuento
    utilizada en el ejercicio, pero al compararlo con el costo de oportunidad típico
    de inversiones tecnológicas (rendimientos deseados del \(25\%\)-\(30\%\))
    la rentabilidad se sitúa en un nivel \emph{moderado}.  
    Adicionalmente, la estructura de financiamiento (crédito bancario
    \(\approx 70\) millones y capital propio) implica un riesgo financiero que debe
    gestionarse con especial cuidado en los primeros años.
\end{enumerate}

En síntesis, \emph{Equilibrado} es \textbf{estratégica, técnica, ambiental y
socialmente viable}, y \textbf{financieramente aceptable} bajo las hipótesis actuales,
aunque con margen de mejora para alcanzar los retornos que usualmente se esperan de un
emprendimiento de base tecnológica.

\subsection{¿Qué quedó faltando y qué se debería evaluar diferente?}

Si el proyecto fuera a implementarse en la realidad, sería recomendable complementar
el trabajo realizado con los siguientes elementos:

\begin{enumerate}
    \item \textbf{Validación clínica y de impacto en salud.}  
    Realizar estudios piloto con EPS o centros de rehabilitación que permitan medir
    de forma rigurosa:
    \begin{itemize}
        \item reducción de la frecuencia de caídas;
        \item mejora en indicadores de equilibrio y movilidad;
        \item impacto en costos evitados (hospitalizaciones, días de incapacidad,
        etc.).
    \end{itemize}
    Esta evidencia sería clave para fortalecer la propuesta de valor frente a
    pagadores (EPS, aseguradoras, entidades públicas).

    \item \textbf{Investigación de mercado más profunda.}  
    Ampliar las encuestas a otros municipios del Valle de Aburrá y a diferentes
    estratos socioeconómicos; realizar pruebas de \emph{pricing} más finas para
    validar elasticidad del precio y niveles de cobertura que EPS o gobiernos
    estarían dispuestos a financiar.

    \item \textbf{Detallamiento técnico del producto.}  
    Profundizar en:
    \begin{itemize}
        \item especificación de hardware (BOM), costos unitarios y plan de
        fabricación o ensamblaje;
        \item arquitectura de software y nube con mayor nivel de detalle (capas,
        servicios, esquemas de redundancia y seguridad);
        \item hojas de ruta de evolución del producto (módulos de IA, integración
        con otros dispositivos, nuevas métricas clínicas).
    \end{itemize}

    \item \textbf{Análisis financiero por escenarios.}  
    Incluir análisis de sensibilidad del VPN y la TIR frente a:
    \begin{itemize}
        \item variaciones en el número de usuarios (escenarios conservador, base y
        optimista);
        \item cambios en los costos de infraestructura cloud;
        \item distintas combinaciones de financiación (más capital propio, fondos
        de innovación, subvenciones, etc.).
    \end{itemize}
    Esto permitiría entender mejor la resiliencia del proyecto ante cambios de
    contexto y ajustar la estrategia comercial (por ejemplo, acelerar el componente
    B2B con EPS).

    \item \textbf{Plan de gestión del riesgo ampliado.}  
    Complementar los riesgos legales y financieros ya identificados con un mapa de
    riesgos operativos y tecnológicos (ciberataques, caídas de servicios cloud,
    falla de dispositivos en campo) y definir planes de contingencia específicos.

    \item \textbf{Profundización en la estrategia de impacto social.}  
    Diseñar métricas claras de impacto social (número de adultos mayores
    beneficiados, reducción de caídas reportadas, mejora en percepción de
    autonomía) e integrar estos indicadores en los reportes a EPS y entidades
    públicas. Esto podría abrir puertas a convocatorias de innovación social y
    financiación con enfoque de impacto.
\end{enumerate}

\subsection{Recomendaciones finales}

\begin{enumerate}
    \item \textbf{Iniciar con un piloto controlado B2B2C.}  
    La recomendación principal es comenzar con un piloto de escala limitada con una
    o dos EPS en Medellín, priorizando adultos mayores con alto riesgo de caídas.
    Esto permitirá ajustar el producto, validar supuestos de uso y generar evidencia
    clínica y económica.

    \item \textbf{Consolidar la ruta regulatoria y de datos desde el inicio.}  
    Es fundamental que, en paralelo al desarrollo tecnológico, se avance en la
    clarificación del estatus del dispositivo ante Invima y en la implementación de
    un sistema de gestión de datos personales acorde con la normatividad vigente.

    \item \textbf{Optimizar la estrategia financiera.}  
    Considerar fuentes de financiación complementarias al crédito bancario
    (fondos de emprendimiento, capital semilla, convocatorias de innovación en salud)
    que reduzcan el costo promedio de capital y mejoren la rentabilidad para los
    socios.

    \item \textbf{Fortalecer la experiencia de usuario y la alfabetización digital.}  
    Invertir en programas de capacitación para adultos mayores y cuidadores
    (talleres, tutoriales, acompañamiento inicial) con el fin de reducir la brecha
    digital y aumentar la adherencia al uso de la app y de los dispositivos.

    \item \textbf{Mantener la sostenibilidad ambiental como ventaja competitiva.}  
    Comunicar explícitamente las acciones de ecodiseño, uso responsable de energía y
    reciclaje de dispositivos como parte de la propuesta de valor, alineando el
    proyecto con tendencias de responsabilidad ambiental y criterios ESG.
\end{enumerate}

En conclusión, \emph{Equilibrado} es un proyecto con una base sólida y coherente en sus
distintos estudios. Aunque existen retos importantes en términos de regulación,
adopción tecnológica y rentabilidad exigida, las oportunidades de impacto social,
creación de valor para el sistema de salud y escalabilidad tecnológica justifican
continuar con su maduración, prototipado y búsqueda de aliados estratégicos para su
implementación real.
