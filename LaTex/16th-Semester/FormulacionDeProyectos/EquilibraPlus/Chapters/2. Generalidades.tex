\section{Generalidades}
\subsection{Antecedentes del proyecto}
El envejecimiento poblacional (Colombia $\sim$14\% mayores 60+, tendencia al 20\% en 2040) incrementa el riesgo de caídas (1 de cada 3 por año) y del deterioro cognitivo, generando altos costos sanitarios y pérdida de autonomía. La oferta actual es fragmentada: apps sin sensores (baja objetividad) o plataformas clínicas costosas y de baja accesibilidad domiciliaria.

\subsection{Concepción de la idea}
Equilibrado integra \textbf{dispositivos lúdicos asequibles} con \textbf{analítica clínica} y \textbf{experiencia social/gamificada} para elevar adherencia, permitir seguimiento remoto y generar evidencia objetiva para profesionales y aseguradoras.

\subsection{Grupos de interés}
Adultos mayores; cuidadores/familia; fisioterapeutas y clínicas; EPS/aseguradoras; gobiernos locales. 

\subsection{Problema central y propósito}
\textbf{Problema central:} Falta de herramientas y procesos efectivos para la prevención de caídas, con baja continuidad y escaso seguimiento de datos clínicos.\\
\textbf{Propósito:} Sistema integral y accesible que previene caídas y deterioro cognitivo, incrementa adherencia, habilita monitoreo continuo y reportes clínicos.

\subsection{Objetivos de negocio}
MAU $\geq$ 8\,000 al mes 18; 10\,000 activos y 50 clínicas en 24 meses; ARR $>$ USD 2M (año 2); RMS sway $\downarrow$ 15\% y tiempo de reacción $\uparrow$ 20\%.

