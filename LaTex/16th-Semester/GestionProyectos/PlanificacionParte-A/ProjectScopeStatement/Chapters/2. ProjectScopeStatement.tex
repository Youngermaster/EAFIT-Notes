% -----------------------------------------------------------------
% 2. Project Scope Statement
% -----------------------------------------------------------------
\section{Project Scope Statement}

\subsection{Datos generales}

\begin{tabularx}{\textwidth}{@{}lX@{}}
\textbf{Project Title:} & Hub Financiero \\
\textbf{Date Prepared:} & 19/08/2025 \\
\end{tabularx}

\subsection{Project Scope Description}

Construir y dotar un laboratorio financiero avanzado (hub financiero) que permita fortalecer la formación académica y práctica de los estudiantes, docentes y empresas en temas de finanzas, economía y mercados de capitales.

\subsection{Project Deliverables}

\begin{enumerate}
    \item \textbf{Infraestructura física:} construcción del nuevo bloque con aulas interactivas, salas de simulación, oficinas de soporte y espacios de coworking.
    \item \textbf{Dotación tecnológica:} estaciones de trabajo con software financiero especializado, pantallas de datos en tiempo real, Bloomberg terminals (o similares), equipos de videoconferencia y servidores.
    \item \textbf{Ambiente de simulación:} integración de plataformas de trading y sistemas de análisis de mercados para prácticas de la vida real en comparación al mercado.
    \item \textbf{Áreas de apoyo:} zonas de estudio colaborativo, espacios de descanso, baños.
    \item \textbf{Documentación del proyecto:} manuales técnicos de equipos y protocolos de seguridad tecnológica y estructural.
\end{enumerate}

\subsection{Product Acceptance Criteria}

\begin{enumerate}
    \item \textbf{Infraestructura física:} el lab financiero se debe entregar como un edificio construido completamente en el espacio donde era el bloque 2, con 3 salones de análisis de mercados en el primer piso, 2 salones grandes en el segundo piso para dictar clase y conferencias, y una terraza funcional en el tercer piso. Adicional, se deben entregar los baños y áreas de apoyo terminadas, equipadas y operativas.
    \item \textbf{Dotación:} cada salón debe contar con estaciones de trabajo equipadas con un software de finanzas (sea Reuters, Bloomberg, etc.), conectividad a internet, pantallas, sistema de audio y cámaras. Adicionalmente, las aulas del segundo piso deben estar dotadas con proyectores, televisores, pantallas interactivas y audio de alta calidad en funcionamiento.
    \item \textbf{Seguridad:} se debe entregar la documentación de las obras según los estándares de calidad y de seguridad ocupacional.
    \item \textbf{Emergencias:} el edificio deberá tener salidas de emergencia, señalización, extintores y protocolos de salida de emergencia.
\end{enumerate}

\subsection{Project Exclusions}

\begin{enumerate}
    \item El diseño arquitectónico y estructural (ya realizado en un proyecto anterior).
    \item Demolición del Bloque 3 (proyecto independiente y anterior).
    \item Desmontaje del actual laboratorio financiero (este será responsabilidad de otra área y proyecto).
    \item Mantenimiento posterior a la entrega (preventivo o correctivo).
    \item Actualizaciones futuras de hardware, software o mobiliario de las salas.
    \item Contratación de personal académico o administrativo para operar el hub.
    \item Servicios de cafetería (ni construcción de cafetería, ni servicios de venta de alimentos o bebidas).
    \item Remodelación o adecuación de otros bloques fuera del Bloque 3.
\end{enumerate}

\subsection{Project Constraints}

\begin{enumerate}
    \item \textbf{Tiempo:} el proyecto del hub financiero se debe completar en un plazo máximo de 16 meses desde la fecha en la que se apruebe.
    \item \textbf{Presupuesto o costo:} no podrá superar el monto de 15{,}000 millones de pesos (incluyendo la construcción, dotación de tecnología, mobiliario y un colchón para imprevistos).
    \item \textbf{Calidad:} debe cumplir estándares de seguridad y nacionales para estructura, instalaciones eléctricas y datos.
    \item \textbf{Regulatorio/ambiental:} debe contar con licencia ambiental para la construcción y otros permisos aprobados por el ANLA o la CAR.
\end{enumerate}

\subsection{Project Assumptions}

Para este proyecto se asume que:

\begin{enumerate}
    \item Tenemos el terreno disponible y con el Bloque 3 ya demolido. Se asume que la demolición finalizará a tiempo para que se empiece la construcción sin retrasos.
    \item Se asume que tenemos todos los permisos y licencias para construir en el espacio del Bloque 3 y que todo estará aprobado antes de iniciar la obra.
    \item Asumimos que contamos con los recursos presupuestados aprobados y que los mismos estarán disponibles durante el proceso de construcción y adecuación.
    \item Se asume que los proveedores, tanto de tecnología como de mobiliario, cumplirán con los tiempos pactados para entregar todo bajo los estándares de calidad necesarios.
    \item Se asume que tendremos el personal calificado como ingenieros civiles, arquitectos, técnicos y docentes asesores.
\end{enumerate}
