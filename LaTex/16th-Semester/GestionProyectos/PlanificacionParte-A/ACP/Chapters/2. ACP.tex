% -----------------------------------------------------------------
% 2. Acta de Constitución del Proyecto
% -----------------------------------------------------------------
\section{Acta de Constitución del Proyecto}

\subsection{Datos Generales}

\begin{tabularx}{\textwidth}{@{}lX@{}}
\textbf{Nombre del proyecto:} & Hub Financiero \\
\textbf{Fecha:} & 21/08/2025 \\
\textbf{Gerente:} & Pepito Pérez \\
\textbf{Sponsor:} & EAFIT \\
\end{tabularx}

\subsection{Propósito}

Construir y dotar un laboratorio financiero avanzado (hub financiero) que permita fortalecer la formación académica y práctica de los estudiantes, docentes y empresas en temas de finanzas, economía y mercados de capitales.

\subsection{Descripción de alto nivel del proyecto}

Construir un espacio de estudio interactivo y de experimentación para los estudiantes de la Escuela de Administración y Finanzas de la Universidad, mejorando así la infraestructura actual que no da abasto con la demanda de los estudiantes. El proyecto consta de 3 pisos destinados al laboratorio financiero y una terraza.  

El edificio contará con:
\begin{itemize}
    \item Aulas interactivas sobre mercados financieros.
    \item Salas de simulación de mercado.
    \item Espacios colaborativos y coworking.
    \item Salas de soporte y salas Bloomberg.
    \item Dotación tecnológica de última generación, mobiliario ergonómico y sistemas de conectividad.
\end{itemize}

\subsection{Límites}

\textbf{Incluye:}
\begin{itemize}
    \item Obras civiles: construcción de cimientos, estructuras, muros, pisos, techo, acabados y pintura del hub financiero.
    \item Instalaciones eléctricas, hidráulicas, ventilación, red de datos y seguridad física (cámaras, alarmas, control de acceso).
    \item Infraestructura interior: aulas interactivas, espacios de simulación, oficinas administrativas, coworking, zonas de estudio colaborativo y baños (hombres, mujeres y accesibles).
    \item Dotación tecnológica: estaciones de trabajo de alto rendimiento (AMD Ryzen 9, 32GB RAM, 512GB SSD, Windows 11, teclados Bloomberg), software especializado (Bloomberg, Reuters, Matlab, Excel, etc.), equipos de videoconferencia (audífonos Sony WH-1000XM6, cámaras Logitech Brio 500), pantallas digitales y tableros de datos en tiempo real.
    \item Mobiliario: tableros, sillas ergonómicas, escritorios, mesas redondas con sombrillas para terraza, baños completamente dotados.
\end{itemize}

\textbf{Excluye:}
\begin{itemize}
    \item Diseño arquitectónico y estructural (ya realizado en un proyecto anterior).
    \item Demolición del Bloque 3 (proyecto independiente).
    \item Desmontaje del actual laboratorio financiero.
    \item Mantenimiento posterior a la entrega.
    \item Actualizaciones futuras de hardware, software o mobiliario.
    \item Contratación de personal académico o administrativo.
    \item Servicios de cafetería.
    \item Remodelación o adecuación de otros bloques fuera del Bloque 3.
\end{itemize}

\subsection{Entregables Clave}

\begin{itemize}
    \item Construcción del bloque de 3 pisos.
    \item Instalaciones básicas (eléctricas, hidráulicas, de datos y seguridad).
    \item Dotación tecnológica de equipos de cómputo con software financiero y conexión a las redes de la Universidad.
    \item Dotación de mobiliario académico, laboratorios financieros y espacios colaborativos.
\end{itemize}

\subsection{Requerimientos de Alto Nivel}

\begin{itemize}
    \item El bloque debe cumplir con normativas de seguridad, construcción y accesibilidad.
    \item Capacidad suficiente para 250 estudiantes de manera simultánea.
    \item Salas de simulación y aulas con tecnología moderna.
    \item Red estable y rápida en todo el edificio.
    \item Mobiliario adecuado, ergonómico y flexible.
    \item Ambientes diseñados para investigación y simulación de mercados.
\end{itemize}

\subsection{Riesgos Generales}

\textbf{Políticos}
\begin{itemize}
    \item Debido a cambios en normativas de construcción o educación superior, puede ocurrir retraso en licencias o permisos, lo que provocaría demora en el inicio de la obra.
    \item Debido a inestabilidad política y políticas fiscales, puede ocurrir un aumento en impuestos, lo que provocaría mayores costos de ejecución.
\end{itemize}

\textbf{Económicos}
\begin{itemize}
    \item Debido a inflación y volatilidad de los mercados, puede ocurrir un aumento en precios de acero y materiales, lo que provocaría sobrecostos en obra civil.
    \item Debido a la fluctuación de la tasa de cambio, puede ocurrir un encarecimiento de equipos importados, lo que provocaría necesidad de ampliar el presupuesto.
    \item Debido a variación en tasas de financiación, puede ocurrir incremento en intereses de créditos, lo que provocaría presión financiera sobre el proyecto.
\end{itemize}

\textbf{Sociales}
\begin{itemize}
    \item Debido a resistencia cultural, puede ocurrir una baja adopción del hub, lo que provocaría menor aprovechamiento.
    \item Debido a la alta demanda de cupos, puede ocurrir saturación de la capacidad, lo que provocaría insatisfacción de usuarios.
\end{itemize}

\textbf{Tecnológicos}
\begin{itemize}
    \item Debido a retrasos en importación de hardware y software, puede ocurrir demora en la dotación, lo que provocaría incumplimiento de plazos.
    \item Debido a incompatibilidad entre software y la infraestructura de la universidad, puede ocurrir un fallo en integración tecnológica, lo que provocaría limitaciones de uso.
\end{itemize}

\textbf{Ambientales}
\begin{itemize}
    \item Debido a condiciones climáticas adversas, puede ocurrir interrupción de obra, lo que provocaría retrasos en el cronograma.
    \item Debido a mayores exigencias ambientales, puede ocurrir necesidad de implementar medidas adicionales, lo que provocaría incremento en costos.
    \item Debido al alto consumo energético, puede ocurrir un sobrecosto de operación, lo que provocaría mayores gastos recurrentes.
\end{itemize}

\textbf{Legales}
\begin{itemize}
    \item Debido al incumplimiento de normas de accesibilidad o seguridad, pueden ocurrir sanciones, lo que provocaría retrasos y sobrecostos.
    \item Debido a mal manejo de licencias de software, puede ocurrir sanción legal, lo que provocaría suspensión en uso de plataformas.
    \item Debido a accidentes laborales, pueden ocurrir demandas, lo que provocaría retrasos y costos adicionales.
\end{itemize}

\subsection{Metas Medibles con Criterios de Éxito}

\textbf{Alcance}
\begin{itemize}
    \item Entregar un bloque de 3 pisos completamente dotado con aulas, coworking, salas de simulación y Bloomberg.
    \item Al menos 40 estaciones de trabajo de alto rendimiento conectadas a la red EAFIT.
    \item Instalación de software especializado en el 100\% de estaciones.
\end{itemize}

\textbf{Tiempo}
\begin{itemize}
    \item Duración máxima: 16 meses desde la aprobación.
    \item Entrega parcial de obra gris en 8 meses.
    \item Instalación tecnológica y mobiliario en los últimos 4 meses.
\end{itemize}

\textbf{Costo}
\begin{itemize}
    \item Mantener el presupuesto aprobado (±10\% de tolerancia).
    \item Cumplir con hitos financieros trimestrales aprobados por el sponsor.
\end{itemize}

\textbf{Calidad}
\begin{itemize}
    \item Cumplimiento de normas de accesibilidad y seguridad.
    \item Certificación en seguridad eléctrica y de datos (ISO/IEC).
    \item Satisfacción $\geq$ 75\% de usuarios en encuesta piloto.
\end{itemize}

\textbf{Otros}
\begin{itemize}
    \item Uso inicial $\geq$ 70\% en el primer semestre.
    \item Incorporación de criterios de sostenibilidad.
\end{itemize}

\subsection{Objetivos Macro del Proyecto}

\begin{tabularx}{\textwidth}{|X|c|}
\hline
\textbf{Objetivo} & \textbf{Fecha de entrega} \\
\hline
Construcción del bloque de 3 pisos & Ago-26 \\
\hline
Instalación de servicios básicos & Dic-26 \\
\hline
Acabados del edificio & Ene-27 \\
\hline
Adecuación de interiores & Feb-27 \\
\hline
Dotación del laboratorio financiero y simuladores & May-27 \\
\hline
Instalación de mobiliario académico & Jul-27 \\
\hline
Entrega del bloque en funcionamiento & Dic-27 \\
\hline
\end{tabularx}

\subsection{Recursos Financieros Pre Aprobados}

De acuerdo con el flujo de caja y la disponibilidad presupuestal de la Universidad, se asignan recursos financieros pre aprobados equivalentes a 15{,}000 millones de pesos, incluyendo construcción, dotación tecnológica, mobiliario y un colchón para imprevistos.

\subsection{Interesados Clave}

\begin{tabularx}{\textwidth}{|X|X|}
\hline
\textbf{Interesado} & \textbf{Rol} \\
\hline
Universidad EAFIT / Rectora & Sponsor principal, aprueba el presupuesto y toma decisiones estratégicas. \\
\hline
Pepito Pérez / Project Manager & Responsable de la metodología y supervisión del proyecto. \\
\hline
Escuela de Administración y Finanzas & Usuario directo, define requerimientos y expectativas. \\
\hline
Estudiantes & Beneficiarios directos que usan el hub para formación práctica. \\
\hline
Profesores de finanzas, economía y negocios internacionales & Beneficiarios directos, principales usuarios en enseñanza. \\
\hline
\end{tabularx}

\subsection{Criterios de Éxito del Proyecto}

El proyecto se considerará exitoso si y solo si cumple con:
\begin{itemize}
    \item Entrega de los entregables definidos en este acta.
    \item Cumplimiento de inspecciones técnicas y pruebas de seguridad.
    \item Aprobación de calidad por parte del Sponsor y comité académico.
    \item Puesta en funcionamiento del hub dentro del plazo y presupuesto establecidos.
\end{itemize}

\subsection{Nivel de Autoridad del Gerente del Proyecto}

\begin{itemize}
    \item \textbf{Decisiones de equipo:} organiza y asigna tareas.
    \item \textbf{Decisiones financieras:} autoriza gastos dentro del presupuesto; cambios grandes deben ser aprobados por el sponsor.
    \item \textbf{Recursos:} administra y reasigna recursos.
    \item \textbf{Decisiones técnicas:} define soluciones alineadas a lineamientos.
    \item \textbf{Conflictos:} resuelve conflictos del equipo; los mayores se escalan al sponsor.
\end{itemize}

\subsection{Nivel de Autoridad del Sponsor}

El sponsor tiene autoridad para aprobar y modificar el presupuesto, validar el alcance y autorizar cambios de costo, tiempo, calidad, infraestructura, tecnología o mobiliario. Además, resuelve conflictos estratégicos o críticos como máxima instancia.

\subsection{Firmas}

\begin{tabularx}{\textwidth}{XX}
\includegraphics[height=2cm]{Figures/0. General/fake-signature.jpg} & \includegraphics[height=2cm]{Figures/0. General/fake-signature.jpg} \\
\centering \textbf{Gerente de Proyecto} & \centering \textbf{Sponsor} \\
\end{tabularx}
