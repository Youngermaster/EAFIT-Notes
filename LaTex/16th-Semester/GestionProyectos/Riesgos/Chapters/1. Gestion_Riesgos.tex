% Chapters/3. Gestion_Riesgos.tex
% =========================================================
% PLAN DE GESTIÓN DE RIESGOS - PROYECTO METRO DE LA 80
% =========================================================

\section{Plan de Gestión de Riesgos: Proyecto Metro de la 80}

\subsection{Contexto del proyecto}

El ``Metro de la 80'' (también llamado corredor de la 80 o metro ligero de la 80) es un sistema ferroviario de mediana capacidad que se construye en el occidente de Medellín como parte del Plan Maestro de Expansión del Metro de Medellín. 
El trazado proyectado recorre el eje de la Avenida 80 con múltiples estaciones (del orden de 14 a 17) y busca conectar desde el norte (cercanías de Caribe) hasta el sur (sector de Aguacatala) en aproximadamente media hora, integrándose tarifaria y operativamente con el sistema Metro y los buses alimentadores existentes. % :contentReference[oaicite:0]{index=0}

El esquema financiero es de cofinanciación entre el Gobierno Nacional y el Municipio/Metro de Medellín. 
Históricamente se ha planteado que la Nación aporte la mayor parte (del orden de la mitad o más del CAPEX) y el Municipio el resto mediante vigencias futuras locales, además de recursos de valorización y captura de plusvalía asociada al desarrollo urbano alrededor de las estaciones. % :contentReference[oaicite:1]{index=1}
En 2024 el Gobierno Nacional hizo giros importantes (cerca de \$0{,}48 billones COP) para asegurar flujo de caja del proyecto, reforzando el compromiso de financiación central. % :contentReference[oaicite:2]{index=2}

El corredor atraviesa una zona altamente urbanizada, con redes de servicios públicos de Empresas Públicas de Medellín (EPM), comercios formales e informales, y tráfico vehicular intenso. La construcción implica compra y reasentamiento de predios, afectaciones viales temporales, desvíos de tránsito y manejo de percepción ciudadana. % :contentReference[oaicite:3]{index=3}
Al mismo tiempo, el proyecto se percibe como una oportunidad de transformación social, mejora de movilidad sostenible y renovación urbana en el occidente de Medellín. % :contentReference[oaicite:4]{index=4}

Con base en las buenas prácticas del \textit{PMBOK Guide} del PMI \cite{PMBOK} y la literatura clásica de gestión de proyectos \cite{Kerzner2017}, a continuación se formula el Plan de Gestión de Riesgos solicitado.

\newpage
% =========================================================
% 1. IDENTIFICACIÓN DE RIESGOS
% =========================================================

\subsection{1. Identificación de los 10 riesgos principales}

Se identifican amenazas (riesgos negativos) y oportunidades (riesgos positivos). Cada riesgo está descrito de manera específica para el Metro de la 80 y hace referencia a costo, plazo, alcance, calidad, aceptación social o sostenibilidad.

\begin{table}[H]\centering\small
\begin{tabular}{p{1.2cm}p{3cm}p{8cm}p{2cm}}
\toprule
\textbf{ID} & \textbf{Nombre corto} & \textbf{Descripción} & \textbf{Tipo} \\
\midrule
R1  & Flujo de cofinanciación & Retrasos o reducción en el giro oportuno de recursos de la Nación / Municipio que limiten caja para obra física y adquisiciones mayores. Afecta cronograma y puede forzar créditos puente costosos. & Amenaza \\
R2  & Escalada de costos de obra & Aumento de precios de materiales críticos (acero, concreto, sistemas ferroviarios) por inflación e indexación contractual, lo que ya ha presionado el presupuesto del proyecto frente a estimaciones iniciales. % :contentReference[oaicite:5]{index=5}
& Amenaza \\
R3  & Gestión predial y reasentamientos & Retrasos o conflictos sociales en la compra de predios, traslado de comercios y reubicación de residentes del corredor de la Av.~80; riesgo de tutelas o acciones legales por compensación percibida como insuficiente. % :contentReference[oaicite:6]{index=6}
& Amenaza \\
R4  & Interferencia con redes de servicios públicos & Ductos de acueducto, alcantarillado, energía y telecom de EPM y terceros que cruzan el corredor. Rediseños, reprocesos o paradas de obra para relocalizar redes pueden generar costos indirectos y atrasos. & Amenaza \\
R5  & Movilidad y aceptación ciudadana durante obra & Cierres parciales de calzadas en la Av.~80 y desvíos de tránsito prolongados pueden generar congestión, quejas ciudadanas y presión política para frenar o limitar frentes de obra (situación común en proyectos de metro urbano en Colombia). % :contentReference[oaicite:7]{index=7}
& Amenaza \\
R6  & Permisos y licencias ambientales & Exigencias adicionales por ruido, vibración y calidad del aire por parte de autoridades ambientales metropolitanas o nacionales; riesgo de condicionamientos que obliguen a rediseñar soluciones constructivas o a limitar horarios. & Amenaza \\
R7  & Cambios políticos y de gobernanza & Cambio de administración local o nacional que altere prioridades, modifique el alcance técnico u operativo, o reabra discusiones de trazado / fases, generando incertidumbre y demoras en decisiones. & Amenaza \\
R8  & Integración multimodal efectiva & Oportunidad de integración tarifaria y operativa con la red Metro/Metroplús para capturar demanda y asegurar altos niveles de ocupación desde el día 1, mejorando el flujo de caja operativo futuro. % :contentReference[oaicite:8]{index=8}
& Oportunidad \\
R9  & Renovación urbana / captura de plusvalía & Potencial de valorización del suelo y densificación alrededor de estaciones (desarrollo orientado al transporte, TOD). Esto puede habilitar fuentes complementarias de financiación municipal mediante plusvalía / APPs inmobiliarias. & Oportunidad \\
R10 & Transferencia tecnológica local & Formación de capacidades locales en diseño, construcción y operación ferroviaria ligera (material rodante, señalización, mantenimiento). Disminuye dependencia externa en futuros proyectos y genera empleo calificado en Medellín. & Oportunidad \\
\bottomrule
\end{tabular}
\end{table}

\textbf{Nota:} Estos riesgos se derivan de factores técnicos (interferencia de redes), financieros (cofinanciación Nación--Municipio), socioambientales (reasentamientos, licencias), político-institucionales (gobernanza), y estratégicos (integración multimodal, captura de valor urbano).


% =========================================================
% 2. CLASIFICACIÓN RBS Y ASIGNACIÓN DE DUEÑOS
% =========================================================

\subsection{2. Clasificación en la RBS y asignación de dueño}

Para este taller se usa una \textbf{RBS (Risk Breakdown Structure)} de 2 niveles inspirada en \cite{PMBOK}:

\begin{itemize}
  \item \textbf{A. Externo / Político--Legal}
    \begin{itemize}
      \item A1. Financiero / Contractual con la Nación
      \item A2. Regulatorio / Licenciamiento / Ambiental
      \item A3. Gobernanza y voluntades políticas
    \end{itemize}
  \item \textbf{B. Técnico / Constructivo}
    \begin{itemize}
      \item B1. Ingeniería civil / redes de servicios públicos
      \item B2. Interferencia con movilidad y operación urbana existente
      \item B3. Tecnología ferroviaria / integración sistémica
    \end{itemize}
  \item \textbf{C. Socioeconómico / Predial / Urbano}
    \begin{itemize}
      \item C1. Gestión predial y reasentamientos
      \item C2. Aceptación ciudadana / reputación
      \item C3. Oportunidad de renovación urbana y plusvalía
    \end{itemize}
  \item \textbf{D. Operación y Largo Plazo}
    \begin{itemize}
      \item D1. Sostenibilidad financiera operativa
      \item D2. Desarrollo de capacidades locales
    \end{itemize}
\end{itemize}

El responsable (\textit{risk owner}) es la instancia que debe monitorear ese riesgo, preparar respuesta y reportar al Comité de Riesgos del proyecto.

\begin{table}[H]\centering\small
\begin{tabular}{p{1cm}p{4cm}p{4cm}p{5cm}}
\toprule
\textbf{ID} & \textbf{Categoría RBS} & \textbf{Dueño primario} & \textbf{Rol del dueño} \\
\midrule
R1 & A1 Externo--Financiero & Dirección Financiera del Proyecto / Alcaldía de Medellín & Garantizar flujo de cofinanciación Nación--Municipio; gestionar vigencias futuras y créditos puente. \\
R2 & B1 Técnico--Costos de obra & Gerencia de Costos \& Contratación / Interventoría & Control de costos, negociación de ajustes de precios y compras anticipadas de insumos críticos. \\
R3 & C1 Socioeconómico--Predial & Unidad de Gestión Predial y Social del Proyecto & Compra de predios, reasentamientos, compensaciones y manejo de conflictos comunitarios. \\
R4 & B1 Técnico--Redes de servicios & Mesa Técnica de Redes (Proyecto + EPM Infraestructura) & Coordinación de traslados de redes, cronograma por tramos, ventanas de intervención. \\
R5 & B2 Técnico--Movilidad urbana & Secretaría de Movilidad de Medellín + Comunicaciones del Proyecto & Plan de Manejo de Tráfico (PMT), comunicación ciudadana y mitigación de inconformidad pública. \\
R6 & A2 Externo--Ambiental & Equipo Ambiental del Proyecto / Autoridad Ambiental Metropolitana & Trámite de licencias y permisos, control de ruido/vibración, planes de manejo ambiental. \\
R7 & A3 Externo--Gobernanza & Gerencia General Metro de Medellín / Junta del Proyecto & Asegurar continuidad institucional, acuerdos interadministrativos y decisiones técnicas estables. \\
R8 & B3 Técnico--Integración operativa & Operación Metro de Medellín (Gerencia de Operaciones) & Asegurar integración tarifaria, interoperabilidad de recaudo y frecuencia de servicio desde el día 1. \\
R9 & C3 Socioeconómico--Renovación urbana & Secretaría de Planeación / Agencia de APP Municipales & Estructurar captura de plusvalía / proyectos TOD alrededor de estaciones sin desplazar población vulnerable. \\
R10& D2 Largo plazo--Capacidades & Gerencia de Tecnología del Metro + Universidades locales & Diseñar convenios de transferencia tecnológica y formación técnica especializada. \\
\bottomrule
\end{tabular}
\end{table}


% =========================================================
% 3. ANÁLISIS CUALITATIVO DE RIESGOS
% =========================================================

\subsection{3. Análisis cualitativo de riesgos}

El análisis cualitativo se hace siguiendo las recomendaciones del PMI: definir escalas de probabilidad e impacto comunes, estimar ambos por cada riesgo, y calcular una prioridad (score) multiplicando Probabilidad $\times$ Impacto (P$\times$I). \cite{PMBOK}

\subsubsection*{3.1 Escalas de probabilidad (P)}

Se adopta una escala ordinal de 1 a 5 (aplica tanto a amenazas como a oportunidades):

\begin{itemize}
  \item 1 = Muy baja ($<10\%$ de que ocurra)
  \item 2 = Baja (10--30\%)
  \item 3 = Media (31--60\%)
  \item 4 = Alta (61--80\%)
  \item 5 = Muy alta ($>80\%$)
\end{itemize}

\subsubsection*{3.2 Escalas de impacto (I)}

También se usa una escala 1 a 5. Para \textbf{amenazas} medimos impacto negativo (costo extra, retraso, reputación). Para \textbf{oportunidades} medimos impacto positivo (ahorro, aceleración, aceptación social, ingresos futuros). El criterio se inspira en prácticas de control de valor ganado y control de cambios de alcance \cite{PMBOK}:

\begin{itemize}
  \item 1 = Menor: efecto $<2\%$ del presupuesto o $<1$ mes de atraso/aceleración local
  \item 2 = Moderado: 2--5\% del presupuesto o 1--2 meses
  \item 3 = Significativo: 5--10\% del presupuesto o 2--4 meses
  \item 4 = Mayor: 10--20\% del presupuesto o 4--6 meses
  \item 5 = Crítico: $>20\%$ del presupuesto o $>6$ meses / riesgo de frenar obra
\end{itemize}

\subsubsection*{3.3 Matriz cualitativa P$\times$I}

Para \textbf{amenazas} se reporta el score como negativo (porque es daño); para \textbf{oportunidades} se reporta positivo (beneficio). Así priorizamos tanto lo que debemos \emph{mitigar/evitar} como lo que debemos \emph{potenciar/explotar}.

\begin{table}[H]\centering\small
\begin{tabular}{p{0.8cm}p{3cm}p{1.3cm}p{1.3cm}p{1.5cm}p{6cm}}
\toprule
\textbf{ID} & \textbf{Riesgo} & \textbf{P (1-5)} & \textbf{I (1-5)} & \textbf{Score P$\times$I} & \textbf{Notas cualitativas} \\
\midrule
R1 & Flujo de cofinanciación & 3 & 5 & $-15$ & Sin flujo oportuno de Nación/Municipio se frena frente de obra crítico; obliga a créditos puente caros. \\
R2 & Escalada de costos & 4 & 4 & $-16$ & Inflación en insumos de infraestructura de transporte ha presionado el CAPEX inicial, ya cercana a medio billón COP extra frente a estimaciones tempranas. \\
R3 & Gestión predial / reasentamientos & 3 & 4 & $-12$ & Riesgo social/jurídico si hay desacuerdo en avalúos o reasentamientos en Av.~80. \\
R4 & Redes de servicios públicos & 4 & 3 & $-12$ & Reubicación de redes EPM puede detener obra civil si no hay ventana técnica. \\
R5 & Movilidad / aceptación ciudadana & 4 & 3 & $-12$ & Cierres viales prolongados en corredores troncales han generado rechazo ciudadano en otros proyectos de metro urbano en Colombia; podría pasar en Av.~80. \\
R6 & Licencias ambientales & 2 & 3 & $-6$ & Condicionamientos ambientales (ruido, vibración nocturna) pueden limitar horarios y alargar plazo. \\
R7 & Cambios políticos & 3 & 4 & $-12$ & Cambio de administración podría ralentizar decisiones, renegociar alcance o ritmos de obra. \\
R8 & Integración multimodal & 4 & 3 & $+12$ & Si el sistema nace integrado al Metro y buses alimentadores, se garantiza alta demanda, legitimidad social y caja operativa temprana. \\
R9 & Renovación urbana / plusvalía & 3 & 4 & $+12$ & Densificación alrededor de estaciones puede generar nuevas rentas urbanas que ayuden a cerrar la brecha de financiación municipal. \\
R10& Transferencia tecnológica local & 3 & 3 & $+9$  & Fortalecer know-how ferroviario local reduce dependencia de proveedores externos y mejora sostenibilidad a largo plazo. \\
\bottomrule
\end{tabular}
\end{table}

\paragraph{Priorización.}
\begin{itemize}
  \item \textbf{Amenazas más críticas (por magnitud absoluta del score)}:
  R2 ($-16$), R1 ($-15$), R3 ($-12$). 
  \item \textbf{Oportunidades más relevantes}:
  R8 ($+12$), R9 ($+12$), R10 ($+9$).
\end{itemize}

Siguiendo la instrucción del enunciado, estos seis riesgos priorizados alimentan el punto 4 (plan de respuesta).


% =========================================================
% 4. PLANIFICACIÓN DE LA RESPUESTA A LOS RIESGOS
% =========================================================

\subsection{4. Plan de respuesta a los riesgos priorizados}

De acuerdo con el PMI \cite{PMBOK}, las tácticas típicas son:
\begin{itemize}
  \item Para \textbf{amenazas}: \textit{evitar}, \textit{mitigar}, \textit{transferir}, \textit{aceptar}.
  \item Para \textbf{oportunidades}: \textit{explotar}, \textit{mejorar}, \textit{compartir}, \textit{aceptar}.
\end{itemize}

Cada riesgo incluye:
(i) Estrategia propuesta,
(ii) Acción concreta,
(iii) Contingencia (costo y/o tiempo),
(iv) Riesgos secundarios potenciales.

\subsubsection*{4.1 Amenazas críticas}

\begin{table}[H]\centering\small
\begin{tabular}{p{1cm}p{2cm}p{1.5cm}p{3.5cm}p{3cm}p{3.5cm}}
\toprule
\textbf{ID} & \textbf{Riesgo} & \textbf{Estrategia (PMI)} & \textbf{Acción / Plan de respuesta} & \textbf{Contingencia} & \textbf{Riesgos secundarios} \\
\midrule
R2 & Escalada de costos de obra & Mitigar & 
Negociar cláusulas de ajuste de precios claras y techos máximos en contratos críticos; compras anticipadas de acero/banquetes de vía; uso de coberturas frente a TRM si hay importados. 
&
Reserva de contingencia de costos del orden del 10\% de insumos críticos (acero, sistemas ferroviarios). Ajustar cronograma de pagos para suavizar picos de caja. 
&
Sobrestock y costos de almacenamiento; riesgo financiero por comprar antes de necesitar. \\
\midrule
R1 & Flujo de cofinanciación & Transferir / Mitigar & 
Firmar convenios Nación--Municipio de vigencias futuras \emph{irrevocables}, crear una fiducia o cuenta maestra donde la Nación gire automáticamente y desde la cual se pagan hitos de obra. Estructurar línea de crédito puente con condiciones preferenciales en caso de atraso nacional.
&
Contingencia de caja: línea de liquidez equivalente a $\sim$3 meses de obra prioritaria (p.ej. 5\% del CAPEX anual), para no parar frente crítico. 
&
Costo financiero adicional (intereses); riesgo político si se percibe endeudamiento “innecesario”. \\
\midrule
R3 & Gestión predial / reasentamientos & Mitigar / Evitar & 
Implementar \textbf{Plan de Gestión Predial y Social} temprano: avalúos independientes, compensación justa y reasentamiento digno; mesas de concertación por tramo; acompañamiento jurídico y psicosocial a comerciantes/residentes. Comunicación transparente de cronogramas y beneficios locales (mejor acceso futuro). 
&
Presupuesto socio-predial adicional $\sim$2--3\% del CAPEX civil del tramo para compensaciones y apoyo al traslado. Incluir holgura de $\sim$2 meses por tramo en el cronograma de traslado predial antes de iniciar excavaciones pesadas. 
&
Riesgo de “efecto llamada”: aparición de ocupaciones informales buscando compensación; presión mediática sobre montos pagados. \\
\bottomrule
\end{tabular}
\end{table}

\subsubsection*{4.2 Oportunidades relevantes}

\begin{table}[H]\centering\small
\begin{tabular}{p{1cm}p{2cm}p{1.5cm}p{3.5cm}p{3cm}p{3.5cm}}
\toprule
\textbf{ID} & \textbf{Riesgo (oportunidad)} & \textbf{Estrategia (PMI)} & \textbf{Acción / Plan de respuesta} & \textbf{Contingencia / Inversión} & \textbf{Riesgos secundarios} \\
\midrule
R8 & Integración multimodal efectiva & Explotar & 
Diseñar, desde ahora, interoperabilidad tarifaria y física: estaciones de intercambio directo con Metro y buses, un único sistema de recaudo, frecuencias coordinadas, imagen de marca unificada. 
&
Inversión temprana en sistemas de recaudo integrados y señalización común (CAPEX TI y señalización $\sim$0{,}5--1\% adicional). 
&
Dependencia tecnológica de un único sistema de recaudo; riesgo de indisponibilidad si falla la plataforma común el día 1. \\
\midrule
R9 & Renovación urbana / captura de plusvalía & Compartir / Mejorar & 
Estructurar \textbf{proyectos TOD (Transit Oriented Development)} con privados alrededor de estaciones: usos mixtos, comercio formal, vivienda densa. Asociar instrumentos de plusvalía y valorización para financiar parte del CAPEX y del mantenimiento urbano del corredor. 
&
Requiere ajustar POT y expedir lineamientos urbanísticos estación-por-estación. Contingencia: fondo de mitigación social para evitar desplazamiento de población vulnerable (1--2\% adicional del presupuesto urbano local). 
&
Riesgo de gentrificación y presión sobre arriendos; conflicto social si se percibe expulsión de residentes históricos. \\
\midrule
R10 & Transferencia tecnológica local & Mejorar & 
Incluir en los contratos de suministro ferroviario cláusulas obligatorias de transferencia tecnológica, formación de personal local y alianzas con universidades de Medellín para operación y mantenimiento especializado. 
&
Asignar $\sim$0{,}5\% del CAPEX del sistema de vía/material rodante a programas de formación técnica y compra de herramientas/softwares. 
&
Riesgo de fuga de talento: personal formado puede migrar a otros proyectos ferroviarios nacionales/internacionales. \\
\bottomrule
\end{tabular}
\end{table}

\subsubsection*{4.3 Riesgos aceptados activamente}

Algunos riesgos se deciden \textbf{aceptar activamente} con reservas de contingencia explícitas:
\begin{itemize}
  \item \textbf{R2 (escalada de costos):} Aunque se mitiga con compras anticipadas y topes contractuales, se asume que siempre habrá presión inflacionaria en infraestructura masiva. Se deja una \emph{reserva de gestión} de al menos el 10\% de los insumos críticos para absorber picos de precios sin frenar obra.
  \item \textbf{R8 (integración multimodal):} Se asume como objetivo estratégico (oportunidad). Se “acepta” que invertir más hoy en integración tarifaria/operativa es costo hundido inicial, pero el beneficio social y reputacional del sistema entrando en operación ya integrado justifica esa inversión.
\end{itemize}

\subsubsection*{4.4 Síntesis ejecutiva}

\begin{itemize}
  \item Las \textbf{tres amenazas más severas} (R2, R1, R3) están ligadas a sobrecosto, flujo financiero y conflicto social. Las respuestas priorizan: asegurar caja estable (fiducia y crédito puente), blindar al proyecto frente a inflación de insumos, y gestionar el componente predial/social con compensaciones claras y cronogramas realistas.
  \item Las \textbf{tres oportunidades más fuertes} (R8, R9, R10) están alineadas con la sostenibilidad a largo plazo: legitimidad ciudadana desde la puesta en marcha (integración multimodal), captura de valor urbano para fondear el proyecto, y creación de capacidades ferroviarias locales.
  \item Se dejan \textbf{contingencias cuantificadas} (porcentaje del CAPEX, meses de holgura por tramo, fondos sociales específicos) y se identifican \textbf{riesgos secundarios} como costo financiero adicional, percepción de endeudamiento, gentrificación o dependencia tecnológica.
\end{itemize}

\textbf{Conclusión.}  
Este plan cumple con los lineamientos del \emph{PMBOK} \cite{PMBOK}: se identifican riesgos positivos y negativos, se clasifican en una RBS con dueños asignados, se realiza un análisis cualitativo con escalas de probabilidad e impacto para priorizar mediante P$\times$I, y se definen respuestas concretas (acción, contingencia y riesgos secundarios) para las amenazas y oportunidades más críticas del proyecto.
